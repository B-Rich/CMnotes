\chapter{Modeling classical systems}

We divide classical systems into two kinds: one is the systems with only one particle or object; the other one is the systems with many interacting particles. 

For the first kind of systems, we can either use physics laws such as Newton's laws or Lagrangian/Hamiltonian mechanical knowledge to establish a set of equations of motion of the object. 
The systems can usually be described by up-to second-order differential equations of displacement. Examples are given in Section~\ref{sec:equationofphysics} and Chapter~\ref{chap:dampledosc}. The basis of lagrangian and Hamiltonian mechanics is studied in Chapter~\ref{chap:lagham}. 

For the second kind of systems, the interactions between particles should be considered. Therefore, if the particles are discrete, we can conduct the modeling process similar to the sections from~\ref{sec:particles} to~\ref{sec:defects}; if the particles can be viewed as continua, as proved in Section~\ref{sec:field}, a field theory model can be obtained by reducing the distance of particles in a multi-particle model. Specifically, based on continuity equation and conductive relations, a fluid field can be simplified to typical field models described by diffusion/telegrapher's/wave/Fisher's equations. In most cases of this course, we make our equations dimensionless for simplicity.

As a special case of the second kind of systems, synchronization phenomenon is common in few-body systems (see Chapter.~\ref{chap:synchronization}). This model has its counterparts in two-level quantum systems. 

Notice that, for the first kind of systems, equations are usually derived with respect to displacement. But for the second kind of systems, especially for field problems, the equations are usually established with respect to wave/density/probability functions which is no longer a direct observable physics quantity. Usually, to model the second kind of systems, we need to know the interaction rules between elements and some extra boundary/natural conditions or relations to the variables. Finding a proper analyzable variable and completed relations is the key to model a complicated system. 

Once we obtain the equations governing the motion of systems, we can perform all sorts of skills to solve the mathimatical equations. Alternatively, we can also study some of the properties of the system by looking at the stability, singularity and solvability of the equations. One related theory called flow and bifurcation theory is introduced in Chapter~\ref{chap:bifurcation}. 

By solving the equations of classical mechanical systems, we can analogize the systems to basic physics models, including physical pendulum (see Chapter~\ref{chap:dampledosc}), wave propagator, diffusion (see Section~\ref{sec:field} and Chapter~\ref{chap:fluid}) and so on. In most cases, the system can be analyzed using Green function (in the $ x $- or $ k $-domain) or memory function (in the time-domain, see Chapter~\ref{chap:memoryfunc}) theory to treat the system as a propagator evolving from a initial point or time to any position and time points.  