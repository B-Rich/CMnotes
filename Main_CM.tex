\documentclass[12pt]{report}

\usepackage{pdfpages} % to include pdf documents
\usepackage{titlesec}
\titleformat{\chapter}
  {\normalfont\LARGE\bfseries}{\thechapter}{1em}{}
\titlespacing{\chapter}{0pt}{3.5ex plus 1ex minus .2ex}{2.3ex plus .2ex}
\setcounter{secnumdepth}{3}
\renewcommand\thesection{\arabic{chapter}.\arabic{section}}
\renewcommand\thesubsection{\Roman{subsection}}
\makeatletter
\renewcommand\p@subsection{\thesection.}
\makeatother


\usepackage{amsmath}
\usepackage{bm}
\usepackage{listings}
% % \textwidth 16cm \textheight 23.5cm
% \renewcommand{\baselinestretch}{1.2}
\usepackage{graphicx}
\usepackage{graphics}
\usepackage{epsfig}
\usepackage{color}
\usepackage{multirow}
\usepackage[colorlinks]{hyperref}
\usepackage{fancyhdr}
\usepackage{calc}
\usepackage[numbers]{natbib}
\usepackage{bibentry}
% underline tool
\usepackage[normalem]{ulem}
\usepackage{xcolor}
%\uline{foo}	Underlines foo
%\uuline{foo}	Double underlines foo
%\uwave{foo}	Underlines foo with a wavy line
%\sout{foo}	Strikesout foo
%\xout{foo}	Crosses out foo with ��/�6�7��
% triple lines in colors
\makeatletter
\newcommand\uuuline{\bgroup\markoverwith%
   {%
     \textcolor{red}{\rule[-0.5ex]{2pt}{0.4pt}}%
     \llap{\textcolor{blue}{\rule[-0.7ex]{2pt}{0.4pt}}}%
     \llap{\textcolor{green}{\rule[-0.9ex]{2pt}{0.4pt}}}%
   }%
   \ULon}
\makeatother


\usepackage{amsmath,soul} % underline with a number
% usage example: $\underset{4}{\text{\ul{This is short text}}}$
% another package to use, but did not work for me.
% From: http://tex.stackexchange.com/questions/45341/labeling-underlined-text-over-multiple-lines
%\usepackage{soulpos}
%\ulposdef{\ulnumaux}{%
%   $\underset{\saveulnum}{\rule[-.7ex]{\ulwidth}{.4pt}}$}
%
%\newcommand{\ulnum}[2]{%
%  \def\saveulnum{#1}%
%  \ulnumaux{#2}}

% todo list and commands
\usepackage{todonotes}
%% to avoid the conflict with amths package % not working
%\makeatletter
%\providecommand\@dotsep{5}
%\makeatother
%\listoftodos\relax
% my comment
\newcounter{mycomment}
\newcommand{\mycomment}[2][]{%
% initials of the author (optional) + note in the margin
\refstepcounter{mycomment}%
{%
\setstretch{0.7}% spacing
\todo[color={red!100!green!33},size=\small]{%
\textbf{[\uppercase{#1}\themycomment]:}~#2}%
}}
% colored todo comment
\newcommand{\todored}[2][]
{\todo[color=red, #1]{#2}}
\newcommand{\todogreen}[2][]
{\todo[color=green, #1]{#2}}
\newcommand{\todoblue}[2][]
{\todo[color=blue, #1]{#2}}

% cancel signs
\usepackage{cancel}
% empheq
\usepackage{empheq}



%%%%%%%%% use pdf and eps at the same time %%%%%%
%\newif\ifpdf
%\ifx\pdfoutput\undefined
%   \pdffalse
%\else
%   \pdfoutput=1
%   \pdftrue
%\fi
\usepackage[update,prepend]{epstopdf}
\usepackage{ifpdf}

\ifpdf
   \usepackage{graphicx}
   \usepackage{epstopdf}
   \epstopdfsetup{suffix=}
   \DeclareGraphicsRule{.eps}{pdf}{.pdf}{`epstopdf #1}
   \pdfcompresslevel=9
\else
   \usepackage{graphicx}
\fi

% packages from homework template
\usepackage{lastpage}
\usepackage{extramarks}
\usepackage{chngpage}
\usepackage{soul}
\usepackage{courier}

% page setting
\pagestyle{fancy}
\headheight = 15pt

\linespread{1.5}
\fancyhead[R]{\thepage}
\fancyfoot{}
\hoffset =-1 cm
\textwidth 424 pt
\renewcommand{\headrulewidth}{0.4pt}
\headwidth 424 pt
\parindent 1 cm
% \usepackage[numbers]{natbib}

% % % % % % My definitions of variables % % % % 
% self-definition for short hand

% symbols and math operators
\DeclareMathOperator{\spn}{span}
\DeclareMathOperator{\tr}{tr}
% definition of grammars % formats related
\definecolor{MyDarkGreen}{rgb}{0.0,0.4,0.0}
\newcommand{\greek}[1]{{\selectlanguage{greek}#1}} % will look for grmn font: tlmgr install cbfonts (65 MB)

% functions
\newcommand{\sn}{\mathrm{sn}}
\newcommand{\cn}{\mathrm{cn}}
\newcommand{\dn}{\mathrm{dn}}

% constants
\newcommand{\invtpi}{\frac{1}{2\pi}}

% vectors and tensors
\def\en{\mathbf{e}_n}
\def\eye{\mathbf{I}}
\newcommand\lvec[1]{\accentset{\leftarrow}{#1}}

% derivatives and integrals
% ordinary derivatives
\newcommand{\drv}{\mathrm{d}}
\newcommand{\dt}[1]{\frac{{\mathrm d} {#1}}{{\mathrm d}t}}
\newcommand{\dx}[1]{\frac{{\mathrm d} {#1}}{{\mathrm d}x}}
\newcommand{\dtau}{\frac{{\mathrm d} }{{\mathrm d}\tau}}
\newcommand{\dd}[2]{\frac{{\mathrm d} {#1}}{{\mathrm d} {#2}}}
\newcommand{\sdt}[1]{\frac{{\mathrm d}^2 {#1}}{{\mathrm d}t^2}}
\newcommand{\sdx}[1]{\frac{{\mathrm d}^2 {#1}}{{\mathrm d}x^2}}
\newcommand{\sdd}[2]{\frac{{\mathrm d}^2 {#1}}{{\mathrm d}{#2}^2}}
\newcommand{\ddn}[3]{\frac{{\mathrm d}^{#1} #2}{{\mathrm d} #3 ^{#1}}}

% partial derivatives
\newcommand{\pt}[1]{\frac{\partial {#1}}{\partial t}}
\newcommand{\px}[1]{\frac{\partial {#1}}{\partial x}}
\newcommand{\pp}[2]{\frac{\partial {#1}}{\partial {#2}}}
\newcommand{\spt}[1]{\frac{\partial^2 {#1}}{\partial t^2}}
\newcommand{\spx}[1]{\frac{\partial^2 {#1}}{\partial x^2}}
\newcommand{\spp}[2]{\frac{\partial^2 {#1}}{\partial {#2}^2}}
\newcommand{\ppn}[3]{\frac{\partial^{#1} #2}{\partial #3 ^{#1}}}
% integrals
\newcommand{\intl}[2]{\int_0^\infty\! #1 \mathrm{d}#2}
\newcommand{\intf}[2]{\int_{-\infty}^\infty\! #1 \mathrm{d}#2}


% quantum operators
\newcommand{\ssp}{\braket{\sigma^{+}(t)\sigma^{-}(t)}}
\newcommand{\aap}{\braket{a^{\dagger}(t)a(t)}}
\newcommand{\as}{\braket{a^{\dagger}(t)\sigma^{-}(t)}}
\newcommand{\sa}{\braket{a(t)\sigma^{+}(t)}}
\newcommand{\Hssp}{\braket{\sigma^{+}\sigma^{-}}}
\newcommand{\Haap}{\braket{a^{\dagger}a}}
\newcommand{\Has}{\braket{a^{\dagger}\sigma^{-}}}
\newcommand{\Hsa}{\braket{a\sigma^{+}}}
\newcommand{\adag}{a^{\dagger}}
\newcommand{\sigm}{\sigma^{-}}
\newcommand{\sigp}{\sigma^{+}}
\newcommand{\sigz}{\sigma^{z}}
\newcommand{\gp}{\gamma^{\prime}}
\newcommand{\oal}{\omega_a-\omega_0}
\newcommand{\ocl}{\omega_c-\omega_0}


% function related
\def\Gn{\mathbf{G}^n}
\def\Gm1{\mathbf{G}^{n-1}}
\def\GN{\mathbf{G}^N}
\def\G0{\mathbf{G}^0}
\def\G1{\mathbf{G}^1}
\def\Gi{\mathbf{G}^i}
\def\flamr{\mathbf{f}_\lambda(\mathbf{r})}
\def\rrn{\mathbf{r},\mathbf{r}_n}
\def\rnrn{\mathbf{r}_n,\mathbf{r}_n}
\def\rr{\mathbf{r},\mathbf{r}'}


% braket.sty          Macros for Dirac bra-ket <|> notation and sets {|}
%
\def\bra#1{\mathinner{\langle{#1}|}}
\def\ket#1{\mathinner{|{#1}\rangle}}
\def\braket#1{\mathinner{\langle{#1}\rangle}}
\def\Bra#1{\left<#1\right|}
\def\Ket#1{\left|#1\right>}
{\catcode`\|=\active
  \gdef\Braket#1{\left<\mathcode`\|"8000\let|\BraVert {#1}\right>}}
\def\BraVert{\egroup\,\mid@vertical\,\bgroup}
{\catcode`\|=\active
  \gdef\set#1{\mathinner{\lbrace\,{\mathcode`\|"8000\let|\midvert #1}\,\rbrace}}
  \gdef\Set#1{\left\{\:{\mathcode`\|"8000\let|\SetVert #1}\:\right\}}}
\def\midvert{\egroup\mid\bgroup}
\def\SetVert{\egroup\;\mid@vertical\;\bgroup}
% Some stuff deleted
% Macros for Dirac bra-ket <|> notation
\def\bra#1{\mathinner{\langle{#1}|}}
\def\ket#1{\mathinner{|{#1}\rangle}}
\def\Braket#1#2{\mathinner{\langle{#1}\! \mid\! {#2} \rangle}}
\def\ketbra#1{\ket{#1}\!\!\bra{#1}}
\newcommand{\Ketbra}[2]{\ket{#1}\!\!\bra{#2}}
\def\sandwich#1#2{\bra{#1}\! #2 \! \ket{#1}}
\def\Sandwich#1#2#3{\bra{#1}\! #2\! \ket{#3}}
%
% END  braket.sty     Macros for Dirac bra-ket <|> notation and sets {|}



% For faster processing, load Matlab syntax for listings
\lstloadlanguages{Matlab}%
\lstset{language=Matlab,
        frame=single,
        basicstyle=\small\ttfamily,
        keywordstyle=[1]\color{Blue}\bf,
        keywordstyle=[2]\color{Purple},
        keywordstyle=[3]\color{Blue}\underbar,
        identifierstyle=,
        commentstyle=\usefont{T1}{pcr}{m}{sl}\color{MyDarkGreen}\small,
        stringstyle=\color{Purple},
        showstringspaces=false,
        tabsize=5,
        % Put standard MATLAB functions not included in the default
        % language here
        morekeywords={xlim,ylim,var,alpha,factorial,poissrnd,normpdf,normcdf},
        % Put MATLAB function parameters here
        morekeywords=[2]{on, off, interp},
        % Put user defined functions here
        morekeywords=[3]{FindESS},
        morecomment=[l][\color{Blue}]{...},
        numbers=left,
        firstnumber=1,
        numberstyle=\tiny\color{Blue},
        stepnumber=5
        }
        
        
        
% Includes a figure
% The first parameter is the label, which is also the name of the figure
%   with or without the extension (e.g., .eps, .fig, .png, .gif, etc.)
%   IF NO EXTENSION IS GIVEN, LaTeX will look for the most appropriate one.
%   This means that if a DVI (or PS) is being produced, it will look for
%   an eps. If a PDF is being produced, it will look for nearly anything
%   else (gif, jpg, png, et cetera). Because of this, when I generate figures
%   I typically generate an eps and a png to allow me the most flexibility
%   when rendering my document.
% The second parameter is the width of the figure normalized to column width
%   (e.g. 0.5 for half a column, 0.75 for 75% of the column)
% The third parameter is the caption.
\newcommand{\scalefig}[3]{
  \begin{figure}[ht!]
    % Requires \usepackage{graphicx}
    \centering
    \includegraphics[width=#2\columnwidth]{#1}
    %%% I think \captionwidth (see above) can go away as long as
    %%% \centering is above
    %\captionwidth{#2\columnwidth}%
    \caption{#3}
    \label{#1}
  \end{figure}} % %
% % % % % % % % % % % % % % % % % % % % % % % % % % % % % %


% Setup the header and footer
\pagestyle{fancy}                                                       %
\lhead{\leftmark }                                                 %
\chead{}
\rhead{\firstxmark}                                                      %
\lfoot{\lastxmark}                                                      %
\cfoot{}                                                                %
\rfoot{Page\ \thepage\ of\ \protect\pageref{LastPage}}                  %
\renewcommand\headrulewidth{0.4pt}                                      %
\renewcommand\footrulewidth{0.4pt}                                      %

% This is used to trace down (pin point) problems
% in latexing a document:
%\tracingall

%%%%%%%%%%%%%%%%%%%%%%%%%%%%%%%%%%%%%%%%%%%%%%%%%%%%%%%%%%%%%
% Some tools
% Part of the preample is from a template for exercises/Q&A sheet shared at http://links.tedpavlic.com/tex/homework_new.tex
% If there is any problem on using the content, please contact the original author.
\def\changemargin#1#2{\list{}{\rightmargin#2\leftmargin#1}\item[]}
\let\endchangemargin=\endlist

\newcommand{\enterProblemHeader}[1]{\nobreak\extramarks{#1}{#1 continued on next page\ldots}\nobreak%
                                    \nobreak\extramarks{#1 (continued)}{#1 continued on next page\ldots}\nobreak}%
\newcommand{\exitProblemHeader}[1]{\nobreak\extramarks{#1 (continued)}{#1 continued on next page\ldots}\nobreak%
                                   \nobreak\extramarks{#1}{}\nobreak}%

\newlength{\labelLength}
\newcommand{\labelAnswer}[2]
  {\settowidth{\labelLength}{#1}%
   \addtolength{\labelLength}{0.25in}%
   \changetext{}{-\labelLength}{}{}{}%
   \noindent\fbox{\begin{minipage}[c]{\columnwidth}#2\end{minipage}}%
   \marginpar{\fbox{#1}}%

   % We put the blank space above in order to make sure this
   % \marginpar gets correctly placed.
   \changetext{}{+\labelLength}{}{}{}}%

\setcounter{secnumdepth}{3}
\newcommand{\homeworkProblemName}{}%
\newcounter{homeworkProblemCounter}%
\newenvironment{homeworkProblem}[1][Problem \arabic{homeworkProblemCounter}]%
  {\stepcounter{homeworkProblemCounter}%
   \renewcommand{\homeworkProblemName}{#1}%
   \section{\homeworkProblemName}%
   \enterProblemHeader{\homeworkProblemName}}%
  {\exitProblemHeader{\homeworkProblemName}}%

\newcommand{\problemAnswer}[1]
  {\noindent\mbox{\begin{minipage}[c]{\columnwidth}#1\end{minipage}}}
  % \fbox instead of \mbox can be used to get a frame.

\newcommand{\problemLAnswer}[1]
  {\labelAnswer{\homeworkProblemName}{#1}}

\newcommand{\homeworkSectionName}{}%
\newlength{\homeworkSectionLabelLength}{}%
\newenvironment{homeworkSection}[1]%
  {% We put this space here to make sure we're not connected to the above.
   % Otherwise the changetext can do funny things to the other margin

   \begin{changemargin}{0.3in}{.05in}
   \renewcommand{\homeworkSectionName}{#1}%
   \settowidth{\homeworkSectionLabelLength}{\homeworkSectionName}%
   %\addtolength{\homeworkSectionLabelLength}{-0.35in}%
   \changetext{}{-\homeworkSectionLabelLength}{}{}{}%
   \subsection{\homeworkSectionName}%
   \enterProblemHeader{\homeworkProblemName\ [\homeworkSectionName]}}%
  {\end{changemargin}\exitProblemHeader{\homeworkProblemName }%

   % We put the blank space above in order to make sure this margin
   % change doesn't happen too soon (otherwise \sectionAnswer's can
   % get ugly about their \marginpar placement.
   \changetext{}{+\homeworkSectionLabelLength}{}{}{}}%

\newcommand{\sectionAnswer}[1]
  {% We put this space here to make sure we're disconnected from the previous
   % passage

   \noindent\fbox{\begin{minipage}[c]{\columnwidth}#1\end{minipage}}%
   \enterProblemHeader{\homeworkProblemName}\exitProblemHeader{\homeworkProblemName}%
   \marginpar{\fbox{\homeworkSectionName}}%

   % We put the blank space above in order to make sure this
   % \marginpar gets correctly placed.
   }%



% Includes a MATLAB script.
% The first parameter is the label, which also is the name of the script
%   without the .m.
% The second parameter is the optional caption.
\newcommand{\matlabscript}[2]
  {\begin{itemize}\item[]\lstinputlisting[caption=#2,label=#1]{#1.m}\end{itemize}}

%%%%%%%%%%%%%%%%%%%%%%%%%%%%%%%%%%%%%%%%%%%%%%%%%%%%%%%%%%%%%


\begin{document}

\title{Notes on Classical Mechanics}
\author{Xiaodong Qi (i2000s@hotmail.com)}
\date{\today}



% \pagebreak
% \tableofcontents
% \pagebreak

\maketitle
\begin{abstract}
These are my notes on Classical Mechanics (advanced level), initially taken for the PHYS503 class lectured by Dr. Kenkre at the University of New Mexico, Fall 2012. \\ 

``Main$\_$CM.tex'' is the TeX code outlined the whole set of notes. The part on Lagrangian and Hamiltonian mechanics (Chapter~\ref{chap:lagham}) was modified from a \LaTeX  source code provided by Ninnat ``Tom'' Dangniam (thank Tom for letting me sharing his works online). \\


These notes are now openly shared for academic purpose, and can be redistributed and revised under the MIT license. Because most of the notes were taken on-fly, there are definitely numerous errors. 
I haven't found a chance to check the grammar errors carefully enough to hide to the readers that English is not my native language. 
If you find any errors, please help update with a better version in a Pull Request on the Git repository. Thanks for the contributions to these notes and derived works, if any. 
Enjoy learning and creating! \\

Xiaodong Qi, 

University of New Mexico 

Jan 25, 2016
\end{abstract}
\pagebreak
 \tableofcontents
 
\chapter{Introduction}

\begin{itemize}\itemsep2pt
\item[Course]: Classical Mechanics.
\item[Textbook] Alexander Fetter. Theoretical mechanics of particles and continua.
\item[Unusual Topics]
  \begin{itemize}
    \item Singular perturbation theory.
    \item Nonlinear damping $\rightarrow$ elliptic functions.
    \item Bifurcations.
    \item Synchronization.
    \item Memory function formalism.
  \end{itemize}
\end{itemize}

The role of mechanics in physics: foundation.

\section{The philosophy of mechanics}
A discussion on mass in the formula of $F=ma$. Is this formula a definition or natural law of mass?

Mechanics reduce the information of describing natural phenomena through physical laws.

\section{Physics equations in mechanics}\label{sec:equationofphysics}
\begin{itemize}
\item $\dt{x}=\cdots$
\item $\frac{d^2 x}{dt^2}=\cdots \rightarrow$ A force depends on the acceleration.
\item $\frac{d^3 x}{dt^3}=\cdots$
\item $\cdots$
\end{itemize}
In mechanics, we look for the equation up to the second order of $t$, that is $\frac{d^2 x}{dt^2}=\cdots$.

\begin{equation}\label{d2xdt2}
m \frac{d^2 x}{dt^2}=\left\{
                    \begin{array}{cl}
                      0 & \rightarrow \text{e.g. the first law of motion}.\\
                      const & \rightarrow \text{e.g. the second law of motion}.\\
                      f(x) & \rightarrow \text{e.g. the spring force}.\\
                      g\left(\dt{x}\right) & \rightarrow \text{e.g. damping}.\\
                      \!\! \text{All of the above}\!\! & \rightarrow \!
                            \text{e.g. damped simple harmonic oscillation:}\\
                      \quad & \qquad \quad m\frac{d^2 x}{dt^2}=-T\dt{x}-kx+A \cos\omega t.
                    \end{array}
                    \right.
\end{equation}

\chapter{Damped simple harmonic oscillations}\label{chap:dampledosc}
\section{Damped SHO equation and general damping equations}
Solve the equation of damped a simple harmonic oscillation given by
\begin{equation}
m\frac{d^2 x}{dt^2}=-T\dt{x}-kx+A \cos\omega t.
\end{equation}

\section{Laplace transformation method}\label {Sec:Laplace}
%\newline 
$\rightarrow\quad$ $f(t) \rightarrow \tilde{f}(s)=\int_0^\infty dt f(t) e^{-st}$.

General shape/form of equations in mechanics:
\begin{itemize}
\item with high power terms such as $\left( \dt{x}\right)^7$...
\item for continuous system...
\item \begin{align*}
        &\text{2nd order differential equations} \\
        &\leftarrow
        \left\{
         \begin{array}{cl} &\text{Newton's formulation:} m \frac{d^2 x}{dt^2}=\cdots \\
            &\text{what if the force cannot be identified?} \rightarrow \text{Lagrangian/Hamiltatian formalism.}
         \end{array}
        \right.
        \end{align*}
\item  $6$-dimension $(\vec{x},\vec{v},t)$.
\end{itemize}

\section{Behaviors of $m \frac{d^2 x}{dt^2}=g\left(\dt{x}\right)$ and variable replacements}\label{sec:vreplace}

$\Leftrightarrow m\dt{v}=g(v) \Rightarrow t+const=\int\frac{dv}{g(v)}$.

e.g. $m\dt{v}=-\Gamma v \rightarrow v(t)=v_0e^{-\frac{\Gamma}{m}t}.$ The solution is that the object tends to stop yet never stops.

Q: what if $v^2,\, v^7\cdots$

In general, let us analyze
$\dt{v}=-av^r$
\begin{align}
\Rightarrow & \int{dv\cdot v^{-r}}=-at + const\\
\Rightarrow & \frac{v^{1-r}}{1-r}=-at + const\\
\Rightarrow & \frac{v^{1-r}(t)}{1-r}=-at + \frac{v^{1-r}_0}{1-r}\\
\Rightarrow & v(t)=\left\{ (1-r)\left[ -at + \frac{v^{1-r}_0}{1-r}\right] \right\}^{\frac{1}{1-r}}.
\end{align}
I have used $v_0=v(t=0)$ as the initial condition.

% several cases
%\begin{homeworkProblem}
\begin{subsubsection}
  {\textbf{Case 1: $r=1$.}} The function of $v(t)$ is given by
   \begin{equation}
     v(t)=v_0 e^{-at}.
   \end{equation}
   The function is plotted in Fig.\ref{../Figs/Plot1_1} in the solid line.
   \scalefig{../Figs/Plot1_1}{0.618}{Plots of $v=v(t)$ with $a=0.5$, $v0=1$.}

   Let us look at some extreme cases.

   \begin{enumerate}
    \item $t=0, \, v(t=0)=v_0$.
    \item $t\rightarrow \infty$, $v(t\rightarrow \infty) \rightarrow 0$. Notice that $v(t\rightarrow \infty)$ tends to $0$ yet cannot reach $0$.
   \end{enumerate}

 \end{subsubsection}

 \begin{subsubsection}
  {\textbf{Case 2: $r=3$.}} The function of $v(t)$ is given by
   \begin{equation}
     v(t)=\left(2at+\frac{1}{v_0^2}\right)^{-1/2}.
   \end{equation}
   The function is plotted in Fig.\ref{../Figs/Plot1_1} in the dashed line.

   Let us look at some extreme cases.

   \begin{list}{\labelitemi}{\leftmargin=1cm}
    \item[1.] $t=0$, $v(t=0)=v_0$.
    \item[2.] $t\rightarrow \infty$, $v(t\rightarrow \infty) \rightarrow 0$. Notice that $v(t\rightarrow \infty)$ tends to $0$ yet cannot reach $0$.
    \item[3.] If $a<0$, at some $t$, $v$ can be a complex number. This is one of the differences from case 1.
   \end{list}


 \end{subsubsection}

 \begin{subsubsection}
  {\textbf{Case 3: $r=\frac{1}{2}$.}} The function of $v(t)$ is given by
   \begin{equation}
     v(t)=\left( -\frac{a}{2}t+v_0^{1/2}\right)^2.
   \end{equation}
   The function is plotted in Fig.\ref{../Figs/Plot1_1} in the dotted line.

      Let us look at some extreme cases.

   \begin{list}{\labelitemi}{\leftmargin=1cm}
   \item[1.] $t=0$, $v(t=0)=v_0$.
   \item[2.] $t\rightarrow \infty$, $v(t\rightarrow \infty) \rightarrow \infty$. Notice that $v(t\rightarrow \infty)$ tends to $+\infty$ rather than $0$. This is one main difference from cases 1 \& 2.
   \item[3.] When $t=\frac{2}{a}v_0^{\frac{1}{2}}$, $v(t)=0$, which means the object stops at $t=\frac{2}{a}v_0^{\frac{1}{2}}$ point. This is another main difference from cases 1 \& 2.
   \end{list}
  \end{subsubsection}

  \begin{subsubsection}
    {\textbf{Case 4: $r=-\frac{1}{2}$.}} The function of $v(t)$ is given by
   \begin{equation}
     v(t)=\left( -\frac{3a}{2}t+v_0^{3/2}\right)^{\frac{2}{3}}.
   \end{equation}
   The function is plotted in Fig.\ref{../Figs/Plot1_1} in the dash-dot line.

      Let us look at some extreme cases.

   \begin{list}{\labelitemi}{\leftmargin=1cm}
   \item[1.] $t=0$, $v(t=0)=v_0$.
   \item[2.] $t\rightarrow \infty$, $v(t\rightarrow \infty) \rightarrow -\infty$. Notice that $v(t\rightarrow \infty)$ becomes negative. This is the main difference from the cases above.
   \item[3.] When $t=\frac{2}{3a}v_0^{\frac{3}{2}}$, $v(t)=0$.
   \end{list}
 \end{subsubsection}

 \begin{subsubsection}
    {\textbf{Case 5: $r=0$.}} The function of $v(t)$ is given by
   \begin{equation}
     v(t)= v_0 -at .
   \end{equation}
   Obviously, the object maintains a uniform linear motion at $v(t)=v_0$. The $v=v(t)$ curve should be a straightly horizontal line if plotted in Fig.~\ref{../Figs/Plot1_1} (not shown in the figure).
 \end{subsubsection}

 In sum, $r=1$ is the critical value of $r$ to judge whether the object can suddenly stop at some time point: if $r\geq 1$, the object does not stop; if $r<1$, the object may suddenly stop. Moreover, $r=0$ is the critical value to judge whether the object can have a negative velocity: if $r>0$, the object always has positive velocity; if $r=0$, the object maintains a uniform linear motion; if $r<0$, the object can move towards the negative direction to the initial velocity.
%\end{homeworkProblem}

\subsubsection{Restrict to linear case}
In the case of linear damping,
$\dt{v}=-\alpha(t)v,$ where the damping factor $\alpha=\frac{\Gamma}{m}$. The forms of $\alpha(t)$ may be as of plotted in Fig.~\ref{../Figs/Plot1_2}, that is
\begin{equation}
\alpha(t)= \left\{ \begin{array}{cc}
                        e^{-at},& \text{case 1;}\\
                        1-e^{-at},& \text{case 2.}
                   \end{array}\right.
\end{equation}
\scalefig{../Figs/Plot1_2}{0.618}{Plots of $\alpha=\alpha(t)$ with $a=0.5$, $v0=1$. }

Now the equation can be rewritten as
\begin{align}
\Rightarrow & \frac{dv}{v}=-\alpha(t)dt=d\tau\\
\Rightarrow & \frac{dv}{d\tau}=-v.
\end{align}

In case one, the solution of $v(t)$ gives
\begin{equation}
v(t)=v_0e^{\frac{1}{a}e^{-at}+c}.
\end{equation}
Using the initial condition that $v(t=0)=v_0$, one can obtain
\begin{equation}
v(t)=v_0e^{\frac{1}{a}e^{-at}-\frac{1}{a}}.
\end{equation}
The result is plotted as the solid line in Fig.~\ref{../Figs/Plot1_2_1}. When $t\rightarrow \infty$, $v(t)\rightarrow v_0e^{-\frac{1}{a}}$. Therefore, the value of $a$ determines the behavior of $v(t)$ when $t$ is large. If $a>0$, $v(t\rightarrow \infty)<v_0$; if $a<0$, $v(t\rightarrow \infty)>v_0$.
\scalefig{../Figs/Plot1_2_1}{0.618}{Plots of $v=v(t)$ with $a=0.5$, $v_0=1$.}


In case two, the solution of $v(t)$ gives
\begin{equation}
v(t)=v_0e^{-\frac{1}{a}e^{-at}-t+c}.
\end{equation}
Using the initial condition that $v(t=0)=v_0$, one can obtain
\begin{equation}
v(t)=v_0e^{-\frac{1}{a}e^{-at}-t+\frac{1}{a}}.
\end{equation}
The result is plotted in the dashed line in Fig.~\ref{../Figs/Plot1_2_1}. When $t\rightarrow \infty$, $v(t)\rightarrow 0$.

As shown in Fig.~\ref{../Figs/Plot1_2_1}, the main differences between case 1 and 2 are the shape of the velocity curve and the value of $v(t\rightarrow \infty)$. In case 2, the velocity approximate to $0$, but case 1 does not.

\section{Energy method for $ \sdt{x}=f(x) $}\label{sec:energymethod}
For equation $m\sdt{x}=A\cos\omega t -B\dt{x}-kx$, there exists a case that
\begin{equation}
\sdt{x}=f(x),
\end{equation}
where $f(x)$ is time-independent. We can rewrite the equation above to be
\begin{equation}
\dt{x}\cdot \sdt{x}=\dt{x}\cdot \underbrace{f(x)}_{no time-dependence},
\end{equation}
\begin{align}
\Longleftrightarrow & \dt{\left[ \frac{1}{2}\left( \dt{x}\right)^2\right]}= \dt{ \overbrace{\left[ \int{f(x)dx}\right]}^{\text{birth of potential energy}}}\\
\Rightarrow & \frac{1}{2}\left( \dt{x}\right)^2 = \int{f(x)dx} +c \\
\Leftrightarrow & \underbrace{\frac{m}{2}\left( \dt{x}\right)^2}_{\text{kinetic energy}} +\underbrace{[-m\int{f(x)dx}]}_{\text{potential energy}}=\underbrace{const\vphantom{\int}}_{\text{total energy}}.
\end{align}

2 practical procedures to solve the movement of an object governed by an ODE with time-independent zero-order terms:
\begin{itemize}
\item[1.] \begin{align}
            &\frac{1}{2}\left( \dt{x}\right)^2 + \underbrace{U(x)}_{potential energy term}=E \rightarrow \text{first order procedure}\\
            \Rightarrow & \dt{x}=\sqrt{2[E-U(x)]}\\
            \Rightarrow & \int{\frac{dx}{\sqrt{2[E-U(x)]}}}=t+const.
         \end{align}
\item[2.] draw picture of $U(x)$ $\rightarrow$ movement of object (escaping and oscillating zones).
\end{itemize}

e.g. spring force $f(x)=-\frac{kx}{m}$. $U(x)=\frac{k}{2m}x^2$.
\begin{align}
\int{\frac{dx}{\sqrt{2[E-U(x)]}}}=\int{\frac{dx}{\sqrt{2E-\frac{k}{m}x^2}}}=\sin^{-1}(*)=t+const.
\end{align}

\section{Elliptic Equation}\label{sec:ellipticequ}
Intro: following the equation discussed in the last section, let us make some revisions:
\begin{align}
m\sdt{x}=\xcancel{A\cos\omega t} -\xcancel{B\dt{x}}-k\cancelto{\text{make it nonlinear}}{x}
\end{align}
 e.g. physics pendulum.

\begin{align}
\text{simple pendulum: } & \sdt{x}=f(x) \rightarrow f(x)=-const\cdot x\\
\text{physics pendulum: } & \sdt{x}=f(x) \rightarrow f(x)=-const\cdot \sin x
\end{align}
%\begin{align}
\begin{empheq}[box={\fboxsep=10pt\colorbox{cyan}}]{align}
t &=\int_0^x\frac{dz}{\sqrt{1-z^2}}, x=\sin t, (z=sin\theta)\\
&= \int \frac{\cancel{cos\theta}d\theta}{\cancel{\cos \theta}} \Rightarrow x=\sin t.
\end{empheq}
%\end{align}


\textbf{Pretend Game 1}: %\todo{Skipped game 1.}:

\begin{align}
\frac{dt}{dx}=\frac{1}{\sqrt{1-x^2}} \Rightarrow \dt{x}=\sqrt{1-x^2},\, \frac{d}{dt} \sin t=\sqrt{1-\sin ^2 t}.
\end{align}
define $\sqrt{1-\sin ^2 t}$ as $\cos t \Leftrightarrow \sin^2t+\cos ^2 t=1,\, \dt{\sin t}=\cos t$.

\textbf{Pretend Game 2}:

examine $t=\int^x_0 \frac{dz}{1-z^2}=\frac{1}{2}\int^x_0 (\frac{1}{1-z}+ \frac{1}{1+z})dz=\frac{1}{2}\ln \frac{1+k}{1-k}.$

Let us set $z=\tanh\theta$, show that it works.

$x=\tanh t.$
\begin{itemize}
\item $\left[ \frac{1}{1-z^2}\right]^{1/2}\rightarrow \sin t$ %\todo{skipped the drawing in PDF.}
\item $\left[ \frac{1}{1-z^2}\right]^{1}\rightarrow \tanh t$ (stretched from $\sin t$.)%\todo{skipped the drawing in PDF.}
\item Similarly, $\cos t \rightarrow sech t.$
\end{itemize}

\[
\begin{array}{ccc}
\frac{1}{\sqrt{1-z^2}} & \text{intro intermediate}  \\
\updownarrow & \frac{1}{\sqrt{1-z^2}\sqrt{1-k^2z^2}} &  \fbox{$\Leftrightarrow                                                                            t=\int^x_0 \frac{dz}{\sqrt{1-z^2}\sqrt{1-k^2z^2}}$}  \\
\frac{1}{1-z^2} &  \underbrace{\underbrace{0}_{sin}\leq k \leq \underbrace{1}_{\tanh}}_{\text{in-between, elliptic func}} & x=sn(t,k)\rightarrow \textbf{elliptic function}.
\end{array}
\]
$k$ is called elliptic modulus $\Leftrightarrow m=k^2$ is called elliptic parameter.

\subsection{Properties of elliptic functions}
Differential equations of elliptic functions:~\footnote{Refs: Abramowitz and Stegun's book for special functions; Byrd's book for elliptic function.}

\begin{align*}
\begin{array}{cl}
\dt{\sn} = \cn \cdot \dn &\rightarrow\left\{ \begin{array}{c}
                                            \dt {\cn} =?\\
                                            \dt {\dn} =?
                                            \end{array}\right.\rightarrow \sdt {\sn} =? \\
\underbrace{\dt{x} =\sqrt{1-z^2}\sqrt{1-k^2z^2}}_{\text{from the integral}} & \left\{ \begin{array}{cc}
                                            \cn(t,k) =\sqrt{1-\sn^2(t,k)}, & \text{elliptic $\cos$}\\
                                            \dn(t,k) = \sqrt{1-k^2\sn^2(t,k)}.
                                            \end{array}\right.
\end{array}
\end{align*}


What if $k>1$?

\begin{align}
 \begin{array}{cll}
 t=\int^x_0 \frac{dz}{\sqrt{1-z^2}\sqrt{1-k^2z^2}} \rightarrow & \left\{
        \begin{array}{cc}
        0\leq k \leq 1,& x=\sn(t,k);\\
        k>1, & \left\{ \begin{array}{l}
                 y=kz \rightarrow z=\frac{1}{k} y, \\
                 dz=\frac{1}{k} dy \rightarrow
                    \int^{y=kx}_{y=0}\frac{1}{k}\frac{dy}{\sqrt{1-\frac{y^2}{k^2}}\sqrt{1-y^2}}\\
                 \chi=\frac{1}{k} \rightarrow \frac{t}{\chi}
                    =\int^{x/k}_{y=0}\frac{dy}{\sqrt{1-y^2}\sqrt{1-k^2y^2}}\\
                 \Rightarrow \frac{x}{\chi}= \sn(\frac{t}{\chi},\chi)
                \end{array} \right.
        \end{array}\right.
 \end{array}
\end{align}
$\Rightarrow x=\frac{1}{k}\sn(kt,\frac{1}{k})\quad (k>1)$.

For $\cn \, \& \, \dn$?

More properties of elliptic functions have been studied in the Problem Set \#2. A set of numerical calculation of three elliptic functions have been shown in Fig.~\ref{../Figs/PlotHW2_2}.
\scalefig{../Figs/PlotHW2_2}{0.9}{\textbf{Elliptic functions with $K=[0,\,0.5,\, 0.9,\, 0.99,\, 0.9999,\,1]$.} The red-solid lines show the $ \sn(t,k) $ function; the blue-dot-dashed lines show the $ \cn(t,k) $ function; the green-dotted lines show the $ \dn(t,k) $ function.}

\section{Physical pendulum and nonlinear effects}
Equation of motion:
\begin{align}
-mg \sin{\theta} &=ml\frac{d^2}{dT^2}\theta\\
\Rightarrow \frac{d^2\theta}{dT^2}+\frac{g}{l} \sin{\theta} &= 0.
\end{align}
Through redefining $ t=\frac{g}{l}T $ and $ x=\theta $ with the factor of $ \frac{g}{l} $, one can get
\begin{empheq}[box={\fboxsep=10pt\colorbox{cyan}}]{align}
\sdt{x}+\sin{x}=0,
\end{empheq}
where we have chosen $ x $ as a dimensionless quantity. 

To solve it, we can use energy method to give
\begin{equation}
\dt{\left[ \frac{1}{2}\left( \dt{x}\right )^2\right]} + \underbrace{\left( \dt{x}\right)\sin{x}}_{\dt{\int sinx dx}}=0
\rightarrow \dt{\overbrace{\left[ \frac{1}{2}\left( \dt{x}\right )^2 + (1-\cos{x})\right]}^{E}}=0.
\end{equation}
\begin{align}
\rightarrow & \left( \dt{x} \right)^2 =2\left[ E-(1-\cos{x})\right]\\
\rightarrow & t + const.=\int \frac{dx}{\sqrt{2[E-(1-\cos{x})]}}.
\end{align}
Now, we can discuss the movement of a physics pendulum. Suppose the initial condition gives $ v_0=\left( \dt{x}\right)|_{t=0},\, x(t=0)=0 $. Specifically, 
\begin{align}
\xrightarrow[x(t=0)=0]{\cos{x}=1} & E=\frac{1}{2}v_0^2\\
\Rightarrow & t=\int_{0}^{x}\frac{dy}{\sqrt{v_0^2-4\sin^2{y/2}}}=\int_{0}^{x}\frac{dy}{2\sqrt{(v_0/2)^2-\sin^2{y/2}}}\\
\Rightarrow & v_0 t =\int_{0}^{x}\frac{dy}{\sqrt{1-\left(\frac{v_0}{2}\right)^2\sin^2{y/2}}}.
\end{align}
Now, let us connect the equation above to the elliptic functions. 
We let $z=\sin(\frac{y}{2})$, therefore, $ dz=\frac{1}{2}\cos{y/2}dy=\frac{1}{2}\sqrt{1-z^2}dy $, $ dy=\frac{dz}{\frac{1}{2}\sqrt{1-z^2}} $.

Now that 
\begin{equation}
v_0 t = \int_{y=0}^{y=x}\frac{dz}{\frac{1}{2}\sqrt{1-z^2}\sqrt{1-\left(\frac{2}{v_0}\right)^2z^2}},
\end{equation}
or
\begin{equation}
\frac{v_0 t}{2} = \int_{z=0}^{z=\sin\frac{x}{2}}\frac{dz}{\sqrt{1-z^2}\sqrt{1-\left(\frac{2}{v_0}\right)^2z^2}}.
\end{equation}
\begin{align}
\Rightarrow \sin{\frac{x}{2}} &= \sn(\frac{v_0 t}{2},\frac{2}{v_0})\\
\Rightarrow x &= 2 \sin^{-1}\left[ \sn(\frac{v_0 t}{2},\frac{2}{v_0})\right].
\end{align}

If $ \frac{v_0}{2} \ll 1 $, $ \frac{x}{2} \approx \sin{\frac{x}{2}}=\frac{v_0}{2} \sn{t,\frac{v_0}{2}} \approx \frac{v_0}{2} \sin t$, or $ x\approx v_0 \sin t $. 

To get the velocity, we can use the relationship that
\begin{equation}
\frac{1}{2} \underbrace{\cos{\frac{x}{2}}}_{\sqrt{1-\sin^2\frac{x}{2}}}\cdot \underbrace{\dt{x}}_{v(t)}= \frac{v_0}{2} \underbrace{\cn(\frac{v_0t}{2},\frac{2}{v_0})}_{\sqrt{1-\sn^2(\frac{v_0t}{2},\frac{2}{v_0})}}\cdot \dn(\frac{v_0 t}{2},\frac{2}{v_0}).
\end{equation}
\begin{equation}
\Rightarrow v=\dt{x}=\frac{v_0}{2} \cn(\frac{v_0 t}{2},\frac{2}{v_0})=v_0 \cn(t,\frac{v_0}{2}).
\end{equation}


Some other useful properties of elliptic functions are shown below. 

Analogy to 
\begin{align}
\mathrm{sin}\left(u+\nu \right) =\frac{\mathrm{sin}u\mathrm{cos}\nu +\mathrm{cos}u\mathrm{sin}\nu }{1},
\end{align} we have the Additional theorem for the elliptic functions as below~\footnote{From Milton Abramowitz's book on \textit{Handbook of Mathematical Functions with Graphs, Formulas, and Mathematical tables}.}.
\begin{align}
\mathrm{\sn}\left(u+\nu ,k\right) &=\frac{\sn\left(u,k\right) {\cn} \left(\nu ,k\right)\dn\left(\nu ,k\right)+\sn\left(\nu ,k\right)\cn\left(u,k\right)\dn\left(u,k\right)}{1-{k}^{2}\sn^{2}\left(u,k\right)\sn^{2}{\left(\nu ,k\right)}},\\
\mathrm{\cn}\left(u+\nu ,k\right) &=\frac{\cn\left(u,k\right) {\cn} \left(\nu ,k\right)-\sn\left(u ,k\right)\dn\left(u,k\right)\sn\left(\nu,k\right)\dn\left(\nu,k\right)}{1-{k}^{2}\sn^{2}\left(u,k\right)\sn^{2}{\left(\nu ,k\right)}},\\
\mathrm{\dn}\left(u+\nu ,k\right) &=\frac{\dn\left(u,k\right) {\dn} \left(\nu ,k\right)- k^2 \sn\left(u ,k\right)\cn\left(u,k\right)sn\left(\nu,k\right)\cn\left(\nu,k\right)}{1-{k}^{2}\sn^{2}\left(u,k\right)\sn^{2}{\left(\nu ,k\right)}}.
\end{align}

\subsection{Physics pendulums and the potential of velocity}\label{sec:pendulum}
Back to the physics pendulum problem. We assume
\begin{align}
\sdt{x}+\sin x &=0, \label{sinxd}\\
x(0)=0, \, & v(0)=\left( \dt{x}\right) = v_0.
\end{align}
The solution can be
\begin{align}
\sin\frac{x}{2} &= \sn(\frac{v_0 t}{2},\frac{2}{v_0})=\frac{v_0}{2} \sn(t,\frac{v_0}{2}),\\
v &=v_0 \dn(\frac{v_0 t}{2} , \frac{2}{v_0})= v_0 \cn(t,\frac{v_0}{2}).
\end{align}

According to the condition of $ k\leq 1 $, we have $v_0=2$ is the
transition point (see Fig.~\ref{../Figs/handdraw01.jpg}).
\scalefig{../Figs/handdraw01.jpg}{0.9}{Evolution of elliptic functions and their integration.}

Evolution of $\cn$ function (see Fig.~\ref{../Figs/handdraw02.jpg}).
\scalefig{../Figs/handdraw02.jpg}{0.9}{Evolution of cn(t,k) function and similar functions.}

\begin{align}
\dot{x}\ddot{x} + (\sin x) \dot{x} &=0,\\
\frac{1}{2} \left( \dot{x} \right)^2 +[-\cos x] &= c,\\
\left( \dot{x} \right)^2 &= 2[c +\cos x].
\end{align}

Define differential from Equ.~\ref{sinxd}:

\begin{align}
\dddot{x} + (\cos x)\dot{x} =0 . \label{eq:thirdorderxcos}
\end{align}
Integrate to give
\begin{align}
\frac{\dot{x}^2}{2}-c - \cos x &=0,\\
\textrm{or}\quad \quad \cos x= \frac{\dot{x}^2}{2}-c.
\end{align}
Substituting the equation above into Equ.~\ref{eq:thirdorderxcos}, we obtain
\begin{align}
\dddot{x}+[\frac{\dot{x}^2}{2}-c] \dot{x} &=0,\\
\ddot{v}+ [\frac{v^2}{2}-c] &=0,\\
%\fbox
\ddot{v}+A v^3 - Bv=0,
\end{align}
where $A=\frac{1}{2}$, $B=c$.
We have (inadvertently) converted a $2^{nd}$ order differential equation in $x$ with  a $2^{nd}$ order differential equation in $v$. Now we obtain the second-order equation of velocity:
\begin{align}
\ddot{v}= g(v)= -A v^3+Bv.
\end{align}
What is the ``velocity potential''? Answer: 
\begin{align}
A\frac{v^4}{4} -B\frac{v^2}{2}.
\end{align}
This is a double well. $ v $ can become trapped in one of the wells, in which cases $x$ becomes
unbounded. See Fig.~\ref{../Figs/handdraw03}
\scalefig{../Figs/handdraw03}{0.9}{Potential of $ x $ and $ v $.}

What learned:
\begin{itemize}
\item There is no law to confine physics. All kinds of equations are valid.
\item Do we need more initial condition? -- We need to know the relationship between $x$ and $v$.
\end{itemize}

\subsection{Coupled multiple level quantum system and physics pendulums}\label{sec:quantumlevels}
\textbf{Sep 11}.

Something about Quantum mechanics:
\begin{equation}
\bra{m}i\hbar \frac{\partial}{\partial t} \ket{\Psi(t)}=\bra{m}H\ket{\Psi(t)}.\label{Schrod}
\end{equation}
\begin{equation}
H=H_0 + V.
\end{equation}
\begin{equation}
H_0 \ket{m} = E_m \ket{m}.
\end{equation}
\begin{align}
C_m(t)&=\bra{m}\Psi(t)\rangle,\\
V_{mn} &= \bra{m}V\ket{n}.
\end{align}

Equ.~\ref{Schrod} is equivalent to 
\begin{equation}
i \dt{C_m}= E_m C_m + \sum_n {V_{mn}C_n}. 
\end{equation}
Equ.~\ref{Schrod}
\begin{equation}
\rightarrow i\frac{\partial p}{\partial t}=\underbrace{[H,p]}_{L_p},
\end{equation}
where $ p=\ket{\Psi (t)}\bra{m} $. 

We want to use $ \bra{m} \rho \ket{n} = \rho_{mn}=C_m^* C_n$, so that the equation above can be written as 
\begin{equation}
i\dt{\rho_{mn}}=(E_m-E_n)\rho_{mn}+ \sum_s{\left( V_{ms}\rho_{sn}-\rho_{ms} V_{sn} \right)}.
\end{equation}
For any operator, the expectation value can be written as
\begin{equation}
\bra{\Psi}{\hat{O}}\ket{\Psi}=Tr(\hat{\rho}\hat{O}).
\end{equation}

Once all the electrons in a crystal, for example, are in the same level,
\begin{equation}
E_m =E^0, \quad \text{for all m.}
\end{equation}

%\missingfigure{911-1}.
\scalefig{../Figs/handdraw911_1}{0.8}{Transition among states.}

If the energy is proportional to the probability of the electron staying in one level,
\begin{equation}
E_m=E^0 - \chi |C_m | ^2 = E^0 -\chi \rho_{mm}. 
\end{equation}
\begin{align}
i\dt{C_{m}} &=-\chi |C_m|^2 C_{m}+ \sum_n{\left( V_{mn}C_{n} \right)},\\
i\dt{\rho_{mn}} &=-\chi  (\rho_{mm}-\rho_{nn})\rho_{mn} + \sum_s{\left( V_{ms}\rho_{sn}-\rho_{ms} V_{sn} \right)}.
\end{align}
These two equations are called discrete nonlinear Schrodinger equation. These equations have not been fully solved!
When $ \chi=0 $, the equations become linear.

To solve these nonlinear equations, one can find a way to simplify it. One way is to look at the two-coefficient case and simplify it as
\begin{align}
i\dt{C_{1}} &=-\chi |C_1|^2 C_{1}+  V {C_2} \quad \&\, C_2 \,{\text eq}\\
i\dt{\rho_{11}} &=V(\rho_{21}-\rho_{12}),\\
i\dt{\rho_{12}} &=V(\rho_{22}-\rho_{11})-\chi (\rho _ {11}- \rho _{22})\rho_{12}.
\end{align} 
Now that $ \chi=0 $,
\begin{align}
i\dt{C_1} &=V {C_2}\quad  \& \quad i\dt{C_2}=VC_1,\\
\text{multiply } & i\dt, \\
\Rightarrow & \underbrace{-\sdt{C_1} = V i\dt{C_2}=V V C_1.}_{ \sdt{C_1}+V^2C_1=0}
\end{align}
$C_1$ can be solved as an oscillator. 
%\missingfigure{911-2.}
\scalefig{../Figs/handdraw911_2}{0.8}{$ C_1 $ can describe the movement of an oscillator.}

Introduce 2 real quantities
\begin{align}
p &=\rho_{11}-\rho_{22},\\
q &= i (\rho _{12}-\rho_{21}),\\
r &=  \rho _{12}+\rho_{21}.
\end{align}
Now, we can get the equation in the form of
\begin{equation}
\frac{d}{dt}\left({\begin{array}{c}
p \\ q \\ r
\end{array}}\right)= \left(\begin{array}{c}
-q \\ -p_+i \chi r q\\ -\sin\chi rq 
\end{array}
\right)\quad (\text{for example...})
\end{equation}
In quantum mechanics, we usually have $ p^2 + q^2 + r^2=1 $ as well.

Homework: solve the $p,q,r$ equations. Then eliminate $ q \& r $ to get an equation solely depends on $ p $. That can be 
\begin{equation}
\sdt{p}=Ap-Bp^3 ,
\end{equation} 
which is the equation of velocity for a physical pendulum. The probability difference will behave as velocity of pendulum does. 

\begin{equation}
\frac{\chi}{4v}\rightarrow k.
\end{equation} 

%\missingfigure{913-1}.
\scalefig{../Figs/handdraw913_1}{0.8}{$ \cn $ function and its limit.}

Velocity as a function of $ t $. Two cases: when the nonlinearity is small, or is large. The details will be studied in the homework (HW2).
%\missingfigure{913-2}.
\scalefig{../Figs/handdraw913_2}{0.8}{One possible diagram of probability distribution based on $ \cn $ and $ \dn $ function.}

%Q: What is 
%\begin{equation}
%\boxed{\cos^3 t}
%\end{equation}
%in terms of $ \cos t \, \& \, \cos 3t$ (as a sum)?



\chapter{Singular perturbation theory}\label{chap:singularperturbation}
\section{Solving a high-order differential equation using perturbation method}
\textbf{Sep 13}.

Equations with nonlinear terms, for example, 
\begin{equation}
\sdt{x}+ \alpha \dt{x} + \omega^2 x = 0
\end{equation}
can be modified as
\begin{equation}
\sdt{x}+ \alpha \left(\dt{x}\right)^{17} + \omega^2 x = 0.
\end{equation}
Can we solve it analytically? This is the technique to learn in this section. One technique to be introduced is called singular perturbation method.

Suppose we have an equation of 
\begin{equation}
\frac{d^2y}{dT^2}+ A \left(\frac{dy}{dT}\right)^{2n+1} + \omega^2 y = 0.
\end{equation}
PS: one can introduce a quantity of $ x=y/y_0 $ and/or a quantity of $ t=T/T_0 $ to make a differential equation to be dimensionless so that one do not need to worry about the dimensions. One dimensionless equation can be 
\begin{equation}
\frac{d^2x}{dt^2}+ \alpha \left(\frac{dx}{dt}\right)^{2n+1} + \omega^2 x = 0,\quad n=0,\,1,\,\cdots 
\end{equation}

Suppose the initial condition gives
\begin{equation}
x(0)=0,\quad \dot{x}(0)=B.
\end{equation}
Now, if $ \alpha=0 $, $ x(t)=B\sin t $. If $ \alpha\neq 0 $, we can set $ x=x_0 +\alpha x_1+ \alpha^2 x_2 + \cdots $ using the normal perturbation method. 

\textit{e.g.} $n=1$, so that 
\begin{equation}
\sdt{x} + \alpha \left( \dt{x}\right)^3 + x =0.\label {alphasdt}
\end{equation}
Using perturbation method, we have
\begin{equation}
\frac{d^2}{dt^2}(x_0 + \alpha x_1 + \cdots)+ \alpha \left[ \dt{x_0} + \alpha \dt{x_1} + \cdots \right]^3 +(x_0 + \alpha x_1 + \cdots)=0.
\end{equation}
\begin{align}
\sdt{x_0} + x_0 &=0, \label{pert1}\\
\sdt{x_1} + \left( \dt{x_0}\right)^3 + x_1 &= 0 \label {pert2},
\end{align}
which are the outcome of the perturbation theory we usually use. 

Equ.~\eqref{pert1} gives
\begin{equation}
x_0(t) = B\sin t.
\end{equation}
Equ.~\eqref{pert2} gives
\begin{equation}
\left(\dt{x_0}\right)^3 = B^3 cos^3 t= \frac{B^3}{4}(cos 3t + 3 cos t),
\end{equation}
using  $ \cos^3 t= \frac{cos 3t + 3 cos t}{4}$. 

Now, Equ.~\eqref{pert2} gives
\begin{equation}
\sdt{x_1}+ x_1 = -\frac{B^3}{4} \cdot \underbrace{\left[\cos 3t + 3 \cos t\right] }_{\text{one fast oscillating term $ + $ an important term}}.
\end{equation}
This equation gives an oscillation that it oscillates normally at the beginning, but will blow off consequently. The terms in the brace are called secular terms. 



It can be shown that the normal perturbation method fails for any $ n $. 

\section{Singular perturbation method}
\textbf{Sep 18}.

A better method: singular perturbation method. This method is based on treating the equations above as partial differential equations with two time scales, which include a fast time scale $ \tau $ and a slow time scale $ t $. With these two time scales, one can convert an ODE $ (t) $ to a PDE $ (t,\tau) $. 

An example,
\begin{equation}
\sdt{x}+ \beta \dt{x}+\omega^2 x=0
\end{equation}
gives
\begin{align}
\epsilon^2 \tilde{x} -\epsilon x_0 - \dot{x_0}+\beta (\epsilon \tilde{x}-x_0)+ \omega^2 \tilde{x}&=0,\\
(\epsilon^2+ \beta \epsilon + \omega^2)\tilde{x}&=\dot{x_0}+ (\epsilon + \beta)x_0
\end{align}
\begin{align}
\Rightarrow \tilde {x}= \frac{\dot{x_0}}{\epsilon^2+\beta \epsilon + \omega^2}
\end{align}

\begin{align}
\tilde{x}= \frac{B}{(\epsilon +\beta/2)^2 + (\omega^2-\beta^2/4)}=\frac{B}{\Omega}\left( \frac{\Omega}{(\epsilon+ \beta/2)^2+\Omega^2}\right),
\end{align}
where $ \Omega = \sqrt{\omega^2 - \frac{\beta^2}{4}} $.  $ \Omega \approx \omega $ in most cases, and 
\begin{align}
x(t)= \frac{B}{\Omega}e^{-\frac{\beta t}{2}}\sin \Omega t.\label {xtdamping} 
\end{align}
This solution has the main feature that $ x(t)=B \frac{\sin t}{\sqrt{\cdots t}} $. The result can be plotted in the figure below. 
\scalefig{../Figs/handdraw918_1}{0.5}{Diagram a damping oscillation described in Equ.~\eqref{xtdamping}.}
%\missingfigure{918-1.}

Let make $ \tau =\delta t = \epsilon t $, where $ \epsilon  $ is small. We go back to solve Equ.~\eqref{alphasdt}. Now, 
\begin{align}
\dt{x } = \frac{\partial x}{\partial t}+ \frac{\partial x}{\partial \tau}\frac{d\tau}{dt}= \frac{\partial x}{\partial t}+ \epsilon \frac{\partial x}{\partial \tau}\label {singularp1}
\end{align}
\begin{align}
\sdt{x} &= \frac{\partial ^2 x}{\partial t^2}+ \epsilon \frac{\partial ^2 x}{\partial t \partial \tau} + \epsilon \left[ \frac{\partial ^2 x}{\partial t \partial \tau}+ \epsilon \frac{\partial ^2 x}{\partial \tau ^2}\right] \\
%\end{align}
%\begin{align}
&=\frac{\partial ^2 x}{\partial t^2} + 2\epsilon \frac{\partial ^2 x}{\partial t \partial \tau}+ \epsilon^2  \frac{\partial ^2 x}{\partial \tau^2}.\label{singularp2}
\end{align}
We can substitute the equations above to Equ.~\eqref{alphasdt} to give
\begin{align}
\frac{\partial ^2 x}{\partial t^2}+ \ldots 
\end{align}
We can regard $\epsilon \sim \alpha $ (in the same order). 

Now let us expand $ x $ in terms of $ \epsilon $ as 
\begin{align}
x=x_0 + \epsilon x_1 + \epsilon^2 x_2^2 + \ldots .
\end{align}
The partial differential equation will give 
\begin{align}
\frac{\partial ^2 x_0 }{\partial t^2}+ x_0 &= 0,\label {partialpert1}
\end{align}
but you will not get $ x_0(t)=B\sin t $, rather $ x_0 (t)= B(\tau) \sin t $. 

PS: Finally, one should obtain $ x_0(t)=\frac{\sin t}{\sqrt{\cdots t}} $. The zero order solution can characterize the main feature of the exact solution of the equation!

The first order perturbation equation is
\begin{align}
\frac{\partial ^2 x_1}{\partial t^2} +x_1 = -2 \frac{\partial ^2 x_0}{\partial t \partial \tau} - \frac{\alpha}{\epsilon} \left( \frac{\partial x_0}{\partial t}\right) . \label {partialpert2}
\end{align}
...

Return to the linear problem which is 
\begin{align}
\sdt{x} + \alpha \dt{x} + \omega^2 x=0.\label {sdewithomega}
\end{align}
Divided by $ \alpha $ to give
\begin{align}
\left( \dfrac{1}{\alpha}\right)\sdt{x} + \dt {x} + \frac{\omega^2}{\alpha}x &=0 \\
\Rightarrow \dt{x} + \Gamma x &= -\frac{1}{\alpha } \sdt{x}.
\end{align}
In some sense $ \alpha \rightarrow \infty $ (Arrestotal approximation), the equation above gives
\begin{align}
\dt{x} +\Gamma x=0.  \label {Gammaapprox}
\end{align}

The solution gives a damping oscillation with two terms as shown in the figure below.
\scalefig{../Figs/handdraw918_2}{0.95}{Solution of a damping oscillation.}
%\missingfigure{918-2.}
If one term is ignored, it will become a pure damping movement. 

   
\textbf{Sep 20}

A summary of singular perturbation theory: For Equ.~\eqref{alphasdt}, using normal perturbation theory, one has
\begin{equation}
x(t)= x_0(t)+ \alpha x_1(t) +\alpha^2 x_2(t).
\end{equation}

\begin{equation}
\rightarrow \left\{ \begin{array}{c}
\sdt{x_0} + x_0 =0\Rightarrow x_0=B\sin t\\
\sdt{x_1}+ x_1 =\text{dangerous terms}
\end{array}\right.
\end{equation}
There is a dangerous term. Let us see how to avoid this dangerous term in the improved singular perturbation theory. 

Using the singular perturbation theory, we make $ x=x(\tau=\epsilon t,t) $ with perturbation coefficient $ \epsilon $ and/or $ \alpha $. 
The coupled partial differential Equs.~\eqref{singularp1} and~\eqref{singularp2} lead to 
Equs.~\eqref{partialpert1} and~\eqref{partialpert2}. 

Back to the result:
\begin{align}
\frac{\partial^2 x_1}{\partial t^2}+x_1 = -2 \frac{dB(\tau)}{d\tau}\cdot \cos t - \frac{\alpha}{\epsilon} B^3(\tau) \underbrace{\cos^3 t}_{\frac{1}{4}(3\cos t+ \cos 3t)}
\end{align}
What is dangerous? -- Terms proportional to $ \cos t $, which can blow off when $ t $ is large.  
Expand these terms to give 
\begin{align}
-2 \frac{d B(\tau)}{d\tau}- \frac{3}{4}\left(\frac{\alpha}{\epsilon} \right)B^3(\tau),
\end{align}
and put them equal to zero.
\begin{itemize}
\item This allows you to determine $ B(\tau) $. 
\item This ensures the removal of dangerous terms.
\end{itemize}

Interestingly, the zero-order approximation of this method has given a good approximation to the exact solution. The zero-order approximation gives
\begin{align}
x_0(t,\tau)=\frac{\sin t}{\sqrt{\frac{3}{4}\left(\frac{\alpha}{\epsilon} \right)\tau +1}}
\end{align}
Now, substitute $ \tau=\epsilon t $, we have
\begin{align}
x(t)\approx x_0 = \frac{\sin t}{\sqrt{\frac{3}{4}\alpha t +1}}.
\end{align}



\chapter{Memory function}\label{chap:memoryfunc}
Now, we look at another equation, which is Equ.~\eqref{sdewithomega}. It approximates to 
\begin{align}
\dt{x}+ \Gamma x=0, \label {sdewithomegaapprox}
\end{align}
where $ \Gamma= \frac{\omega^2}{\alpha} $. Now, we will go from Equ.~\eqref{sdewithomegaapprox} to Equ.~\eqref{sdewithomega} to study the meaning of the coefficients, and so on. 

Starting from Equ.~\eqref{sdewithomegaapprox}, we add a \textit{delay time} to get a \textit{delay-differential equation}:
\begin{align}
\dt{x(t)}= -\Gamma x(t-\tau), 
\end{align}
which is equivalent to 
\begin{align}
\dt{x(t+\tau)}= -\Gamma x(t).
\end{align}
Let us make $ f(t+\tau)= f(t)+ \tau \frac{d f}{dt}+\ldots $, we have
\begin{align}
\dt{x(t)}+\tau \sdt{x(t)}= -\Gamma x(t).
\end{align}
\begin{align}
\Rightarrow \sdt{x}+ \frac{1}{\tau} \dt{x}+ \frac{\Gamma}{\tau}x=0.
\end{align}
If we make $ \alpha = \frac{1}{\tau} $ and $ \frac{\Gamma}{\tau}=\omega^2 $, the equation above gives
Equ.~\eqref{sdewithomega}. 

Notice that Equ.~\eqref{sdewithomegaapprox} gives
\begin{align}
\epsilon \tilde{x}-x(0)+ \Gamma \tilde{x}&=0, \label {sdewithomegaapprox2}
\end{align}
\begin{align}
\boxed{\tilde{x}= \frac{x(0)}{\epsilon + \Gamma}}.
\end{align}
If  $ \Gamma $ were to be replaced by 
\begin{equation}
\tilde{\Gamma}(\epsilon)= \Gamma\cdot \frac{\alpha}{\epsilon + \alpha},
\end{equation}
\begin{align}
\Rightarrow \Gamma(t)= \Gamma \cdot \alpha e^{-\alpha t}.
\end{align}
As $ e^{-\alpha t}\xrightarrow{\text{Laplace Transform}} \frac{1}{\epsilon + \alpha} $ 
%(or we have made $ \tilde{\delta(t)} =1 $)
, we can transform an exponential function to a $\frac{1}{x}$ function. 
\scalefig{../Figs/handdraw920_1}{0.6}{Evolution of various functions.}
%\missingfigure{920-1}

Now, Equ.~\eqref{sdewithomegaapprox2} becomes
\begin{align}
\varepsilon \tilde{x}(\varepsilon)- x(0) + \frac{\omega^2}{\varepsilon + \alpha}\tilde{x}(\varepsilon)=0, \label {sdewithomegaapprox3} 
\end{align}
or, $ \tilde{x}= \frac{x(0)}{\epsilon + \frac{\Gamma \alpha}{\epsilon + \alpha}} $,
\begin{align}
\Rightarrow \tilde{x}= \frac{x(0)(\epsilon+ \alpha)}{\epsilon^2 + \epsilon \alpha +\Gamma \alpha}.
\end{align}
\begin{align}
\Rightarrow x(t)= e^{-\alpha t}[\cos \Omega \Gamma + \frac{\alpha}{2\Omega}\sin \Omega t], \label{sdewithomegaapprox4}
\end{align}
where $ \Omega= \sqrt{\omega^2 - \frac{\alpha^2}{4}} $, and we have made $ \Gamma \alpha= \omega^2 $ and used the Laplace transformations. 

If using the convolution theory on Equ.~\eqref{sdewithomegaapprox3} to invert
\begin{align}
\dt{x(t)}+ \int_0^t dt' \Gamma(t-t')x(t')=0, 
\end{align}
rather than Equ.~\eqref{sdewithomegaapprox}, we can get the memory function which we will discuss next. It can be verified by differentiation that this satisfies the
original differential equation. We have converted a second order equation to a first order
equation with a memory: this is the connection between the Newtonian second order view
and Aristotle?s first order view. In the limit $ \alpha\rightarrow \infty $, the memory function becomes a delta
function and the Aristotle first order equation is recovered.
%This is the damped harmonic oscillator. 

\textbf{Sep 25}

Recall of the damping oscillation problem we are working on: Equ.~\eqref{sdewithomega} can be approximated to Equ.~\eqref{Gammaapprox}, where $ \Gamma=\frac{\omega^2}{\alpha} $. There are several extreme cases: 
\begin{itemize}
\item $ \alpha \rightarrow \infty $
\item $ \omega \rightarrow \infty $,
\end{itemize}
but always $ \frac{\omega^2}{\alpha}=\Gamma $. 

If we want to generalize the equation as
\begin{align}
\dt{x(t)}+ \Gamma x(t)=0,
\end{align}
and  we have multiple timescales $ t_1,\, t_2, \ldots $ which are the times earlier than $ t $, and $ \Gamma x(t)= \Gamma_1 x(t_1)+ \Gamma_2 x(t_2)+\ldots $.
Therefore, 
\begin{align}
\dt{x(t)}+ \sum_{t_{\text{past}}}^t \Gamma_i x(t_i)=0.
\end{align}

%\missingfigure{925-1}.
\scalefig{../Figs/handdraw925_1}{0.5}{Time division before $ t $. }

If the time is continuous, we have
\begin{align}
\dt{x(t)}+ \int_0^t \Gamma(t-t') x(t')dt'=0.
\end{align}
This equation is called memory-possessing equation. The $ \Gamma(t) $ is called memory function. Through Fourier transformation, we can solve the equation above without knowing $ x(t) $. 

For instance, let $ \Gamma(t) \propto e^{-\alpha t} $, the equation above leads to 
\begin{align}
\dt{x(t)}+ \Gamma \alpha \int_0^t e^{-\alpha (t-t')} x(t')dt'=0.
\end{align}
This is the Volterra equation. 

By differentiating the equation above, we obtain
\begin{align}
\sdt{x(t)}+ \Gamma \alpha x (t)= 0.
\end{align}

\begin{align}
\rightarrow \epsilon \tilde{x} &-x(0)+ \frac{\Gamma \alpha}{\epsilon +\alpha}\tilde{x}=0,\\
\tilde{x} & = \frac{x(0)}{\epsilon + \frac{\Gamma \alpha}{\epsilon + \alpha}},
\end{align}
which can recover the result of Equ.~\eqref{sdewithomegaapprox4}.

Now look at the generalized equation
\begin{align}
\dt{x(t)}+ \Gamma \int_0^t \underbrace{\phi (t-t')}_{\phi(t)=\alpha e^{-\alpha t}}x(t')dt' =0,
\end{align}
where $ \int_0^\infty dt \phi(t)=1 $. Here is one way to solve it:
We differentiate the equation above to give
\begin{align}
\sdt{x(t)}+ \Gamma \int_0^t \dot{\phi}(t-t')x(t')dt'
+ \underbrace{\Gamma \phi(t-t)x(t)}_{\Gamma \phi(0)x(t) \rightarrow \Gamma\alpha x(t)=\omega^2 x(t)}.
\end{align} 
Notice that $ \dot{\phi}(t)=const.*\phi(t) $, which implies that $ \phi(t) $ is an exponential function. One possible case is that $ \phi(t) $ is a Gaussian function, whose Fourier transformation is itself. 

Q: can we turn any second-order equations into memory-possessing functions? 

Such equations are Aomeliiues, and are called non-Markoffian equations. These equations are memory-possessing or time non-local equations. 




\chapter {Synchronization and nonlinear systems}\label{chap:synchronization}
Q: what is meant by \textit{nonlinearity}? (nonlinear terms in equations). And so what? (the properties are changed, nonlinear phenomenon and their richness.) 

Two aspects to be explored in this section: toughness and richness.

Comparing 2 equaitons--one is linear, the other one is nonlinear:
\begin{align}
\dt{x} &=-\alpha x,\label {examplelinear1}\\
\dt{x} &=-\beta x^2.\label {examplenonlinear2}
\end{align}

Aside on synchronization. We let

\begin{align}
\dt{\theta_1} &=\omega_1 + l(\theta_2-\theta_1),\label {coupling1}\\
\dt{\theta_2} &= \omega_2 + l(\theta_1-\theta_2).\label {coupling2}
\end{align}

Two of them are coupled together. Add together to give
\begin{align}
\dt{(\theta_1+ \theta_2)} = \omega_1 + \omega_2.
\end{align}
Subtract them to give
\begin{align}
\dt{(\theta_1-\theta_2)}= (\omega_1 - \omega_2) -2l(\theta_1 - \theta_2).
\end{align}
Let $ D=\theta_1 - \theta_2  $, then 
\begin{align}
\dt{D} + 2lD=\omega_1-\omega_2.
\end{align}
If alternating $ \omega_1-\omega_2 \rightarrow \frac{\omega_1-\omega_2}{2l}$, we can write the equation in the form of
\begin{align}
\dt{y}+ \alpha y =B.
\end{align}
The diagram of $ y(t) $ is shown below.
%\missingfigure{925-2}
\scalefig{../Figs/handdraw925_2}{0.5}{Evolution of $ y(t) $.}

This coupling system is studied by Huygens before. If the interaction $ l(\theta_1 - \theta_2) $ is nonlinear, for example, the interaction term is $ \sin(\theta_1- \theta_2) $, there will be rich behaviors of the equation. The synchronization exists or not depends on the interaction term (whether it exists or not). 

The solution of the first equation (Equ.~\eqref{examplelinear1}) is
\begin{align}
x(t)= x(0) e^{-\alpha t}.
\end{align} 
The solution of the second equation (Equ.~\eqref{examplenonlinear2}) is
\begin{align}
x(t)= x(0)\frac{1}{1+x(0)\beta t}.
\end{align}
The second one depends on $ x(0) $, the first one does not. So, the second solution cannot be superimposed.


\textbf{Sep 27}

In a nonlinear case, for the coupling oscillation case, we can have
\begin{align}
\dt{\theta_1} &=\omega_1 - l\sin(\theta_2-\theta_1),\label {coupling1nonlinear}\\
\dt{\theta_2} &= \omega_2 - l\sin(\theta_1-\theta_2).\label {coupling2nonlinear}
\end{align}
Therefore, 
\begin{align}
\dt{(\theta_1 + \theta_2)}&= \omega_1+\omega_2\\
\dt{\theta}+2l\sin\theta &= \Delta,
\end{align}
where $ \theta=\theta_1-\theta_2 $ and $ \Delta=\omega_1 -\omega_2 $. 

Let us solve the equation that
\begin{align}
\dt{\theta}+2l\sin\theta=0.
\end{align}
We obtain 
\begin{align}
\int \frac{d\theta}{\sin \theta}=-2l t + const.
\end{align}
Now, the integral is
\begin{align}
\int \frac{d\theta}{\sin \theta} &= \int \frac{d\left(\theta/2 \right)}{\sin\left(\frac{\theta}{2} \right)\cos\left(\frac{\theta}{2} \right)} \\
&= \int \frac{d\left(\theta/2 \right) \frac{1}{\cos^2\frac{\theta}{2}}}{\dfrac{\sin\left(\frac{\theta}{2} \right)}{\cos\left(\frac{\theta}{2} \right)}}\\
&= \int \frac{dy\sec^2 y}{\tan^2 y}\\
&= \int \frac{d \left(\tan y \right)}{\tan y}\\
&= \ln \left(\tan y \right) ,\quad (\text{depending on the limits})
\end{align}
and the equation can be solved. 

In this lecture, we are going to answer the following questions:

\begin{itemize}
\item What exactly nonlinearity means?
\item What is bad about it?
\item What is good about it?
\end{itemize}

Back to our equations~\eqref{examplelinear1} and~\eqref{examplelinear2}. The solutions are 
\begin{equation}
\begin{array}{ccc}
\boxed{\frac{x(t)}{x(0)}=e^{-\alpha t}} & \text{versus} & \boxed{\frac{x(t)}{x(0)}=\frac{1}{x(0)\beta t +1}}.
\end{array}
\end{equation}

Concepts of Propagators or Green functions are USELESS for the second function case, which is a  nonlinear equation. That means superposition does not work in the nonlinear regime. 

There are two factors affecting the behaviour of the nonlinear equation:
one is the initial condition, the other is the stimulus response. 
If the two factors are both linear, the two factors work as the same effect.

This can be seen in as follows.
\begin{align}
\dt{\theta}+ 2l \theta &= \Delta\\
\Rightarrow \varepsilon \tilde{\theta} &= \theta(0)+2l \tilde{\theta}=\frac{\Delta}{\varepsilon}\\
\tilde{\theta}&= \theta(0)\frac{1}{\varepsilon +2l}+ \frac{\Delta}{\varepsilon}\frac{1}{\varepsilon +2l}
\end{align}
where the $ \frac{1}{\varepsilon + 2l} $ is the linear term of the propagator $ e^{-2lt} $.
\begin{align}
\Rightarrow \theta(t)= \theta(0) G(t)+ \int_0^t dt' G(t-t')\Delta.
\end{align}
The initial condition and the stimulus can be the same if they are linear.

With stimulus, $ S $,
\begin{align}
\dt{x}= -\alpha x +S\quad \text{versus} \quad \dt{x}= -\beta x^2 +S.
\end{align}
In the limit that $ t\rightarrow \infty $, $ x(\infty)=\frac{S}{\alpha} $. 
The general solution gives
\begin{align}
\int \frac{dx}{1-\dfrac{\beta}{S}x^2}= St+const.
\end{align}
The integral can be solved using the following relationship.
\begin{align}
\int \frac{dy}{1-y^2}= \int \frac{dy}{(1-y)(1+y)}=\frac{1}{2} \int \left[\frac{1}{1-y}+ \frac{1}{1+y} \right]dy.
\end{align}

If we have 
\begin{align}
\dt{x}=-\alpha x -\beta x^2- \delta x^3,
\end{align}
it can be solved similarly. 


\chapter{Singulation and bifurcation}\label{chap:bifurcation}
\textbf {Poincare \& Bruce Lee}
A little reminder on energy method, which can give a picture of the solution.

Let us look at the equation
\begin{align}
\dt{x}=f(x).
\end{align}

%\missingfigure{927-1}
\scalefig{../Figs/handdraw927_1}{0.7}{The concept of \textit{fixed} or \textit{stopping} point in the curve of displacement.}

If we look at the curve of $ f(x) $, there may be a fixed point, which is like the eigenvalue in the quantum mechanics, where the object will stop or oscillate around it. 

%\missingfigure{927-2}
\scalefig{../Figs/handdraw927_2}{0.8}{Linear shifting and nonlinear shifting.}
In Fig.~\ref{../Figs/handdraw927_2}, a shift in a linear function of $ f(x) $ generates nothing difficult to the equation; a shift in a nonlinear equation can cause some trouble on solving the equation, and can change the behavior of the equation. This is the starting point of \textit{singularization}.


\textbf{Oct 1}

\textbf{A discussion on equations in physics. } In quantum mechanics, for example, the quantum state sanwitch $ \bra{\Psi}O\ket{\Psi} $ can be nonlinear to describe a nonlinear property of an object. At the same time, the density operator $ \rho= \ket{\Psi}\bra{\Psi} $ obeys a linear equation,
\begin{align}
i\hbar \dt{\rho}=[H,\, \rho],
\end{align}
which is similar to the equation of $ \ket{\Psi} $. Now, since the observables are obtained by the trace defined by the density operator, which is $ \tr{\rho O} $, the properties of a quantum system can be linear or nonlinear. 

Now, look at the equation in classical mechanics,
\begin{align}
\dt{\rho}=\{H,\, \rho\},
\end{align}
which describes the fluid density changes in time (with Poisson brackets). 
%\missingfigure{Oct1-1. }
\scalefig{../Figs/handdraw1001_1}{0.6}{Movement of fluid flow described by density function.}
The movement of the fluid can also be described by a ``trace'' integral with the density function. 
\begin{align}
\int O(x,p)\rho(x,p,t)dxdp,
\end{align}
where 
\begin{align}
\rho(x,p,t)=\delta(x-x(t),p-p(t)).
\end{align}
The formalism is linear. 

Let us go back to the two-body coupling problem. 
\begin{align}
\dt{\theta}= \Delta -2l \sin \theta,
\end{align}
versus
\begin{align}
\dt{\theta}= \Delta -2l \theta.
\end{align}
One is nonlinear, the other is nonlinear. 
%\missingfigure{Oct-2.}
\scalefig{../Figs/handdraw1001_2}{0.7}{ Synchronization with shifts.}

If we regard $ \Delta-2l\sin \theta $ and $ \Delta-2l \theta $ as shifts of $ \dt{\theta} $, the former one cannot synchronize the movement, while latter one can. 

In general, we look at the equation
\begin{align}
\dt{x }= -\alpha x+S,\quad \textrm{and} \quad \dt{x}=-\beta x^2+S.
\end{align}
The shifts are shown in the figure below. 

%\missingfigure{Oct-3. }
\scalefig{../Figs/handdraw1001_3}{0.7}{Shifts for linear and nonlinear equations.}

For the shift of the second case (nonlinear case), something suddenly happened. This leads to the \textit{bifurcation} phenomenon. 

The method is that we plot the function of the stimulus, and analyze the property of the system. Let us make $ S=S(x) $. For example, $ S=\gamma x $. The shift is plotted in the figure below. 

%\missingfigure{Oct-4. }
\scalefig{../Figs/handdraw1001_4}{0.7}{Unstable of shifting.}

The shifts for the linear case are unstable, compared to the shift with a constant $ S $. The shifts for the nonlinear case is analog to the coherent state, or a damping spring (see the figure below). 

%\missingfigure{Oct-5. }  
\scalefig{../Figs/handdraw1001_5}{0.7}{Shift of spring with a linear force.}

The stable zero as a function of $ \gamma $ for the nonlinear shift is plotted in the figure below. This is called the transition of bifurcation. 

%\missingfigure{Oct-6.}
\scalefig{../Figs/handdraw1001_6}{0.7}{ Transition of bifurcation.}

Let us consider the equation
\begin{align}
\dt{x}=-\delta x^3 + \gamma x.
\end{align}

The shifts and the stable points changes are shown in the figure below. Not the only one stable zero point changes to three zero points.

%\missingfigure{Oct-7. }
\scalefig{../Figs/handdraw1001_7}{0.7}{Shifts and transition of bifurcation with cubic terms.}

This change corresponds to some high-density laser and magnetic phenomena. 

References on Bifurcations: Nicolism (Austin, Texas). 


\textbf{Oct 4.}

In a general form of a second-order equation
\begin{align}
\sdt{y}= F(y,\dt{y},t),
\end{align}
we usually have the Aristotal form
\begin{align}
\sdt{y} + \alpha \dt{y} =F(\ldots).
\end{align}
We can usually remove the second-order term to give the Newton form
\begin{align}
\dt{y}=f(y).
\end{align}

In most cases, we can write
\begin{align}
f(y)= \underbrace{f_{system}(y)}_{-\alpha y,\, -\beta y^2,\, -\delta y^3,\, -k\sin y\ldots}+ \underbrace{f_{applied}(y)}_{S\quad \textrm{and} \quad \gamma y}
\end{align}
The rich system terms give rich behavior of these differential equations. 

Now, we go to Strogatz's book on \textit{Nonlinear mechanics and Chaots}, Chapter 2. It shows the richness of the behaviors of differential equations. The idea is that we can plot the vector field in a graph, so that we can analyze the equation visibly. Some phrase to emphasize in the book: phase points, fixed points, stable/unstable points, population growth.

Question for the logistic equation under figure 2.3.2 in page 22: how to describe the meeting of two neighboring species in order to produce or kill some species?  Or, how to make it species-related? This is the homework.

The section on existence and uniqueness is optional. 

On the section of potential, there is a little note. From the $ v(t) $ plot we can obtain the potential plot (see below).

%\missingfigure{Oct4-1. }. 
\scalefig{../Figs/handdraw1004_1}{0.6}{Potential diagram corresponds to Fig.~\ref{../Figs/handdraw1001_7}.}

Exercise 2.3.4 is a homework. Exercise 2.5.6 on leaky bucket is interesting. 

Now go to chapter 3 of Strogatz's book, on bifurcation. On figure 3.4.7, it comes from the combination of $ x^3 $ and $ x^5 $ terms' plots. 

%\missingfigure{Oct4-2. }
\scalefig{../Figs/handdraw1004_2}{0.5}{Bifurcation diagram with both $ x^3 $ and $ x^5 $ terms. This plot is similar to that of explaining first/second-order phase change.}

Section 3.5 on rotating hoop is highly recommended to read carefully. Section 3.6 is good to read as well. 




% Field theory
\chapter{Many-body systems and field theory}



\section{Multiple-particle systems}\label{sec:particles}
\textbf{Oct 9.}

So far, we only studied the system with one variable in  one-dimension. We will study the field of interactions with two or many particles in multiple dimension. The theory can be called as classical field theory. The equations to describe such a multiple particle system are often partial differential equations (PDEs). To make things easy, we limit our discussions in linear regime at the beginning, and will extend to nonlinear regime and nonlinear PDEs. 

Aside on synchronization, a $ N $-particle system may be described as
\begin{align}
\dt{\theta_i}=\omega_i -\sum_j \gamma_{ij} f(\theta_i-\theta_j), \quad (i,j=1,2,\ldots,N).
\end{align}
To make it simple, let us make $ \gamma=\gamma_{ij} $ and $ f=\sin $ to give
\begin{align}
\dt{\theta_i}=\omega_i -\gamma \sum_j  \sin(\theta_i-\theta_j), \quad (i,j=1,2,\ldots,N).
\end{align}
In a range of $ \omega $ (see figure below), we can solve this equation system in computer. 
%\missingfigure{Oct9-1. Range of $ \omega_i $.}
\scalefig{../Figs/handdraw1009_1}{0.4}{Range of $ \omega_i $.}

This system of equations is an example of the first-order coupled equations
\begin{align}
\dt{x_i}=v_i.
\end{align}
This set of equations may be used to describe the synchronization movement of a school of fish, where $ v_i $ include the interactions among fish and the rule of movement of individual fish. 

Let us look at two-particle systems
\begin{align}
\dt{x_1} &= f(x_1,x_2),\\
\dt{x_2} &= g(x_2,x_1).
\end{align}
We may expect to obtain a set of equations with the combination of $ x_1 $ and $ x_2 $, as before, 
\begin{align}
\dt{x_1-x_2} &= F(x_1-x_2),\\
\dt{x_1+x_2} &= G(x_2+x_1).
\end{align}
However, in general case, we can hardly find such a equations set with defined combination of $ x_1 $ and $ x_2 $. 

If we can do so, for example,
\begin{align}
\dt{x_1} &= -k(x_1-x_2),\\
\dt{x_2} &= -k(x_2-x_1).
\end{align}
We can make $ x_- = x_1 - x_2 $ and $ x_+ = x_1+x_2 $ to solve and analyze the set of equations.

Another way to solving the coupled equations is by setting a superposition of states
\begin{align}
x^q= \sum_m a_m^q x_m,
\end{align}
where $ \sdt{x_m}=\ldots $ is known, 
and obtain a new set of equations
\begin{align}
\frac{d^2 x^q}{dt^2}=\ldots.
\end{align}
The number of the new set of equations is the same as the original system, but they are easier to solve. 

For instance, a dimer system with a spring can be described by
\begin{align}
M_1 \sdt{X_1}= -k (X_1- X_2),\\
M_2 \sdt{X_2}= -k (X_2- X_1).
\end{align}
%\missingfigure{Oct9-2. A dimer with a spring.}
\scalefig{../Figs/handdraw1009_2}{0.4}{A dimer with a spring.}
We can add more particles to make is general. When combining, some terms on the right can be
$ -k (X_m-X_{m-1} +X_m-X_{m-1}) = -2k (X_m-X_{m-1})$ or zero. 

%\missingfigure{Oct9-3. N-particle system connected with springs.}
\scalefig{../Figs/handdraw1009_3}{0.6}{N-particle system connected with springs.}
The equation can become
\begin{align}
\sdt{X_m}=-\frac{k}{M_m} (2 X_m- X_{m+1}-X_{m-1}), m=1,2,3,\ldots.
\end{align}
For springs
\begin{align}
\sdt{X_m}=-{\omega^2_m} (2 X_m- X_{m+1}-X_{m-1}), m=2,3,\ldots.
\end{align}
For simplicity, we make $ \omega_m=\omega_0 $, such that
\begin{align}
\sdt{X_m}=-{\omega^2_0} (2 X_m- X_{m+1}-X_{m-1}), m=2,3,\ldots.
\end{align}

This is a problem of eigen functions problem. One can solve it using linear algebra. 

We will solve it using the superposition method mentioned before. We make $ x^q=\sum_m X_m e^{iqm} $. We also suppose that the last mass in the chain is connected with the first mass labeled as $ m=1 $. We have
\begin{align}
-\frac{d^2 \sum_m X_m e^{iqm}}{\omega^2_0 dt^2}= [\sum_m X_m e^{iqm} - \sum_m X_{m-1} e^{iqm}- \sum_m X_{m+1}e^{iqm}],
\end{align}
such that
\begin{align}
-\frac{1}{\omega_0^2} \frac{d^2 x^q}{dt^2}= 2x^q - e^{+iq} \sum_m X_{m-1} e^{iq(m-1)}- e^{-iq} \sum_m X_{m+1} e^{iq(m+1)}.
\end{align}
Now, 
\begin{align}
\sdt{x^q}=-\omega_0^2 [2x^q - e^{iq} x^q -e^{-iq} x^q]= -\omega_0^2 x^q \cdot Q_q,
\end{align}
where $Q_q= 2-e^{iq}- e^{-iq}=2-2\cos q=4\sin^2 \frac{q}{2}$. Hence
\begin{align}
\sdt{x^q}+4\omega_0^2 \sin^2\frac{q}{2} x^q=0.
\end{align}
We can define $ \omega_q^2= 4\omega_0^2 \sin^2\frac{q}{2}$, the equation above can be simplified as
\begin{align}
\sdt{x^q}+\omega_q^2 x^q=0.
\end{align}

Under the boundary condition, $ N+1\rightarrow 1 $, 
\begin{align}
e^{iqm}=e^{iq(m+N)},
\end{align}
or,
\begin{align}
1=e^{iqN},\Rightarrow q=\frac{2\pi}{N} \, \textrm{are integers.}
\end{align}

\section{Homogeneous $N$-particle chain}
\textbf{Oct 23.}

The topic of this part is interaction between particles. See the figure below. 
\scalefig{../Figs/handdraw1023_1}{0.6}{Interacted particles.}

The equation to describe the interaction in the figure above can be given by
\begin{align}
M_m \ddot{x}_m =- k(2x_m-x_{m+1}-x_{m-1}).
\end{align}

We choose independent variables $ x^q=\sum_m a^q_m x_m $, and try derive a new set of equations of $ x^q $. In linear regime, we can always do it. Now, $ a^q_m $ can be very complicated. Usually, we can use $ a^q_m=e^{iqm} $. Here, $ m $ are numbers, the formalism of $ a^q_m $  is similar to Fourier transformation. $ q $ is dimensionless. Let us look at what is $ q $. 

We find that $ q=2\pi n $ ($ n $ is an integer) gives the same $ a^q_m $. Also, the periodical condition gives $ e^{iqm}=e^{iq(m+N)} $, which is a ring with particle number of $ N $. Based on the two facts above, we can write $ q $ as 
\begin{align}
q=\frac{2\pi}{N}m\quad (m=0,1,\ldots , N-1).
\end{align}
This is the reciprocity relationship between $ m $ and $ q $. It is also a form of discrete Fourier transformation. 

\scalefig{../Figs/handdraw1023_2}{0.6}{Two ways of forming a ring with $ N $ particles. }

As $ N $ becomes larger and larger, $ q $ becomes smaller and smaller. The summation  becomes an integral. So it is a Fourier transformation from $ x_m(t) $ to $ x^q(t) $. In the ring or closed chain. $ q $ is the phase ranging from $ -\pi $ to $ \pi $, or $ 0 $ to $ 2\pi $. 

To solve the motion of particles, we use the idea: since the messy equation of $ x_m $ is so complicated, we convert the $ x_m $ equation to the equation of $ x^q_m $ using the discrete Fourier transformation. Once we obtain the solution of $ x^q_m $, we invert them into $ x_m $ again. And the inverting transformation gives
\begin{align}
x_m=\frac{1}{N}\sum_q x^q e^{-iqm}.
\end{align}

If $ N $ is large, the summation becomes an integral, which gives
\begin{align}
x_m=\frac{1}{2\pi}\int_{-\pi}^{\pi} x^q(t) e^{-iqm}dq
\end{align}

However, the system of equations is still complicated, because each particle has a different frequency. Formally,
\begin{align}
\sdt{x^q}+\omega_q^2 x^q=0.
\end{align}
$ \omega_q $ is the intrinsic frequency for one particle. What is it? Usually, using the boundary condition, the frequency can be written as $ \omega_q=2\omega_0 \left| \sin \frac{q}{2} \right| $. The frequency band is formed as shown in the figure below. This is called dispersion relations. The shape of the dispersion relation depends on the range of $ N $. 

\scalefig{../Figs/handdraw1023_3}{0.5}{Band structure of collective oscillations. }

Using the expression of $ \omega_q $, the solution of $ x^q $ is usually a Bessel function (in the form of $ \sin\sin(\cdot) $). See Homework8-2 for details. 

Problem: 
\scalefig{../Figs/handdraw1023_4}{0.5}{A chain of particles in a spring.}
The initial condition is 
\begin{align}
x_1(0)= x_2(0)&= x_3(0)=x_4(0)=0,\\
v_1(0)=v_0, \quad  v_2(0)=-v_0, &\quad v_3(0)=v_4(0)=0.
\end{align}
We want to solve $ x_3(t) $. 

To solve it, we can write down the equation of $ x^q $ as
\begin{align}
x^q(0)&= \sum_m x_m(0) e^{iqm}=0.\\
\dot{x}^q(0) &=\sum_m \dot{x}_m (0) e^{iqm}\nonumber \\
&= v_0 e^{iq1}+ 0+0-v_0 e^{iq4}\\
&= v_0(e^{iq1}-e^{iq4}),
\end{align}
where $ q=\frac{2\pi}{N}(0,1,2,3)=0,\,\frac{\pi}{2},\,\pi,\,\frac{3\pi}{2} $. For example, for $ q=\pi $, we have
\begin{align}
\dot{x} ^\pi (0)=v_0 (e^{i\pi}-e^{i4\pi}).
\end{align}
One can solve the other $ x^q $.

Now, from $ \omega_q= 2\omega_0 \left|\sin\frac{q}{2} \right| $, one can solve the individual $ \omega_q $. For instance, 
\begin{align}
\omega_{\pi/2}= 2\omega_0 \left|\sin\frac{\pi}{4} \right|.
\end{align}

Finally, 
\begin{align}
x_3(t)=\frac{1}{N} \sum_{q=0,\,\frac{\pi}{2},\,\pi,\,\frac{3\pi}{2} } e^{-iq3}x^q(t)
\end{align}
\begin{align}
x_3(t) &= \frac{1}{4}\left[x^0 (t)+e^{-i\frac{3\pi}{2}}x^{\pi/2}(t) + e^{-i3\pi}x^\pi (t) + e^{-i\frac{9\pi}{2}}x^{3\pi/2}(t) \right]\\
&= \cdots
\end{align}

This method can be used to solve similar problems with $ N $ particles in a chain. 

\textbf{Oct 25.}


Summary of what we have learnt recently:
\begin{align}
x^q=\sum_m x_m e^{iqm},\quad x_m=\frac{1}{N}\sum_q x^q e^{-iqm},
\end{align}
where $ q=\frac{2\pi}{N}(0,1,\cdots,N-1) $.
$ \omega_q $ depends on interactions, \textit{e.g.} 
\begin{align}
\omega_q^2=4\omega_0^2 \sin^2 \frac{q}{2}.
\end{align}
As $ N\rightarrow \infty $,
\begin{align}
x_m=\frac{1}{2\pi} \int_{-\pi}^{+\pi} x^q e^{-iqm} dq.
\end{align}
This will lead to Bessel function, \textit{etc.} 

Assume we have a chain of particles connected with springs. 
\begin{align}
x_0(t)&=\frac{1}{2\pi} \int_{-\pi}^{+\pi} x^q(t)e^{-iqm} dq|_{m=0}\\
x_m(t)&=\frac{1}{2\pi} \int_{-\pi}^{+\pi} x^q(t)e^{-iqm} dq \label{eq:xmfromxq}\\
x^q(t) &=A\cos \omega^q t+ B \sin \omega^q t\\
x^q(0) &=A,\quad \dot{x}^q(0)=B\omega^q.
\end{align}
\begin{align}
x^q(0)&=0,\\
\dot{x}^q(0)&=\sum_m \dot{x}_m(0)e^{iqm}\nonumber\\
&= v_0 \sum_m \delta_{m,0} e^{iqm}=v_0.
\end{align}
Suppose
\begin{align}
A=0, \quad B=\frac{v_0}{\omega^q},
\end{align}
we have
\begin{align}
x^q(t)=\frac{v_0}{\omega^q}\sin \omega^q t.
\end{align}
Substituting the equation above back into Equ.~\ref{eq:xmfromxq}, we will get the Bessel function expression.

\section{Periodical chain}\label{sec:periodicchain}
\subsection{Cell method}
Now we look at a chain with periodical units of small mass and big mass. See figure below.

\scalefig{../Figs/handdraw1025_1}{0.7}{Chain of periodical/alternating masses.}

If the masses of the two particles are the same, we have 
\begin{align}
\sdt{x_0}&=-\omega^2 (x_0-x_1)\\
\sdt{x_1}&=-\omega^2 (x_1-x_0).
\end{align}
We consider to make 
\begin{align}
x_0+x_1 &=x^+\\
x_0-x_1 &= x^-.
\end{align}
We will have two modes ($0$ and $ 1 $) for the movement of the center and the relative movement. 

If the masses of the two particles are different, the frequencies for the two particles are different, so that
\begin{align}
\sdt{x_0}&=-\omega_0^2 (x_0-x_1)\\
\sdt{x_1}&=-\omega_1^2 (x_1-x_0).
\end{align}
We can rewrite it in the form of matrix as 
\begin{equation}
\sdt{}\left(\begin{array}{c}
x_0\\x_1
\end{array} \right) = -\left( \begin{array}{cc}
\omega_0^2 & -\omega_0^2\\
-\omega_1^2 & \omega_1^2
\end{array}  \right)
\left( \begin{array}{c}
x_0\\x_1
\end{array} \right).
\end{equation}

A way to look at the equations is to change the variables as shown in Fig.~\ref{../Figs/handdraw1025_1} and treat each unit as an entity. The equations becomes
\begin{align}
M_L \ddot{L_m} e^{iqm} &= -\sum_m k(L_m -R_m + L_m -R_{m-1}) e^{iqm} \label{eq:leftm}\\
{M}_R \ddot{R}_m e^{iqm} &= -\sum_m k (R_m -L_m+R_m -L_{m-1}) e^{iqm} \label{eq:rightm}
\end{align}
\begin{align}
L^q=\sum_m L_m e^{iqm}.
\end{align}

We only look at Equ.~\ref{eq:leftm}, for example, and obtain
\begin{align}
M_ L \ddot{L}^q -2kL^q + kR^q +k\sum_{m=1} R_{m-1}e^{iq(m-1)}\cdot e^{iq}
\end{align}
\begin{align}
\fbox{$M_L\ddot{L}^q=-2kL^q +k(1+e^{iq})R^q$}. 
\end{align}

This equation shows that we can  transfer the individual equations of particles to two equations only involving the particles on the left and right in the unit. 

The equation set looks like this:
\begin{align}
\sdt{}\left(\begin{array}{c}
L^q\\ R^q
\end{array} \right) 
= - \left( \begin{array}{cc}
a_{11} & a_{12}\\
a_{21} & a_{22}
\end{array} \right)\left( \begin{array}{c}
L^q\\
R^q
\end{array} \right)
\end{align}

The solution gives the dispersion relationships as plotted in the figure below. The feature of the relationship is that there is a forbidden band between the two modes. 

\scalefig{../Figs/handdraw1025_2}{0.5}{A dispersion relationships of periodical oscillations.}

Q: is it possible to have a forbidden gap/band? It happens when the two masses of a unit are equal. See below. The result is that the reflection will go away as well.
\scalefig{../Figs/handdraw1025_3}{0.5}{Dispersion without a forbidden band.}

Notice that when $ M_L\rightarrow M_R $, the dispersion relationship should become that of homogeneous chain case. That means the upper band is gone in that extreme case. In fact, the upper band curve moves as the extension of the one-particle chain. 

\subsection{Alternating method}
For the next class: Define a function $ h(m) $ of an integer $ m $ such that 
\begin{align}
h(m)= 1 \quad \mathrm{or}\quad  0
\end{align}
according as  $ m $ is even or odd. $ h(m) $ can be in the following form
\begin{align}
h(m)=\frac{1+(-1)^m}{2}=\frac{1+(\cos \pi)^m}{2}=\frac{1+(\cos \pi+i\sin \pi)^m}{2}=\frac{1+e^{im\pi}}{2}.
\end{align}

\textbf{Oct 30.}

Suppose the large mass and the small mass are $ M \& \mu $. We have the equation of motion given by
\begin{align}
[M+(\mu - M)(\frac{(1+(-1)^m)}{2}) ]\sdt{x_m} &= -k(2x_m - x_{m-1}-x_{m+1})\\
\Leftrightarrow \frac{\mu + M}{2} \sdt{x_m} + (-1)^m \frac{\mu - M}{2} \sdt{x_m} &= -k(2x_m - x_{m-1}-x_{m+1}).
\end{align}

Next, we need to perform the discrete Fourier transformation to simplify the equations. Of importance is the terms of 
\begin{align}
\sum_m (-1)^m x_m e^{iqm}=\sum_m e^{i\pi m } x_m e^{iqm}.
\end{align}
We will get nothing but $ x^{q+\pi} $. And the equation of $ \ddot{x}^q $ will connect with $ x^q $ and $ x^{q+\pi} $. The terms $ \ddot{x}^{q+\pi} \sim x^{q+\pi}\, \& \, x^q$.  

In this form, we can only focus on two samples of normalized particles with masses of $ \mu_+=\frac{\mu + M}{2} $ and $ \mu_-=\frac{\mu-M}{2} $ respectively in the Fourier domain to study the collective movement of all particles. See Homework 8-3 for details. 

Q: what we will do if we have multiple particles in a unit cell. See Fig.~\ref{../Figs/handdraw1030_1}. 
\scalefig{../Figs/handdraw1030_1}{0.6}{A chain with multiple particle in one unit cell. }

It can be solved in a similar way by focusing on the unit cell under discrete Fourier transformation. 

\section{Defects in a propagating chain}\label{sec:defects}
Q: Suppose we have a homogeneous chain of masses except for one mass given by $ \mu $ (see Fig.~\ref{../Figs/handdraw1030_2}). 
\scalefig{../Figs/handdraw1030_2}{0.6}{A chain with a defect mass $ \mu $.}
The equation of motion can be given by
\begin{align}
[M+(\mu - M)\delta_{mn} ]\sdt{x_m} = -k(2x_m - x_{m-1}-x_{m+1})
\end{align}
We label the $ \mu  $ mass as number $ n=0 $ mass. That is
\begin{align}
M\sdt{x_m}  &= -k (2x_m - x_{m+1}-x_{m-1})\quad \mathrm{if} \, m\neq 0 \\
\mu \sdt{x_0} &= -k (2x_0 - x_1 - x_{-1}).
\end{align}

\begin{align}
\rightarrow \quad \sum_m (\mu - M)\delta_{m,0} x_m e^{iqm}= (\mu-M) x_0.
\end{align}

Now, it is hard to solve the problem using the FT method mentioned before. To solve it, we should know that the equation can be rewritten in the form of
\begin{align}
\fbox{$\dt{y_m}=  \sim_m$} -c \delta_{m,0}y_m. \label{eq:nonhomogeneous}
\end{align}
The terms in the frame has the solution in the form of
\begin{align}
y_m(t)= \sum_n G_{m-n}(t)y_n (0).
\end{align}
This is the solution for a transition of homogeneous propagator problem, in terms of Green's function term $ G_{m-n}(t) $. The $ \delta $ term gives a trouble. 

Note that 
\begin{align}
\dt{z(t)}=-\alpha z
\end{align}
has a solution of 
\begin{align}
z(t)= e^{-\alpha t}z(0),
\end{align}
even if $ z $ is an operator. If we change the equation above to be
\begin{align}
\dt{z(t)}=-\alpha z +f(t),
\end{align}
the solution can be given by
\begin{align}
z(t)= e^{-\alpha t}z(0) +\int_0^t dt' e^{-\alpha (t-t')} f(t').
\end{align}

Analog to the case above, Equ.~\ref{eq:nonhomogeneous} has the solution of
\begin{align}
y_m(t)= \sum_n G_{m-n}(t)y_n (0) + \int_0^t dt' \triangle .
\end{align}

In general, we come to the equation that
\begin{align}
\dt{y_m} = \sim_m +g_m (t).
\end{align}
We can verify that the solution can be rewritten as
\begin{align}
y_m =\sum_n G_{m-n}(t) y_n(0) + \int_0^t dt' \sum_n G_{m-n}(t-t') g_n (t'). 
\end{align}
This is a summation of convolutions in space and in time. So, it is a transition of variables in space and so to in time. The first term is a Fourier transformation, while the second term is a Laplace transformation. 
If $ g_m (t) $ is given, we can solve it in the form above. 

Now, we go back to our defect equation. We write
\begin{align}
\tilde{y}_m (\epsilon)= \underbrace{\sum_{n} \tilde{G} _{m-n }(\epsilon) y_n (0)}_{\tilde{\eta}_m(\epsilon)} + \sum_n \tilde{G}_{m-n}(\epsilon) \tilde{g}_n(\epsilon)
\end{align}
Therefore,
\begin{align}
\tilde{y}_m= \tilde{\eta}_m -c \sum_n \tilde{G}_{m-n}(\epsilon) \delta_{n,0} \tilde{y}_n (\epsilon)=\tilde{\eta}_m-c\tilde{G}_m\tilde{y}_0,
\end{align}
where we have made $ \tilde{g}_n(\epsilon)=-c\delta_{n,0}\tilde{y}_n(\epsilon) $.

The solution above is a pseudo-solution, because there are unknown terms. 

If $ c $ is small, we can use the iteration method to solve it. That is
\begin{align}
\tilde{y}_m=\tilde{\eta}_m-c\tilde{G}_m[\tilde{\eta}_0-c\tilde{G}_0[\tilde{\eta}_0-c\tilde{G}_0[\cdots]]]
\end{align}
Alternatively, we start with $ m=0 $, $\tilde{y}_0=\tilde{\eta}_0-c\tilde{G}_0\tilde{y}_0  $, and solve $ \tilde{y}_0=\frac{\tilde{\eta}_0}{1+c\tilde{G}_0} $ and so on. 

If $ c $ is large or the interaction is strong, 
\begin{align}
\tilde{y}_m = \tilde{\eta}_m -c \frac{\tilde{G}_m\tilde{\eta}_0}{1+c\tilde{G}_0}.
\end{align}

Above, we find a way to solve a first-order equation of Equ.~\ref{eq:nonhomogeneous}. If we want to solve the second-order equation of the chain with a defect, we need to transfer between the two forms of equations. 



\textbf{Nov 1.}

A summary of technique we learned in the earlier sections: Some techniques to solve this kind of transitions between multiple interacting particles, which has the equation of motion
\begin{align}
\sdt{x}+ Ax=0,
\end{align}
where 
\begin{align}
x=\left( \begin{array}{c}
x_1\\
\vdots\\
x_m\\
\vdots
\end{array} \right).
\end{align}
This is equivalent to 
\begin{align}
\sdt{x_m}+ \sum_n A_{mn}x_n=0.
\end{align}
We can use the Fourier transformation of
\begin{align}
x^q=\sum_m e^{iqm},
\end{align}
so that the equation can be transferred to 
\begin{align}
\sdt{x^q}=-A^q x^q.
\end{align}

Moreover, if we have driving equation
\begin{align}
\sdt{x}+ Ax=f,
\end{align}
or 
\begin{align}
\sdt{x_m}+ \sum_n A_{m-n}x_n=f_m(t),
\end{align}
we can write the solution of the Green function with a propagator for the homogeneous equation first, which is
\begin{align}
x_m(t) =\sum_n G_{m-n}(t)x_m (0)=\sum_n \eta_m (t)
\end{align}
and then plus the driving term to give 
\begin{align}
x_m(t) =\sum_n \eta_m (t) + \int_0^t dt' \sum_n G_{m-n}(t-t')f_n(t').
\end{align}

For the case of defects, if the defect is not ineligible, we can use the formalism of solution to solve the problem explicitly. The key for solving this kind of equations is the Laplace transformation and its inversion, which will be taught in class.

By solving the last part of the solution using Laplace transformation technique, and put the solution back into the equation, we can obtain the full solution of the problem. This is known as \textit{renormalization} technique. 

\section{Field theory}\label{sec:field}
Now, we start a new topic. We look at the homogeneous chain problem.
\begin{align}
\sdt{x_m}= -\omega_0^2 (2x_m-x_{m+1}-x_{m-1} ).\label{eqfield:xm}
\end{align}
We rewrite it as
\begin{align}
\sdt{x_m}= a \omega_0^2 \left( \frac{(x_{m+1}-x_m)}{a}-\frac{(x_m-x_{m-1})}{a} \right).
\end{align}
We rewrite it further more,
\begin{align}
\sdt{\Psi_m}= a^2 \omega_0^2 \frac{\left( \frac{(\Psi_{m+1}-\Psi_m)}{a}-\frac{(\Psi_m-\Psi_{m-1})}{a} \right)}{a}.
\end{align}
If we take the limit $ a\rightarrow 0$, and treat $ \Psi=\Psi(x,t) $, this equation above may give us the equation of
\begin{align}
\spt{\Psi(x,t)}=\diamondsuit  \spp{\Psi(x,t)}{x},
\end{align}
which is the wave equation. This works only when we take the limit of $ a $. The term $ \diamondsuit=c^2=a^2\omega_0^2 $ is the square of the speed of wave. Now, we obtain the equation of a field
\begin{align}
\spt{\Psi(x,t)}=c^2  \spp{\Psi(x,t)}{x}.
\end{align}

If we have damping term, we should add the term of $ -B\dt{x_m} $ into Equ.~\ref{eqfield:xm}. Therefore, the field equation becomes
\begin{align}
\spt{\Psi(x,t)}+\alpha \pt{\Psi(x,t)}=c^2  \spp{\Psi(x,t)}{x}.
\end{align}
This is called the telegrapher's equation. 

When $ \alpha\rightarrow 0 $, the damping terms disappears. When $ \alpha\rightarrow \infty $, only when $ D=\frac{c^2}{\alpha} $ is a constant, the equation of field becomes Fourier equation or diffusion equation:
\begin{align}
\pt{\Psi(x,t)}=D\spp{\Psi(x,t)}{x}.
\end{align}
This is also a form of Schrodinger equation. More generally, 
\begin{align}
\pt{\Psi(x,t)}=D\spp{\Psi(x,t)}{x}+a\Psi(x,t)-b\Psi^2(x,t)
\end{align}
is called Fisher's equation. Consider the competition interaction to be non-local in space:
\begin{align}
\pt{\Psi(x,t)}=D\spp{\Psi(x,t)}{x}+a\Psi(x,t)-b\Psi(x,t)\int  \underbrace{f(x-y)}_{influence function}\Psi(y,t)dy.
\end{align}
The influence function gives rich patterns in nature. To study the characteristic of the equation above, we can put 
\begin{align}
\Psi(x,t)= \Psi_0 + c\cos(kx)\exp(\phi t)
\end{align}
into the equation to give
\begin{align}
\phi=-Dk^2 -a F(k),
\end{align}
where 
\begin{align}
F(k)=\int \cos(kz) f(z) dz,
\end{align}
which is the Fourier transformation of the influence function. Condition for steady state patterns is
\begin{align}
\lambda>2\pi \sqrt{\frac{D}{-a F(\lambda)}}.
\end{align}
Diffusion if too strong will wash out patterns. Influence function should have a sharp cut-off. Some parameters may influence the pattern: wavelength of the pattern, cut-off length, and so on (the notes on Fisher's equation comes from Friday seminar by Dr. Kenkre). 


\textbf{Nov 6.}

Content we discussed for continuum limit can be seen in Fig.~\ref{../Figs/handdraw1106_1}.
\scalefig{../Figs/handdraw1106_1}{0.6}{ To the continuum limit.}

The equation of field becomes
\begin{align}
\spt{\Psi(x,t)} + \alpha \pt{\Psi(x,t)} = c^2 \spp {\Psi(x,t)}{x}.
\end{align}
The second term has the following limits:
\begin{itemize}
\item $ \alpha\rightarrow 0 $: Wave equation: $ \spt{\Psi}=c^2 \spp{\Psi}{x} $.
\item $ \alpha,c \rightarrow \infty, \frac{c^2}{\alpha}=D $: Diffusion equation $ \pt{\Psi}=D \spp{\Psi}{x} $. 
\end{itemize}

Instead of $ \Psi $, we use $ P $ and can obtain the diffusion equation: $ \pt{P(x,t)} = D \spp{P(x,t)}{x} $. 

To solve the equation above, we can use the Laplace/Fourier transformation (trial). We make the following transformation relationships:
\begin{align}
\hat{P} (k,t) &= \int_{-\infty}^{\infty} P(x,t) e^{ikx} dx,\\
P(x,t) &= \frac{1}{2\pi} \int_{-\infty} ^ {\infty} \hat{P}(k,t) e^{-ikx} dk. 
\end{align}

We also define the $ \delta $-function as
\begin{align}
\delta{(k)} &= \frac{1}{2\pi} \int_{-\infty} ^ {\infty} e^{ikx} dx. 
\end{align}

We disturb the continuum of $ x $ with a bump (See Fig.~\ref{../Figs/handdraw1106_2}). 
\scalefig{../Figs/handdraw1106_2}{0.6}{ Break of a continuum medium.}
The Fourier transformation of the diffusion equation gives
\begin{align}
\pt{\hat{P}(k,t)} &= -D k^2 \hat{P}(k,t). \label{eq:hatpk}
\end{align}

Compared with the discrete second-derivative of a similar equation, we found that there is a $ Dk^2 $ term instead of $ \omega ^2 (1-\cos q) $. This is the result of the limit of $ a\rightarrow 0 $. To see this, we write
\begin{align}
e^{i\frac{q}{a} ma} \sim e^{ikx}=e^{ikma},
\end{align}
and
\begin{align}
1-\cos q= 1- \cos\frac{q}{a} a = 1- \cos k\cdot a = 1- (1- \frac{k^2a^2}{2!} +\ldots ) \rightarrow \frac{k^2 a^2}{2}. 
\end{align}

The equation of \ref{eq:hatpk} has a solution given by
\begin{align}
\hat{P}(x,t) &= \hat{P} (k,0) e^{-Dk^2t}\\
P(x,t) &= \frac{1}{2\pi}\int_{-\infty}^{\infty} \hat{P}(k,0) e^{-Dk^2 t} e^{-ikx} dk,
\end{align}
where
\begin{align}
\hat{P} (k,0) = \int_{-\infty}^{\infty} P(y,0)e^{iky} dy.
\end{align}

We can obtain
\begin{align}
P(x,t) = \frac{1}{2\pi} \int_{-\infty} ^ {\infty}  dy \int_{-\infty}^{\infty} dk e^{-Dk^2t} e^{-ikx } e^{+iky} P(y,0).
\end{align}
If $ P(x,0) = \delta (x) $, the integral gives
\begin{align}
G (x-y) =  \frac{1}{2\pi} \int_{-\infty} ^ {\infty} e^{-Dk^2t}e^{-ik(x-y)} dk.
\end{align}
We have
\begin{align}
\fbox{$ P(x,t) = \int dy G(x-y,t) P(y,0) .$}
\end{align}

Some details to derive the equation above. 

Job1: To get 
\begin{align}
\frac{1}{2\pi} \int_{-\infty}^{\infty} e^{-Dk^2 t } e^{-ikz} dz = \frac{e^{-\frac{z^2}{4Dt}}}{\sqrt{4\pi Dt}}.
\end{align}
The solution looks like a Gaussian function ($ e^{-z^2} $). 
\scalefig{../Figs/handdraw1106_3}{0.6}{ Two Gaussian shape lines with different damping coefficients.}

To obtain the Gaussian integral above, we can complete the function to give
\begin{align}
\int \exp\{-Dt[\underbrace{k^2 + \frac{2ikz}{2Dt} + (\frac{iz}{2Dt})^2}_{ (k+ \frac{iz}{2Dt})^2 } - (\frac{iz}{2Dt})^2 ] \} dk,
\end{align}
and use the fact that 
\begin{align}
\int _{-\infty}^\infty e^{-bz^2} dz = \sqrt{\frac{\pi}{b}}.
\end{align}

Job2: to solve 
\begin{align}
P(x,t) = \int G(x-y) P(y,0) dy = \int \frac{e^{\frac{-(x-y)^2}{4 Dt }}}{\sqrt{4 \pi Dt}} \cdot \sqcap(y) dy.
\end{align}
Now we arrive at the error function in the form of
\begin{align}
h \int _{-l} ^{+l} \frac{e^{\frac{-(x-y)^2}{4 Dt }}}{\sqrt{4 \pi Dt}} dy.
\end{align}

Q: How to solve arbitrary problem such as $ \braket{x^2}= \int_{-\infty}^\infty x^2 P(x,t)dx $. 

One way to solve it is to realize that we only need to calculate
\begin{align}
\braket{x^2}=-\spp{\hat{P}(k,t)}{k}. 
\end{align}
This leads to the famous Einstein equation which gives $ \braket{x^2}\sim t $. 

Totally unrelated way to get the diffusion equation: 
suppose we go back to the chain of masses, and a person walk through it. Suppose the walker has no goal and is randomly walking back and forth. What is $ \dt{P_m (t)} $? 
\begin{align}
\dt{P_m(t)} = F(-P_m -P_m + P_{m+1}+ P_{m-1}).
\end{align}
\begin{align}
\rightarrow \pt{P(x,t)} = D \spp{P(x,t)}{x}, \label{eq:walkereq}
\end{align}
where $ D=\lim_{a\rightarrow 0} Fa^2$. See Fig.\ref{../Figs/handdraw1106_4}.
\scalefig{../Figs/handdraw1106_4}{0.6}{ Walker on a chain as a way to get the Diffusion equaiton.}

\textbf{Nov 8.}

We will continue to discuss Equ.~\ref{eq:walkereq}. It can be used in several scenarios: one is for spring connected masses, another one is a random walker in a chain; also, it can be a string with a raising force at one point (see Fig.~\ref{../Figs/handdraw1108_1}). 
\scalefig{../Figs/handdraw1108_1}{0.6} {A model of string.}

The solution can be obtained from
\begin{align}
\hat{P}(k,t)=\int_{-\infty}^\infty P(x,t)e^{ikx} dx,
\end{align}
and 
\begin{align}
P(x,t) = \frac{1}{2\pi} \int_{-\infty}^\infty \hat{P}(k,t)e^{-ikx} dk.
\end{align}

We know that 
\begin{align}
\frac{1}{2\pi} \int_{-\infty}^\infty e^{ikx} dk &= \delta(x),\\
\frac{1}{2\pi} \int_{-\infty}^\infty e^{ikx} dx &= \delta(k),\\
G(x,t) &= \frac{e^{-\frac{x^2}{4Dt}}}{\sqrt{4\pi Dt}}
\end{align}
There will be a homework on solving the equation of diffusion. 

For arbitrary equation with $\braket{x^n}=\int_{-\infty}^\infty x^n P(x,t) dx  $ (take $ n=2 $ for example), we can derivative it to give
\begin{align}
\dt{\braket{x^2}} = D \int_{-\infty}^\infty x^2 \spp{P(x,t)}{x} dx . 
\end{align}
If the integral above gives $ 2D $, we can get the Einstein equation
\begin{align}
\braket{x^2} = \braket{x^2}_{t=0} + 2Dt.
\end{align}
Hence, the mean value of $ x^2 $ connected with the initial value and a linear term of $ t $. 

Notice that we have supposed that $ D $ is a constant. Q: what if $ D $ is dependent to $ t $ or space? If $ D=D(t) $, we have
\begin{align}
\frac{1}{D(t)} \pt{P(x,t)} = \spp{P(x,t)}{x},
\end{align}
or 
\begin{align}
\pp{P(x,\tau)}{\tau} = l \spp{P(x,\tau)}{x},
\end{align}
where $ d\tau = D(t)dt $, or $ \tau = \int^t D(s) ds $. Therefore, $ \braket{x^2} $ can be solved. 

For $ D=D(x) $, we have
\begin{align}
\pt{P} = D(x) \spp{P}{x}.
\end{align}
We may want to find out what is the relationship to $ \frac{\partial}{\partial x} \left[ D(x) \pp{P}{x} \right] $. 

All the solving process is based on Fourier transform. Now, let us try using Laplace transform. We have
\begin{align}
\epsilon \tilde{P} (x,\epsilon) - P(x,0) -D\spp{\tilde{P}(x,\epsilon)}{x} = 0.
\end{align}
\begin{align}
\spp{\tilde{P}(x,\epsilon)}{x} = \frac{\epsilon}{D}  \tilde{P} (x,\epsilon) - \frac{1}{D} P(x,0).
\end{align}

We can get
\begin{align}
\sdx{Z(x)}{x} = A Z(x) -B (x).
\end{align}
Similarly, it can be solved with a given boundary condition. There are detailed discussion in Fetter's book. 

If there is no damping, we have from the telegrapher's equation that
\begin{align}
\spt{P(x,t)} = c^2 \spp{P(x,t)}{x}. \label{eq:nodamping}
\end{align}
Now, we have
\begin{align}
\sdt{\braket{x^2}} = 2c^2.
\end{align}
Further, 
\begin{align}
\dt{\braket{x^2}} = \underbrace{\left[ \dt{\braket{x^2}} \right]_{t=0}}_{\textrm{initial}} + c^2 t^2. 
\end{align}

By using Fourier transform, we obtain from Equ.~\ref{eq:nodamping}
\begin{align}
\sdt{\hat{P}(k,t)} + k^2 c^2 \hat{P}(k,t) = 0.
\end{align}
So,
\begin{align}
\hat{P}(k,t) = \hat{P}(k,0) \cos (kct) + B(k) \sin(kct).
\end{align}
The $ B(k) $ is related to $ \pt{P(x,t)} $. Because the FT of $ e^{ikct} $ is in the form of $ \delta(x\pm ct) $, and it is a d'Alembert solution of a wave equation with two wave packages moving to the left and right, the question is which part corresponds the the $ \sin $ or $ \cos $ term?

\scalefig{../Figs/handdraw1108_2}{0.6}{ Two propagating wave packages of the d'Alembert solution.}

Hint: We can use
\begin{align}
\frac{f(x-ct)+f(x+ct)}{2}\rightarrow +\frac{1}{2} \int_{x-ct}^{x+ct} g(z) \ldots dz.
\end{align}

Generally, 
\begin{align}
\spt{P} + \alpha \pt{P} = c^2 \spp{P}{x}.
\end{align}
The FT gives
\begin{align}
\sdt{\hat{P}(k,t)} + \alpha \dt{\hat{P}(k,t)} + c^2 k^2 \hat{P}(k,t) = 0.
\end{align}
This is a damping pendulum equation. 
There are two parts of the solution: one is the left-propagator; and the other is the right-propagator. We need to work it out by ourselves. It may give a shape of light-cone. 
\scalefig{../Figs/handdraw1108_3}{0.6}{ Progagator with different parameters.}

What will happen for this case that
\begin{align}
\braket{x^2} = \int_{-\infty}^{\infty} x^2 P(x,t) dx. 
\end{align}
It will give
\begin{align}
\sdt{\braket{x^2}} + \alpha \dt{\braket{x^2}} = 2c^2. 
\end{align}
This can be solved by making $ v=  \dt{\braket{x^2}}$. The solution can be plotted blow. 
\scalefig{../Figs/handdraw1108_4}{0.6}{ Solution of $ \braket{x^2} $.}
The first part is a coherent wave; the remaining part is diffusion wave. 

We investigate
\begin{align}
\pt{P(x,t)} + c \pp{P(x,t)}{x} = D\spp{P(x,t)}{x}. 
\end{align}
It is called advective-diffusion equation or driven diffusion equation. The wave moves in a speed of $ c $ with a driven force. 



\chapter{Fluid dynamics}\label{chap:fluid}
\section{Continuity equation and constructive relations}

\textbf{Nov 29.}

\scalefig{../Figs/handdraw1129_1}{0.4}{Fluid motion.}
The equation of fluid can be written as
\begin{align}
\fbox{$\pt{\rho} + \bigtriangledown \cdot j =0$}.\label{eq:continuity3d}
\end{align}
This is the continuity equation of fluid. For 1-D case, 
\begin{align}
\pt{\rho} + \px{j} &=0.\label{eq:continuity1d}
\end{align}


Another relationship between $ \rho $ and $ j $ can be given through constitutive relation (we will play with different cases first). 

If we have
\begin{align}
\pm c\rho &=j,
\end{align}
Equ.~\eqref{eq:continuity1d} gives 
\begin{align}
\pt{\rho } \pm \px{c\rho} &=0,\\
\mathrm{or,}\qquad\qquad \qquad  \spt{\rho} &=c^2 \spx{\rho}.
\end{align}
This is the wave equation.

If we have
\begin{align}
-D \px{\rho} &=j.
\end{align}
Hence
\begin{align}
\frac{\partial}{\partial x} \left( -D\px{\rho} \right) + \px{\rho} &=0,\\
\pt{\rho} &= D\spx{\rho}.
\end{align}
This is the diffusion equation. 

If we combine the two cases above and make
\begin{align}
j &=\rho c -D \px{\rho}\\
\Rightarrow \quad \frac{\partial }{\partial x} \left[ \rho c -D\px{\rho } \right] + \pt{\rho} &=0,\\
\Rightarrow \quad \pt{\rho} + c \px{\rho } &= D\spx{\rho}.
\end{align}
Finally, we arrive at the telegrapher's equation.

$ j $ may be like the velocity of a particle, and $ \rho $ may be like the position of a particle. The equation of fluid is like the equation of the movement of a particle. 


What constitutive relation can give the telegrapher's equation?
\begin{align}
\spt{\rho} +\alpha \pt{\rho} &= c^2 \spx{\rho}.
\end{align}
Alternatively, we have the memory equation
\begin{align}
\pt{\rho(x,t)} &= D\int_0^t \drv t' \phi (t-t') \spx{\rho(x,t)}, 
\end{align}
where $ \phi(t)=\alpha e^{\alpha t} $ and $ D\alpha=c^2 $. 

\section{Smoluchowski equation and methods to solve PDEs}
Now, we go back to the third case. In that case, the $ \rho $ is given almost as a constant. What will happen if all variables are not constant? This is like a tetherzed game (I have no idea what it is...). 

Firstly, we change the $ c $ in the advective differential equation, and consider $\text{``velocity''}=-\gamma x$. One suggestion:
\begin{align}
\sdt{x} +\mathrm{const}\dt{x} + c x&=0.
\end{align}
We drop the first and the third terms, 
\begin{align}
\dt{x} &= -\mathrm{const}\, x.
\end{align}

Next, we consider to form a term of $ \px{c\rho} $ in the third case mentioned above. 
Possibly, we can make
\begin{align}
j &= -\gamma x\rho -D\px{\rho},
\end{align}
so that
\begin{align}
\pt{\rho} &= \frac{\partial}{\partial x} \left( \gamma x\rho \right) + D\spx{\rho},\label{eq:Smoluchowski0}\\
\Leftrightarrow\quad \pt{\rho} &= \frac{\partial}{\partial x} \left[ \gamma x\rho  + D\px{\rho}\right].\label{eq:Smoluchowski1}
\end{align}
The term of $\frac{\partial}{\partial x} \left( \gamma x\rho \right)  $ is tethezing. The term of $ D\spx{\rho} $ is homogeneous. The equation above is called Smoluchowski equation. 

If the fluid is in equillibrium or stable state, then $ \pt{\rho}=0 $. So,
\begin{align}
\px{x} + \frac{\gamma x}{D}\rho &=0,\\
\Rightarrow\quad \frac{\drv \rho}{\rho} &= -\frac{\gamma}{D} x \drv x.
\end{align}
It will give a Gaussian solution. 

\scalefig{../Figs/handdraw1129_2}{0.4}{ Flow in stable state.}

Recall that the Fourier transforms give
\begin{align}
\hat{f}(k) &= \intf{f(x)e^{ikx}}{x}\\
f(x) &= \frac{1}{2\pi} \intf{\hat{f}(k)e^{-ikx}}{k}
\end{align}

We can obtian
\begin{align}
\intf{xf(x)e^{ikx}}{x} &= -i \pp{\hat{f}(k)}{k}\\
\frac{-i}{2\pi} \intf{k\hat{f}(k)e^{-ikx}}{k} &= \px{f(x)}.
\end{align}

F.T. of 
\begin{align}
xf(x) \quad &\mathrm{is}\quad -i\pp{\hat{f}(x)}{k}\\
\px{f(x)} \quad &\mathrm{is}\quad -ik\hat{f}(k).
\end{align}

\begin{align}
\pt{\hat{\rho}} &= -ik \gamma (-i)\pp{\hat{\rho}(k)}{k} -Dk^2 \hat{\rho} (k)\\
\pt{\hat{\rho}(k,t)} &+ \gamma k \pp{\hat{\rho}(k,t)}{k} + Dk^2 \hat{\rho} (k,t) = 0.
\end{align}

There are two methods to solve the equation above: one is called brvte force; the other is a clever one. 

The brvte force method is related to the method of characteristics for the first-order PDE's. A good discussion is performed in the first question of Midterm Exam 3. Meanwhile, Ornstern and Vhlenbeck discovered another method of solving the equations above. We will discuss the latter method in the next section (lecture in Dec 4). 


\textbf{Dec 4.}

\begin{align}
\pt{\rho(x,t)} &= \px{} \left[ \gamma x \rho(x,t)\right] + D \spx{\rho(x,t)}\\
j &= \rho (-\gamma x).\label{eq:Smoluchowski2}
\end{align}
We can make $ \gamma = \frac{k}{\Gamma} $, compared to the damped pendulum equation in the form of 
\begin{align}
M \sdt{x} + \Gamma \dt{x  } + \frac{kx}{\Gamma} &=0.
\end{align}

Since the Smoluchowski equation (Equ.~\eqref{eq:Smoluchowski2}) is a second-order PDE, it is hard to solve.  Fortunately, through Fourier Transformation mentioned above, we can transfer it into a first-order PDE so that we can solve it using the method of characteristics. The F.T. yields
\begin{align}
\pt{\hat{\rho}(k,t)} + \gamma k \pp{\hat{\rho}(k,t)}{k} + D k^2 \hat{\rho} (k,t) &=0.
\end{align}


In this section, we will play with the equation in another way. 
Let us make $ U'(x) = \dx{U(x)} $ to replace $ \gamma x $. 

In steady state, we have
\begin{align}
0 &= \px{} \left[ U'(x)\overbrace{\rho(x)}^{\rho(x,\infty)}+ D \px{\rho(x)} \right].
\end{align}
Hence, we can make
\begin{align}
U'(x)\overbrace{\rho(x)}^{\rho(x,\infty)}+ D \px{\rho(x)} &= 0.
\end{align}

\begin{align}
\px{\rho(x)} &= - \frac{U'(x)}{D} \rho(x),\\
\Rightarrow \rho &= \mathrm{const.}\cdot e^{-\frac{U(x)}{D}}
\end{align}
This result is similar to Boltzman distribution function in statistaical mechanics. 

Let us try 
\begin{align}
\hat{\rho} &= \mathrm{const.} e^{-Dk^2 \underbrace{f(t)}_{\text{instead of $ t $, we use $ f(t) $}}}
\end{align}
\begin{align}
\mathrm{const.} e^{-Dk^2 f(t)} \left[ -Dk^2 \dot{f}(t) \right] &+ \mathrm{const.} e^{-Dk^2 f(t)} \gamma k\left[ -2Dkf(t) \right] + \mathrm{const.} e^{-Dk^2 f(t)} \left[ Dk^2 \right] =0\\
\Leftrightarrow -\dot{f}(t) - 2\gamma f(t) +1 &=0\\
\Leftrightarrow \dot{f}(t) + 2\gamma f(t) &=1.
\end{align}
It can be solved and give
\begin{align}
f(t) &= \frac{1- e^{-2\gamma t}}{2\gamma}\quad\qquad \qquad \mathrm{for } \quad f(0)=0.
\end{align}

If $ \rho(x,0) =\delta(x) $, then with 
\begin{align}
\tau &= \frac{1- e^{-2\gamma t}}{2\gamma}
\end{align}
we have 
\begin{align}
\rho (x,t) &= \frac{e^{-\frac{x^2}{4D\tau(t)} }}{\sqrt{4\pi D\tau(t)}} 
= \frac{e^{-\frac{x^2}{4D\frac{1- e^{-2\gamma t}}{2\gamma}} }}{\sqrt{4\pi D\frac{1- e^{-2\gamma t}}{2\gamma}}}
\end{align}

We look at the relationship between $ \tau  $ and $ t $. If $ t\rightarrow 0 $, $ \tau = \frac{1-(1-2\gamma t+(2\gamma)^2t^2)/2}{2\gamma}\approx t $; if $ t $ is large, $ \tau $ approaches and is stabalized at $ \frac{1}{2\gamma} $. 
\scalefig{../Figs/handdraw1204_1}{0.5}{ The evolution of $ \tau(t) $.}

This method of solving equations is called Ornstein-Uhlenbeck method. 

Generally, for $ \rho(x,0)=\delta(x-y) $, we can replace $ x^2 $ by $(x-ye^{-\gamma t})^2$. The memory function gives
\begin{align}
\rho(x,t) &= \intf{G(x,y,t)\rho(y,0)}{y},
\end{align}
where
\begin{align}
G(x,y,t) &= \frac{e^{-\frac{(x-ye^{-\gamma t})^2}{4D\tau(t)}}}{\sqrt{4\pi D \tau (t)}}.
\end{align}


Convective differential equation 
\begin{align}
\pt{\rho} + c \px{\rho} &= D \spx{\rho}
\end{align}
with \begin{align}
j &= c\rho -D \px{\rho}
\end{align}
can be solved in the similar way. We can make a transformation of $ x \rightarrow x- ct $. The behavior of the Gaussian function is determined by the propagator moving in speed $ c $. 


\section{Burger's equation and nonlinear cases}
All the features of the equations above come from the form of $ j $ which may generate diffusion and fluctuations. 

\scalefig{../Figs/handdraw1204_2}{0.6}{Changing $ j $ to give a different equation.}
Now, we consider a nonlinear $ j=j(\rho) = c\rho -b\rho^2$ (see Fig.~\ref{../Figs/handdraw1204_2}). Using $ \px{j} + \pt{\rho}=0 $ , we have
\begin{align}
\pt{\rho} + \pp{\left( c\rho -b\rho^2 \right)}{\rho} \px{\rho} &= 0.
\end{align}

Alternatively, for a homogeneous system
\begin{align}
\pt{\rho} + \left( c-2b\rho \right) \px{\rho} &=0.
\end{align}
We have a nonlinear partial differential equation of
\begin{align}
\pt{\rho} + \left( c-2b\rho \right) \px{\rho} &= D \spx{\rho}.
\end{align}

We multiply $ -2b $ and add $ c $ to the $ \rho $ in the equation above, we have
\begin{align}
\pt{(c-2b \rho)} + (c- 2b\rho ) \px{}\left[ c-2b\rho \right] &= D\spx{\left[c-2b\rho\right]}\\
\Leftrightarrow\qquad  \pt{V} + V \px{V} &= D\spx{V},
\end{align}
where $ V= c- 2b\rho $. This equation is called Burger's equation. 

Compared with the propagator of $ x\rightarrow x-ct $, the $ V $ has the same form of a propagator with a speed $ 2b $ on $ \rho $ axis. 

This equation has the exact solution as follows. The Cole-Hopb transformation
\begin{align}
V(x,t) &= -2D \ln \phi(x,t)
\end{align}
is the solution, where $ \phi(x,t) $ obeys
\begin{align}
\pt{\phi} &= D \spx{\phi}.
\end{align}

We discuss the behavior of the solution now. The initial wave can be seen in Fig.~\ref{../Figs/handdraw1204_3}.
\scalefig{../Figs/handdraw1204_3}{0.4}{ Wave of $ V(x,0) $. It will move and fall down as time marches.} 

If $ D=0 $, the solution will have singularities and shocks. 

The Burger's equation can be used to describe some behaviors of ions in certain crystals. 

\textbf{Dec 6.}

If we start from 
\begin{align}
j &= c\rho -b\rho^2 -D \px{\rho},
\end{align}
we will get the Burger's equation
\begin{align}
\pt{V} + V \px{V} &= D \spx{V}
\end{align}
with $ V = c- 2b\rho $. The solution is 
\begin{align}
V &= -2D \px{}\left[ \ln \phi(x,t) \right],
\end{align}
where $ \phi(x,t) $ comes from the Cole-Hopf transformation:
\begin{align}
\pt{\phi} &= D \spx{\phi}.
\end{align}

If we take an initial wave like the one in Figs.~\ref{../Figs/handdraw1204_3}. This gives the brith of shocks and so on. 

Using the initial value of $ V(x,0) $, we integrate it to give two parts which can be written as 
\begin{align}
\int V(x,0)\drv x &= \ln \phi(x,0) + \mathrm{const.} V_0. 
\end{align}
\scalefig{../Figs/handdraw1206_1}{.4}{ Initial wave slope of $ V(x,0) $.}

Therefore, we can obtain the initial condition for $ \phi(x,t) $, and solve $ \phi(x,t) $ using the Cole-Hopf transformation equation. Next, we can obtain the solution for $ V(x,t) $. 

Notice that $ D=0 $ and $ D\neq 0 $ can give very different results. 

Nonlinear Schrodinger equation, first introduced in optics, can be solved in the similar way. It can be discussed using eigenfunction techniques, and can be transformed into linear equations used in classical mechanics. Characteristic method is useful in solving similar equations. 



We look at the equation of 
\begin{align}
\pt{\rho} &= \underbrace{D \spx{\rho}}_{diffusion} + \underbrace{a\rho -b\rho^2}_{lvqister}
\end{align}
which is called Fisher-*-*-* equation. It can produce or can be transformed into many other forms of equations. So far, it cannot be solved analytically. 

We can solve the steady state equation in terms of elliptic functions based on the pendulium models we discussed in earlier chapters. 

Now, 
\begin{align}
D \spx{\rho} + a\rho -b\rho^2 &=0.
\end{align}
Suppose $ a=b=0 $, it will give a second-order function of $ \rho $. What if $ a=0 $ but $ b\neq 0 $? It will grow, until it reaches $ \frac{a}{b} $. What if $ a\neq 0 $ but $ b=0 $? It will grow and never reaches a stop value. What if $ a\neq 0 $ and $ b\neq 0 $? It will grow and move like a sunami. See Fig.~\ref{../Figs/handdraw1206_2}. 
\scalefig{../Figs/handdraw1206_2}{.4}{ Evolution of Fisher's equation.}

Murray's book tells more details on this Fisher's equation. 



\chapter{Lagrangian and Hamiltonian mechanics}\label{chap:lagham}
\section{Some general theories}
\textbf{This part is lectured by Tzu-Cheng Wu. A part of Eular-Lagrange equation derivation is from Prof. Dunlap's lecture.}

\subsection{The principle of extremum action}\label{sec:action}
For a particle moving in  a Potential $ U $, the equation of motion can be given by
\begin{align}
\sdt{x}=-\pt{U},
\end{align}
\begin{align}
x(0)=x_0,\quad \dot{x}(0)=v_0.
\end{align}
\begin{align}
x(t+dt)\approx x(t)+\dot{x(t)}dt.
\end{align}

\scalefig{../Figs/handdraw1018_1}{0.5}{$ x(t+dt) $.}

Action is defined as
\begin{align}
S(x)= \int_{t_1}^{t_2} L(x,\dot{x},t)dt,
\end{align}
which is a functional. 
\begin{align}
\delta{S}&=0\\
&= \delta\int L(x,\dot{x},t)dt\\
&= \int \delta L(x,\dot{x},t)dt\\
&= \int [\px{L}\delta x + \pp{L}{\dot{x}}\delta \dot{x}+ \pp{t}{t}\delta t ]dt
\end{align}

\begin{align}
\delta S &= \int [\px{L}\delta x + \pp{L}{\dot{x}}\frac{d}{dt}\delta x ]dt\\
&= \int  [\px{L}\delta x + \frac{d}{dt}\pp{L}{\dot{x}}\delta x ]dt\\
&=  \int  [\px{L} + \frac{d}{dt}\pp{L}{\dot{x}}]\delta x dt\\
&=0
\end{align}

\begin{align}
\int \pp{L}{\dot{x}} \frac{d}{dt} \delta x dt &= \cancelto{0}{\left.\delta x \pp{L}{\dot{x}}\right|^{t_2}_{t_1}} - \int \frac{d}{dt} \pp{L}{\dot{x}} \delta x dt
\end{align}
This leads to the Eular-Lagrange equation:
\begin{align}
\pp{L}{x} + \pp{\dot{x}}{t}\pp{L}{\dot{x}}=0.
\end{align}
Alternatively,
\begin{align}
 & \boxed{\frac{\partial L}{\partial x}-\frac{d}{dt}\frac{\partial L}{\partial x}=0}.
\end{align}


\scalefig{../Figs/handdraw1127_1}{0.4}{Action and least action principle. }
Here is another way to derive the Eular-Lagrange equation. We let 
\begin{align}
S &=\int_{t_1}^{t_2} \drv t L(q,\dot{q},t),\\
S^i &= \int_{t_1}^{t_2} \drv t L(q+\eta(t),\dot{q}+\dot{\eta}(t),t),
\end{align}
where $ L(q+\eta(t),\dot{q}+\dot{\eta}(t),t)=L(q,\dot{q},t)+\pp{L}{q}\eta+\pp{L}{\dot{q}}\dot{\eta} $ is the Lagrangian of the $ i $-th path (see Fig.~\ref{../Figs/handdraw1127_1}). We make 
\begin{align}
\Delta S &=S^i-S=\delta S\\
&= \int_{t_1}^{t_2} \drv t \left( \pp{L}{q}\eta+\pp{L}{\dot{q}}\dot{\eta} \right)\\
&= 0. 
\end{align}
Using that \begin{align}
\dt{} \left( \pp{L}{q}\eta \right) &= \dt{} \left( \pp{L}{q} \right) \eta + \pp{L}{q}\dot{\eta},
\end{align}
we have
\begin{align}
\delta S &= \int_{t_1}^{t_2} \drv t \left( \pp{L}{q}\eta + \dt{}\left( \pp{L}{\dot{q} \eta} \right)-\eta \dt{}\left(\pp{L}{\dot{q}} \right) \right)\\
\implies \qquad 0 &= \cancelto{0}{\left. \pp{L}{\dot{q}} \eta\right|_{t_1}^{t_2}  } + \int_{t_1}^{t_2} \drv t \eta \left(\pp{L}{q}-\dt{}\pp{L}{\dot{q}} \right).\label{eq:inteular}
\end{align}
There is a term goes to zero. This is because we always choose $ \eta(t_1)=\eta(t_2)=0 $. 

Immediately, Equ.~\eqref{eq:inteular} leads to the Eular-Lagrange equation
\begin{align}
\pp{L}{q} - \dt{} \left(\pp{L}{\dot{q}} \right)=0.
\end{align}

Now, we define the force as
\begin{align}
\vec{F}
=- \nabla U(x),
\end{align}
where $ U(x) $ is the potential energy. 

The kinetic energy and Lagrangian are hence
\begin{align}
T =T(\dot{x}) &= \frac{1}{2} m\dot{x}^2,\\
L \equiv T-U &=T(\dot{x})-U(x).
\end{align}
We have
\begin{align}
\frac{\partial L}{\partial x} &= - \frac{\partial U}{\partial x},\\
\frac{d}{dt} \frac{\partial L}{\partial \dot{x}} &= \dt{}  T= \frac{d}{dt} (m\dot{x}) = m\ddot{x},\\
\frac{\partial L}{\partial \dot{x}} &= P.
\end{align}

\begin{equation}
\Rightarrow \fbox{$m\ddot{x}=-\frac{\partial U}{\partial x}$}.
\end{equation}

\scalefig{../Figs/handdraw1018_2_1}{0.5}{Free paths and functionals.}

Now the derivative of work gives
\begin{align}
\delta W= F\delta x= F\pp{x}{q}\delta q \equiv Q\delta q.
\end{align}
We have that
\begin{align}
\dt{} \pp{T}{q}- \pp{T}{q}=Q.
\end{align}
where 
\begin{align}
Q &= Q_{con}+Q_{nc}\\
&= \pp{U(q)}{q}+ Q_{nc}.
\end{align}
For general nonconservative
force field, we have separated the generalized force
$Q$ into the conservative and nonconservative parts.  For a conservative force field, In contrast, $L=T-U$ leads to the familiar equation
of motion $m\ddot{x}=-\frac{\partial U}{\partial x}$. 

For nonconservative
force field, the Euler-Lagrange equation can be modified to take into
account the nonconservative force.  
Then
\begin{align}
\frac{d}{dt}\frac{\partial T}{\partial\dot{q}}-\frac{\partial T}{\partial q} & =Q\\
\implies\frac{d}{dt}\frac{\partial L}{\partial\dot{q}}-\frac{\partial L}{\partial q} & =Q_{\mbox{nc}}.
\end{align}

In an arbitrary coordinate, \textit{e.g.} cyclic coordinate, the Eular-Lagrange equation gives
\begin{align}
\pp{L}{q}=\dt {} \pp{L}{\dot{q}}.
\end{align}
If
\begin{align}
L &=L(\dot{x},t)\\
0 &=\dt{} \pp{L}{\dot{x}},
\end{align}
we have
\begin{align}
\pp{L}{\dot{x}}=const.=p.
\end{align}

\subsection{$ H $ and the energy of a system}
One question is whether the Hamiltonian $ H $ equals to total energy $ E $? We assume that
\begin{align}
L=L(q,\dot{q},t)=L(q,\dot{q})\rightarrow \textrm{Not explicitly change in time.}
\end{align}
\begin{align}
\dt{} L &= \pp{L}{q}\dot{q}+\pp{L}{\dot{q}}\ddot{q}+\cancel{\pt{L}}\\
&= \dt{} \pp{L}{\dot{q}} \dot
{q}+ \pp{L}{\dot{q}}\ddot{q}\\
&= \dt{} \left(\dot{q}\pp{L}{\dot{q}} \right)\quad \textrm{To be checked.}
\end{align}
Immediately, we have
\begin{align}
 \dt{} \left(\dot{q}\pp{L}{\dot{q}}-L \right)&=0.\\
\Rightarrow \qquad \qquad \qquad \qquad  \pp{H}{t} &= 0,
\end{align}
where we have defined the Hamiltonian 
\begin{align}
H\equiv \dot{q}\pp{L}{\dot{q}}-L.
\end{align}
It shows that $ H $ is a constant of motion, and usually we think of $ H $ as the energy of the system. The question is then whether $ H $ is actually the energy of the system? The answer is ``not always".

To demonstrate this point, we look at an N-dimensional system. For a N-D case, we can define
\begin{align}
H\equiv \sum_i \dot{q_i}\pp{L}{\dot{q}_i}-L. 
\end{align}
The kinetic energy term gives
\begin{align}
T=T(\dot{q})=\sum_i \frac{1}{2}m \dot{x}_i^2= \sum_{j,k} A_{jk}\dot{q}_j \dot{q}_k,
\end{align}
where $ x=x(q,\dot{q},t)=x(q) $ and $ A_{jk}= \frac{1}{2}m \pp{x_i}{q_j}\pp{x_i}{q_k} $. We can replace $ \dot{q}\rightarrow p $. 

Now that 
\begin{align}
L &= T(\dot{q})-U(q),\\
H&= \sum_i \dot{q}_i \pp{T}{\dot{q}_i}-T+U.
\end{align}
Therefore, 
\begin{align}
\pp{T}{\dot{q}_i}= \sum_k A_{ik} \dot{q}_k+\sum_j A_{ij} \dot{q}_j=2\sum_k A_{ik}\dot{q}_k,
\end{align}
\begin{align}
\Rightarrow \qquad \sum_i \dot{q}_i \pp{T}{\dot{q}_i}= \sum_{ik} A_{ik}\dot{q}_i\dot{q}_k= 2T. 
\end{align}
\begin{align}
\Rightarrow \qquad H=2T-T+U= T+U=E.
\end{align}
We see that, in this example, the Hamiltonian is the energy of the system only because
\begin{itemize}
\item[(a) ] $ \pt{L}=0 $;
\item[(b) ] $ U=U(q) $;
\item[(c) ] $ T=T(\dot{q}) $ is quadratic ($q$ can be in matrix form).
\end{itemize}

We make $ H=H(q,\dot{q},t) =H(q,p,t)$.
\begin{align}
dH &= \pp{H}{q}dq+\pp{H}{p} dp+ \pt{H}dt \label{eq:dH1}\\
&= d(\dot{q}p-L)\\
&= pd\dot{q}+ \dot{q}dp - \left( \pp{L}{q}dq + \pp{L}{\dot{q}}d\dot{q}+ \pt{L}dt \right)\\
&= -\dot{p}dq + \dot{q}dp - \pt{L}dt,\label{eq:dH3}
\end{align}
where we have defined $ \fbox{$\pp{L}{\dot{q}}=p$} $.

Comparing each corresponding terms in Equ.~\ref{eq:dH1} and~\ref{eq:dH3},  we have
\begin{equation}
\left\{ \begin{array}{l}
\left. \begin{array}{c}
\dot{p}= -\pp{H}{q}\\
\dot{q}=\pp{H}{p}
\end{array}\right\}\rightarrow \textrm{$p$ and $q$ are conjugate.} \\
-\pt{L}= \pt{H}\rightarrow \textrm{if }\pt{L}=0=\pt{H}\rightarrow \textrm{conservation of $H$.}
\end{array} \right.
\end{equation}

Now, we define the Poisson bracket as
\begin{align}
\{ A,B\}= \pp{A}{q}\pp{B}{p}-\pp{A}{p}\pp{B}{q}.
\end{align}
The term $ \{q,p \} $ describes kinetic transformation in classical mechanics. For the poison bracket $ \{q',p' \} $, we have
\begin{equation}
\left\{ 
\begin{array}{c}
\dot{p}'= -\pp{H'}{q'}\\
\dot{q}'= \pp{H'}{p'}.
\end{array}\right.
\end{equation}
For a given function $ G=G(q,p,t) $, we have
\begin{align}
\dt {} G(q,p,t)&= \pp{G}{q}\dot{q} + \pp{G}{p} \dot{p}+ \pt{G}\\
&= \pp{G}{q}\pp{H}{p}-\pp{G}{p}\pp{H}{q}+\pt{G} \\
&=\{G,H \}+ \pt{G}.
\end{align}
Therefore, the dynamic equation for $ G $ is 
\begin{equation}
\fbox{$\dt {} G(q,p,t)=\{G,H \}+ \pt{G}$.}
\end{equation}
We know that in quantum mechanics, 
\begin{align}
\dt{} \braket{G}=\frac{1}{i\hbar}\braket{[G,H]}+ \braket{\pt{G}}.
\end{align}
The two equations above shows that there is a connection between classical and quantum mechanics. 


\textbf{Example. }Consider a planar system with a bead confined to
move only along a hoop of radius $r$ whose center is attached to
the end of a stick of length $R$. The stick is rotated with a constant
angular velocity $\omega$ as shown in the picture. There is no gravity.

\scalefig{../Figs/handdraw1018_2_2}{0.8}{An example of using Lagrangian mechanics to solve practical problems.}

In the plane of motion, the coordinate can be given by
\begin{align}
(x,y) & =\left(R\cos(\omega t)+r\cos\left(\omega t+\theta\right),R\sin(\omega t)+r\sin\left(\omega t+\theta\right)\right)\\
\implies\quad  T & =\frac{1}{2}m\left[R^{2}\omega^{2}+r^{2}\left(\omega+\theta\right)^{2}+2R\omega\left(\omega+\dot{\theta}\right)\cos\theta\right]\\
\implies \quad \ddot{\theta}&+\frac{\omega^{2}R}{r}\sin\theta  =0,
\end{align}
which is the equation of motion of the physical pendulum even if there
is no gravity here. The reason is that an observer sitting at the
end of the stick would feel a fictitious centrifugal force outward
having the magnitude of $m\omega^{2}R\sin\theta$.


Another important observation is that $H$ is not the energy since
apparently $T$ is not quadratic. $\frac{\partial H}{\partial t}=0$,
but the energy $\frac{1}{2}I\omega^{2}$ is not conserved since to
keep the angular momentum $L$ constant while $\omega$ is also constant,
the moment of inertia $I$ has to vary in time.


%\missingfigure{Oct18-2-2. An example of using Lagrangian mechanics to solve practical problems.}


%\clearpage 

\textbf{Sections below in this chapter are taught by Prof. Dunlap. }


%\section{Legendre Transformation and Hamiltonian mechanics} 

\textbf{Nov 13.}





%\textbf{Following are 3 lectures in handwriting, taught by Dr. Dunlap.}
%\newpage
%\includepdf[pages={-}]{CM_1_Nov13.pdf}
%
%\includepdf[pages={-}]{CM_2_Nov15.pdf}
%
%\includepdf[pages={-}]{CM_3_Nov27.pdf}

%\newpage



%\section{The principle of extremum action}
%
%Newtonian formalism describes a mechanical system as a differential
%equation and thus is local in time. In Lagrangian formalism, we assume
%that we are given two times $t_{1}$ and $t_{2}$ with a particle
%being at the position $x_{1}$ and $x_{2}$ respectively, the path
%of the system on the configuration space is the one that extremizes
%the action
%\begin{align}
%S & =\int_{t_{1}}^{t_{2}}L\left(x,\dot{x},t\right)dt.\\
%0=\delta S & =\int_{t_{1}}^{t_{2}}\delta L\left(x,\dot{x},t\right)dt\\
% & =\int\left(\frac{\partial L}{\partial x}\delta x+\frac{\partial L}{\partial\dot{x}}\underbrace{\delta\dot{x}}_{d\left(\delta x\right)/dt}+\frac{\partial L}{\partial t}\underbrace{\delta t}_{0}\right)dt\\
% & =\int\left(\frac{\partial L}{\partial x}\delta x+\frac{\partial L}{\partial\dot{x}}\frac{d}{dt}\delta x\right)dt
%\end{align}
%Using integration by path, the second term on the RHS can be expressed
%as
%\begin{align}
%\int dt\frac{\partial L}{\partial\dot{x}}\frac{d}{dt}\delta x= & \underbrace{\delta x\frac{\partial L}{\partial\dot{x}}\Biggr|_{t_{1}}^{t_{2}}}_{0}-\int\frac{d}{dt}\frac{\partial L}{\partial x}\delta xdt
%\end{align}
%where the first term vanishes because the end points are fixed. Therefore,
%\begin{align}
%0 & =\int\left(\frac{\partial L}{\partial x}-\frac{d}{dt}\frac{\partial L}{\partial x}\right)\delta xdt.
%\end{align}
%It can be shown that this implies the Euler-Lagrange equation
%\begin{align}
% & \boxed{\frac{\partial L}{\partial x}-\frac{d}{dt}\frac{\partial L}{\partial x}=0}.
%\end{align}
%For a conservative force field, $L=T-U$ leads to the familiar equation
%of motion $m\ddot{x}=-\frac{\partial U}{\partial x}$. For nonconservative
%force field, the Euler-Lagrange equation can be modified to take into
%account the nonconservative force by separating the generalized force
%$Q$ into the conservative and nonconservative parts
%\begin{align}
%Q & =-\frac{\partial U(q)}{\partial q}+Q_{\mbox{nc}}.
%\end{align}
%Then
%\begin{align}
%\frac{d}{dt}\frac{\partial T}{\partial\dot{q}}-\frac{\partial T}{\partial q} & =Q\\
%\implies\frac{d}{dt}\frac{\partial L}{\partial\dot{q}}-\frac{\partial L}{\partial q} & =Q_{\mbox{nc}}.
%\end{align}



\section{Legendre transform}

A puzzle: given a function $f(x)$, let $\frac{df}{dx}\Bigr|_{x_{0}}=\phi$.
Can we find a function $G(\phi)$ such that $\frac{dG}{d\phi}=x$?

If $f(x)=ax^{n}$, then $G(\phi)=F\left(x(\phi)\right)$ works since
$\frac{dG}{d\phi}=\frac{df}{dx}\frac{dx}{d\phi}=\phi\frac{dx}{d\phi}=\phi\frac{d}{d\phi}\left(\frac{\phi}{an}\right)^{\frac{1}{n-1}}\propto x$.
Note that we have to invert $\phi(x)$ to find $x(\phi)$. The general
recipe for any function of $x$ is the \emph{Legendre transform}
\begin{align}
 & \boxed{G(\phi)=-\phi x(\phi)-F\left(x(\phi)\right)}
\end{align}
since
\begin{align}
\frac{dG}{d\phi} & =x(\phi)+\phi\frac{dx}{d\phi}-\phi\frac{dx}{d\phi}=x.
\end{align}
Geometrically, $-G(\phi)$ is the intercept of the tangent of $f$ at a point $x_{0}$ with the $x=0$ axis (see Fig.~\ref{../Figs/handdraw1113_1}).

At $x_{0}$,
\begin{align}
y-F\left(x_{0}\right) & =\frac{dF}{dx}\Bigr|_{x_{0}}(\overbrace{x}^{0}-x_{0})\\
y & =F\left(x_{0}\right)-\phi x.
\end{align}

\scalefig{../Figs/handdraw1113_1}{0.5}{ Functions of variables ($ x,\phi $).}

\textbf{Example:}
\begin{align}
\quad F(x)= \frac{ax^2}{2},\quad \dd{F}{x}|_{x_0} = ax|_{x_0} \equiv \phi.
\end{align}
Main problem:\begin{align}
F(x) \Rightarrow \dd{F}{x} = \phi.
\end{align}
Can we find $G(\phi )=x$?
Ex: \begin{align}
F(x) &= \frac{ax^2}{2}\\
\dd{F}{x} & = \phi =ax\quad \leftarrow \mathrm{joint}\, x=\frac{1}{a}\phi.
\end{align}
What is $ G(\phi) $? 
\begin{align}
G(\phi) &= F(x(\phi)),\\
F(x(\phi)) &= \frac{a}{2}\left( \frac{\phi}{a} \right)^2 = \frac{\phi^2}{2a}=G(\phi).
\end{align}

\textbf{Example: } $f(x,y)=\frac{ax^{2}}{2}+\frac{by^{2}}{2}+cxy$. We define $ \left( \px{F} \right)_y \equiv \phi $, find $ G(\phi,y) $ so that $ \left( \pp{G}{\phi} \right)_y =x $.
Then the process of finding $G(\phi,y)$ is the same given that $y$
is hold constant at all time.

We can define
\begin{align}
\left( \px{F} \right)_y &= \phi = ax+cy\\
\Leftrightarrow\qquad x &= \frac{\phi-cy}{a}.
\end{align}
Hence, 
\begin{align}
G(\phi) &= \phi \frac{\phi-cy}{a} -\frac{a}{2} \frac{(\phi-cy)^2}{a^2} - \frac{by^2}{2} -cy\frac{\phi-cy}{a}\\
&= \frac{1}{2} \frac{(\phi-cy)^2}{a} -\frac{by^2}{2}.
\end{align}
As expected, this implies to
\begin{align}
\left( \pp{G}{\phi} \right)_y &= \frac{\phi-cy}{a}=x. 
\end{align}

\section{Hamiltonian}

The main idea of this section is this: gievn a Lagrangian $L$, the momentum $p$ canonically conjugated
to a coordinate $q$ is defined by
\begin{align}
p\left(q,\dot{q},t\right) & \equiv\frac{\partial L\left(q,\dot{q},t\right)}{\partial\dot{q}}.
\end{align}
Substitute this into the Euler-Lagrange equation gives
\begin{align}
\dot{p}\left(q,\dot{q},t\right) & =\frac{\partial L\left(q,\dot{q},t\right)}{\partial q}.
\end{align}
Inverting the functions of $\dot{q}$ to functions of $p$ gives the
Hamilton's equations of motion.

For details, we can describe the motion of a moving particle using a set of first-order equations
\begin{align}
\left\{ \begin{array}{l}
\dot{q} =v\\
\dot{v} = f(q,v,t).
\end{array}\right.
\end{align}
The second equation above can be written as 
\begin{align}
m\ddot{q} &= \mathrm{force}\qquad \mathrm{or}\qquad \ddot{q}= f(q,\dot{q},t).
\end{align}
\scalefig{../Figs/handdraw1113_5}{0.4}{Definition of $ q $ and the phase diagram of $ v(q) $.}

Once we know the initial value of $ q(0) $ and $ v(0) $, we will know the $ q(t) $ and $ v(t) $ for all future. The movement of the particle can be described by a phase diagram of $ v(q) $ (see Fig.~\ref{../Figs/handdraw1113_5}). The evolution of the system can be fully described by a function of $ (q,v,t) $, which leads us to introduce the concept of Lagrange, $ L=L(q,\dot{q},t) $. 

Now, the momentum can be defined by
\begin{align}
p &\equiv \pp{L}{\dot{q}}\\
\rightarrow \quad p &= f(q,\dot{q},t).
\end{align}
Hence, 
\begin{align}
\dot{p}= \dt{}\left( \pp{L}{\dot{q}} \right)=\pp{L}{q} = g(q,\dot{q},t).
\end{align}
By inverting the two equations above, we can obtain the Hamilton's equation of motion to be
\begin{align}
\dot{q} &= h(q,p,t)\\
\dot{p} &= r(q,p,t).
\end{align}





\textbf{Example. }A harmonic oscillator has a Lagrangion of  $L=\frac{1}{2}m\dot{q}^{2}-\frac{1}{2}kq^{2}$. What is the Hamilton's equations of motion?

\begin{align}
\left\{ \begin{array}{ll}
p=\frac{\partial L}{\partial\dot{q}}=m\dot{q} & \iff\dot{q}=\frac{p}{m},\\
\dot{p}=\frac{\partial L}{\partial q} & \iff\dot{p}=-kq.
\end{array}\right.
\end{align}

There is another way to obtain the Hamilton's equations of motion without inversion (using Legendre Transformation).
We write
\begin{align}
\drv L(q,\dot{q},t) &= \left(\pp{L}{q}\right)\drv q + \left(\pp{L}{\dot{q}} \right)\dot{q} + \left(\pt{L} \right)\drv t,
\end{align}
and try 
\begin{align}
\boxed{H(q,p,t) = p\dot{q}-L}.
\end{align}
We have
\begin{align}
\drv H &= \drv p \cdot \dot{q} + p\drv \dot{q} -\left( p\drv \dot{q}-\dot{p}\drv q + \pt{L} \drv t \right)\\
&= \dot{q} \drv p -\dot{p} \drv q -\pt{L} \drv t.
\end{align}
Compared with the fact that
\begin{align}
\drv H(q,p,t) &= \pp{H}{p}\drv p + \pp{H}{q}\drv q + \pt{H}\drv t,
\end{align}
it implies
\begin{align}
\left\{ \begin{array}{ll}
\dot{q} &=\pp{H}{p}\\
\dot{p} &= -\pp{H}{q}.
\end{array}\right.
\end{align}
This is the Hamilton's equation of motion as well.

To sum up, we define $H$ to be the Legendre transform
of $L$: $H(p,q,t)=p\dot{q}-L$. Then the Hamilton's equation of motion
are
\begin{align}
 & \boxed{\dot{q}=\frac{\partial H}{\partial p},\ \dot{p}=-\frac{\partial H}{\partial q}}
\end{align}

\textbf{Example:}
For a harmonic oscillator $ L=\frac{1}{2} m\dot{q}^2-\frac{1}{2} kq^2 $, find $ H(q,p,t) $, the Hamilton's equations of motion and the time evolution of the system. 

Answer: 
\begin{align}
p &= \pp{L}{\dot{q}}=m\dot{q} \quad \Rightarrow \dot{q} =\frac{p}{m}\\
L(q,p,t) &= \frac{1}{2} m \left( \frac{p}{m} \right)^2 -\frac{1}{2} kq^2.
\end{align}
Hence, 
\begin{align}
H(q,p,t) &= p\dot{q}-L \\
&= p\frac{p}{m} - \left[\frac{1}{2} m \left( \frac{p}{m} \right)^2 -\frac{1}{2} kq^2 \right]\\
&= \frac{p^2}{2m} + \frac{1}{2} kq^2.
\end{align}

The Hamilton's equations of motion can be written as
\begin{align}
\dot{p} &= -\pp{H}{q}=-kq\\
\dot{q} &= \pp{H}{p} = \frac{p}{m}.
\end{align}
The two equations implies that
\begin{align}
\ddot{p} &= -k\dot{q} = -\frac{kp}{m}\\
\Rightarrow \qquad \ddot{p} &+ \frac{k}{m} p=0\\
\Rightarrow \qquad p(t) &= p(0)\cos \omega t + \frac{\dot{p}(0)}{\omega} \sin \omega t.
\end{align}
Similarly, 
\begin{align}
q(t) &= q(0) \cos \omega t + \frac{\dot{q}(0)}{\omega}\sin \omega t,
\end{align}
where $ \omega = \sqrt{\frac{k}{m}} $, $ \dot{p}(0) =-kq(0)$ and $ \dot{q}(0)= \frac{p(0)}{m} $. 

Therefore, 
\begin{align}
 p(t) & =p(0)\cos(\omega t)-\frac{kq(0)}{\omega}\sin(\omega t),\\
q(t) & =q(0)\cos(\omega t)+\frac{p(0)}{m\omega}\sin(\omega t).
\end{align}
The phase diagram is plotted in Fig.~\ref{../Figs/handdraw1113_6}.

\scalefig{../Figs/handdraw1113_6}{0.4}{Phase diagram of a harmonic oscillator.}

Moreover, you can find that
\begin{align}
\frac{1}{2} k q^2(t) + \frac{1}{2} \frac{p^2(t)}{m} &= \frac{1}{2} kq^2(0) + \frac{p^2(0)}{2m} \equiv E\\
\frac{1}{2} \sqrt{\frac{k}{m}} \left( \underbrace{\sqrt{km} q^2}_{Q^2} + \underbrace{\frac{p^2}{m}}_{P^2} \right) &=E.
\end{align}

By defining $Q=q(km)^{\frac{1}{4}}$ and $P=\frac{p}{(km)^{\frac{1}{4}}}$,
the conservation of energy takes on a very simple form
\begin{align}
Q^{2}+P^{2} & =\frac{2E}{\omega},
\end{align}
where the trajectory in phase space is a circle with radius $\sqrt{\frac{2E}{\omega}}$ (see Fig.~\ref{../Figs/handdraw1113_7}).
\scalefig{../Figs/handdraw1113_7}{0.4}{The phase diagram of $ Q(P) $.}
Notice that we have eliminated $q$ first to get the equation of motion
for $p$. The Hamiltonian treats $q$ and $p$ symmetrically.

\textbf{Exercise:}
If we know $ H=\frac{p^2}{m} + \frac{1}{2}kq^2 $, what is $ L $? 

Answer: \begin{align}
\dot{q} &= \pp{H}{p}=\frac{p}{m}\\
\Rightarrow P &= m\dot{q}\\
\Rightarrow L(q,\dot{q},t) &= p\dot{q} -L\\
&= m\dot{q}\dot{q} - \left[\frac{(m\dot{q})^2}{2m} + \frac{1}{2}kq^2 \right]\\
&= \frac{1}{2} m\dot{q}^2 - \frac{1}{2}kq^2.
\end{align}

For higher dimensions:
\begin{align}
L &= L(q_1,q_2,\ldots,q_N;\dot{q}_1,\dot{q}_2,\ldots,\dot{q}_N;t)\\
p_i &= \pp{L}{\dot{q}_i}, \quad  H=\sum_{i=1}^{N}p_i\dot{q}_i-L.
\end{align}


\section{Constants of motion}

Suppose that $\frac{df(q,p,t)}{dt}$ is a constant, then
\begin{align}
0=\frac{df(q,p,t)}{dt} & =\frac{\partial f}{\partial q}\dot{q}+\frac{\partial f}{\partial p}\dot{p}+\frac{\partial f}{\partial t}\\
 & =\underbrace{\left(\frac{\partial f}{\partial q}\frac{\partial H}{\partial p}-\frac{\partial f}{\partial p}\frac{\partial H}{\partial q}\right)}_{\mathrm{Poisson \,bracket:\,}\left\{ f,H\right\} }+\frac{\partial f}{\partial t}
\end{align}
Useful identities for Poisson brackets:
\begin{align}
\left\{ f,gh\right\}  & =g\left\{ f,h\right\} +\left\{ f,g\right\} h.\\
\left\{ \left\{ f,g\right\} ,h\right\}  & =\left\{ \left\{ h,f\right\} ,g\right\} =\left\{ \left\{ g,h\right\} ,f\right\} .
\end{align}

The trick to find a constant of motion is to solve $ q_i(0) $ and $ p_i(0) $ in terms of $ q_i(t) $ and $ p_i(t) $ through integrating the differential equations of motion. As can be seen, $ H=H(q,p) $ itself is a constant of motion, because
\begin{align}
\cancelto{0}{ \{ H,H \} } \qquad +\quad  \cancelto{0}{ \pt{H} } &=0.
\end{align}
Besides itself, any $ H $ should have at least two constants of motion determined by $ q(0) $ and $ p(0) $.

\textbf{Example. }$H=\frac{p^{2}}{2m}-kq$.

Consider an apparently contrived function $f=q-\frac{p}{m}t+\frac{k}{2m}t^{2}$.
It is a constant of motion;
\begin{align}
\frac{df}{dt}=\left\{ f,H\right\} +\frac{\partial f}{\partial t} & =\frac{p}{m}-\left(-\frac{t}{m}\right)\left(-k\right)-\frac{p}{m}+\frac{kt}{m}=0.
\end{align}
Consider another function $g=p-kt$. It is also a constant of motion;
\begin{align}
\frac{dg}{dt}=\left\{ g,H\right\} +\frac{\partial g}{\partial t} & =0\cdot\frac{\partial H}{\partial p}-(-k)-k=0.
\end{align}
In fact, $f$ and $g$ are nothing but initial conditions $q(0)$
and $p(0)$ respectively. These are called \emph{trajectory constants}.
Any constant of motion is a combination of trajectory constans. Thus,
there always exist $2n$ constants of motion whether $H$ depends
explicitly on time or not.

\textbf{Example. }A free particle $H=\frac{p^{2}}{2m}$. $q$ is a
\emph{cyclic coordinate}, i.e. $q$ is absent from the Hamiltonian$\implies$$p=-\frac{\partial H}{\partial q}$
is conserved. 

\textbf{Example. }$H=p_{1}\left(\frac{p_{2}}{m}-\gamma q_{1}\right)$.
There is no $q_{2}$ in the Hamiltonian, so $p_{2}$ is a constant.
\begin{align}
\dot{q}_{1} & =\frac{\partial H}{\partial p_{1}}=\frac{p_{2}(0)}{m}-\gamma q_{1}\\
\implies q_{1}(t) & =q(0)e^{-\gamma t}+\frac{p_{2}(0)}{m}\frac{1-e^{\gamma t}}{\gamma}.
\end{align}


\textbf{Example. $H=p_{1}\left(\frac{p_{2}}{m}-\gamma q_{1}\right)+m\omega^{2}q_{1}q_{2}$}.
\begin{align}
\dot{p}_{2} & =-\frac{\partial H}{\partial q_{2}}=-m\omega^{2}q_{1}\\
\dot{q}_{1} & =\frac{\partial H}{\partial p_{1}}=\frac{p_{2}}{m}-\gamma q_{1}\\
\implies & \boxed{\ddot{q}_{1}+\gamma\dot{q}_{1}+m\omega q_{1}=0}.
\end{align}
We have obtained the equation of motion of a damped harmonic oscillator
even though there is no explicit time dependence in the Hamiltonian!

\scalefig{../Figs/handdraw1115_1}{0.8}{Relative movement of two particles.}
\textbf{Example: }A rotor with two particle (see Fig.~\ref{../Figs/handdraw1115_1}) with masses $m_{1}$ and
$m_{2}$ and the reduced mass $\mu=\frac{m_{1}m_{2}}{m_{1}+m_{2}}$
and the Lagrangian
\begin{align}
L & =\frac{\mu}{2}\left[\dot{r}^{2}+\left(r\dot{\theta}\right)^{2}+\left(r\sin\theta\dot{\phi}\right)^{2}\right]-U(r).
\end{align}
$p_{r}$ is the momentum and $p_{\theta}$ is the angular momentum.
But it is not obvious by looking at $p_{\phi}=\frac{\partial L}{\partial\dot{\phi}}=\mu r^{2}\sin^{2}\theta\dot{\phi}$
whether $p_{\phi}$ is a constant of motion or not. We can write down
the Hamiltonian
\begin{align}
H & =p_{r}\dot{r}+p_{\theta}\dot{\theta}+p_{\phi}\dot{\phi}-L\\
 & =\frac{p_{r}^{2}}{2\mu}+\frac{p_{\theta}^{2}}{2\mu r^{2}}+\frac{p_{\phi}^{2}}{2\mu r^{2}\sin^{2}\theta}+U(r).
\end{align}
Consider
\begin{align}
p_{\theta}\dot{p}_{\theta} & =p_{\phi}^{2}\frac{\cos\theta}{\sin^{2}\theta}\dot{\theta}\\
\frac{1}{2}\frac{d}{dt}\left(p_{\theta}^{2}\right) & =-\frac{1}{2}p_{\phi}^{2}\frac{d}{dt}\frac{1}{\sin^{2}\theta}\\
\implies p_{\theta}^{2}+\frac{p_{\phi}^{2}}{\sin^{2}\theta}\equiv L^{2} & =\mbox{constant}.
\end{align}
Using what we just found out, the Hamiltonian can be written in the
familiar form (as seen in quantum mechanics, for example)
\begin{align}
 & \boxed{H_r=\frac{p_{r}^{2}}{2\mu}+\frac{L^{2}}{2\mu r^{2}}+U(r)}
\end{align}




\section{Liouvillian}

From the Hamilton's equations of motion
\begin{align}
\left\{ 
\begin{array}{rl}
\dot{p} &= -\pp{H}{q} \\
\dot{q} &= -\pp{H}{p} 
\end{array} \right.
\end{align}
we have
\begin{align}
\left\{ 
\begin{array}{rl}
\left\{ p,H \right\} &= \pp{p}{q}\pp{H}{p} - \pp{p}{p}\pp{H}{q}= -\pp{H}{q}=\dot{p} \\
\left\{ q,H\right\} &= \pp{q}{q}\pp{H}{p} - \pp{q}{p}\pp{H}{q}= \pp{H}{q}= \dot{q}.
\end{array} \right.
\end{align}
Therefore, 
\begin{align}
\dot{q} &= \left\{q,H\right\}\\
\dot{p} &= \left\{ p,H\right\},
\end{align}
alternatively, in the form of Liouvillian operator
\begin{align}
\dot{q} &= \mathcal{L} q\\
\dot{p} &= \mathcal{L} p
\end{align}
where
\begin{align}
\mathcal{L} &= \pp{H}{p}\pp{}{q} - \pp{H}{q}\pp{}{p} 
\end{align}
is a differential operator called Liouvillian operator. 


Now we can write the time evolution of a Hamiltonian system in terms
of a linear map that maps the initial $q(0)$ and $p(0)$ to $q(t)$
and $p(t)$ at later times analogous to the time evolution by a unitary
in quantum mechanics.
\begin{align}
q(t) & =q(0)+\int_{0}^{t}dt^{'}\mathcal{L}\left(t^{'}\right)q\left(t^{'}\right)\\
 & =q(0)+\int_{0}^{t}dt^{'}\mathcal{L}\left(t^{'}\right)\left(q(0)+\int_{0}^{t^{'}}dt^{''}\mathcal{L}\left(t^{''}\right)q\left(t^{''}\right)\right)+...\\
 & =\left[1+\underbrace{\int_{0}^{t}dt^{'}\mathcal{L}\left(t^{'}\right)}_{\mathcal{L}}+\underbrace{\int_{0}^{t}dt^{'}\mathcal{L}\left(t^{'}\right)\int_{0}^{t^{'}}dt^{''}\mathcal{L}\left(t^{''}\right)}_{\frac{t^{2}}{2!}\mathcal{L}^{2}}+...\right]q(0)\\
 &= \left. \left[ 1+ \mathcal{L} t + \frac{\mathcal{L}^2}{2!} t^2 + \ldots \right]q\right|_{t=0}\\
 & =e^{\mathcal{L}t}q(0)= \left. e^{\mathcal{L}t}q\right|_{t=0},
\end{align}
and similarly,
\begin{align}
p(t) &= \left. \left[ \underbrace{1}_{\mathcal{L}^0 t^0}+ \mathcal{L} t + \frac{\mathcal{L}^2}{2!} t^2 + \ldots \right]p\right|_{t=0}\\
& =e^{\mathcal{L}t}p(0)= \left. e^{\mathcal{L}t}p\right|_{t=0}.
\end{align}
\textbf{Example. }$H=\frac{p^{2}}{2m}-kq$.
\begin{align}
\mathcal{L} & =\frac{p}{m}\frac{\partial}{\partial q}+k\frac{\partial}{\partial p}.
\end{align}
$\mathcal{L}(p)=k$, $\mathcal{L}^{2}(p)=\mathcal{L}\left(\mathcal{L}(p)\right)=\mathcal{L}k=0$.
So the series terminates and we get that
\begin{align}
p(t) & =e^{\mathcal{L}t}(p)\Bigr|_{0}=p(0)+kt.
\end{align}
$\mathcal{L}(q)=\frac{p}{m}$, $\mathcal{L}^{2}(q)=\frac{k}{m}$,
$\mathcal{L}^{3}(q)=0$. So
\begin{align}
q(t) & =e^{\mathcal{L}t}(q)\Bigr|_{0}=q(0)+\frac{p(0)}{m}t+\frac{k}{2m}t^{2}.
\end{align}
We can go further and write the Liouvillian as a superoperator similar
to the time evolution $A(t)=U^{\dagger}A(0)U$ of an operator $A$
in the Heisenberg picture.


\section{Coordinate transformation}
\scalefig{../Figs/handdraw1115_3}{0.4}{An example of coordinate transformation.}
A mass on a spring is attached to a car moving with velocity $v$ (see Fig.~\ref{../Figs/handdraw1115_3})
such that $H=\frac{p^{2}}{2m}+\frac{1}{2}k(x-vt)^{2}.$ Suppose that
we want to simplify the problem by going into the moving frame and
define $X=x-vt$. What is the conjugate momentum $P$?

Go back to the Lagrangian:
\begin{align}
L & =\frac{1}{2}m\dot{x}^{2}-\frac{1}{2}k(x-vt)^{2}\\
X & =x-vt\\
\dot{X} & =\dot{x}-v.
\end{align}
The new Lagrangian is
\begin{align}
\tilde{L} & =\frac{1}{2}m\left(\dot{X}+v\right)^{2}-\frac{1}{2}kX^{2}.
\end{align}
The new momentum is
\begin{align}
P=\frac{\partial\tilde{L}}{\partial\dot{X}} & =m\left(\dot{X}+v\right).
\end{align}
Then
\begin{align}
\tilde{H}=P\dot{X}-\tilde{L} & =\frac{\left(P-mv\right)^{2}}{2m}+\frac{1}{2}kX^{2}-\underbrace{\frac{1}{2}mv^{2}}_{\mbox{constant}}.
\end{align}
Without going back to the Lagrangian, we might be tempt to simply
write
\begin{align}
P & =m\dot{X}=p-mv\\
\implies\frac{p^{2}}{2m} & =\frac{\left(P+mv\right)^{2}}{2m}.
\end{align}
But that would be off by a sign. 
A correct process of coordinate transformation should be replacing $ x $ and $ \dot{x} $ with $ X\& \dot{X} \Rightarrow \tilde{L}(X,\dot{X},t) \Rightarrow P(\dot{X},t) \rightarrow \dot{X}=\dot{X}(P,t) \rightarrow \tilde{H}(X,P,t)=P\dot{X}-\tilde{L} $. 


\subsection{Least action principle for the Hamiltonian}


We continue from Section~\ref{sec:action}. In searching for a set of transformations that preserve Hamilton's equations of motion, we go back to the action principle. Can the action principle be formulated in terms of the Hamiltonian alone? The answer is yes. Let the action be
\begin{align}
S & =\int_{t_{1}}^{t_{2}}dt\left(p\dot{q}-H(q,p,t)\right).\\
0=\delta S & =\int_{t_{1}}^{t_{2}}dt\left(\delta p\dot{q}+\underbrace{p\delta\dot{q}}_{\frac{d}{dt}(p\delta q)-\dot{p}\delta q}-\frac{\partial H}{\partial q}\delta q-\frac{\partial H}{\partial p}\delta p\right).
\end{align}
Assume that $\delta q$ and $\delta p$ are varied independently.
\begin{align}
0 & =p\underbrace{\delta q\Big|_{t_{1}}^{t_{2}}}_{0}+\int_{t_{1}}^{t_{2}}dt\left(\delta p\left(\dot{q}-\frac{\partial H}{\partial p}\right)-\left(\dot{p}+\frac{\partial H}{\partial q}\right)\delta q\right).
\end{align}
The demand that the action is extremized ($ \delta p $ and $ \delta q $ vanish at extreme points) is equivalent to Hamilton's
equations
\begin{align}
\dot{q}-\frac{\partial H}{\partial p} & =0,\\
\dot{p}+\frac{\partial H}{\partial q} & =0.
\end{align}

Curiously, this does not required the variation in the nomentum at
the end points to vanish, although we will assume that from now on.
Then for any scalar function $F(q,p,t)$, the action can be modified
\begin{align}
S & =\int_{t_{1}}^{t_{2}}dt\left(p\dot{q}-H(q,p,t)+\frac{\drv F(q,p,t)}{\drv t}\right),
\end{align}
where we have added $\dt{} (pq)= \frac{\drv F(q,p,t)}{\drv t}  $ into the original Lagrangian. With this modification, we have actually make 
\begin{align}
H_{ef\! f} &= H(q,p,t)+ \frac{\drv F(q,p,t)}{\drv t}
\end{align}
as the effective Hamiltonian without changing the Hamilton's equation of motion. 



\subsection{Canonical transformations}

It is clear from the previous section that, if we demond the action doesn't change in an infinestmal time period, for any new Hamiltonian
$\mathcal{H}(Q,P,T)$ to give Hamilton's equations, we need
\begin{align}
dt\left(p\dot{q}-H(q,p,t)\right) & =dt\left(P\dot{Q}-\mathcal{H}(Q,P,t)+\frac{dF(q,p,Q,P,t)}{dt}\right)\\
pdq-Hdt-PdQ & =\mathcal{H}dt+dF,
\end{align}
where $F(q,p,Q,P,t)$ is the \emph{generating function }for a canonical
transformation (preserving Hamilton's equations).

The equality above implies to us that we can choose $ F=qQ $ so that
\begin{align}
p\drv q-H \drv t  = PQ-\mathcal{H}dt+\frac{\partial F}{\partial q}dq+\frac{\partial F}{\partial Q}dQ=PQ- \mathcal{H} \drv t + \drv q Q+ q\drv Q.
\end{align}
So, the new set of Hamiltonian and variables are
\begin{align}
\mathcal{H} &= H\\
Q &= p\\
P &=-q.
\end{align}
This kind of transformation is called Cananical transformation. 

Similarly, we define four types of transformations:

Type-1:\textbf{ }$F=F(q,Q)$

Type-2: $F=F(q,P)$

Type-3: $F=F(p,Q)$

Type-4: $F=F(p,P)$

For example, for the type-2 transformation, we can make $ F=qP $, and then we have
\begin{align}
p\drv q-H\drv t &= P \drv Q- \mathcal{H} \drv t-q\drv P -P \drv q\\
\implies 
\begin{cases}
p &= -P\\
Q &= -q.
\end{cases}
\end{align}

Similarly, for type-3 transformation, we make 
\begin{align}
F &=pQ\\
p\drv q -H\drv t &= P\drv Q - \mathcal{H} \drv t-p\drv Q-\drv p Q\\
\implies \begin{cases}
P &= p\\
Q &=Q.
\end{cases}
\end{align}




\textbf{Example: }A type-1 transformation of a harmonic oscillator
$H=\frac{p^{2}}{2m}+\frac{1}{2}m\omega^{2}q^{2}$. It would be nice
if we could transform the Hamiltonian such that it has no $Q$ dependence.
That is, 
\begin{align}
H & =\frac{f^{2}(p)}{2m}.
\end{align}
We therefore consider this transformation
\begin{align}
p & =f(p)\sin Q,\\
q & =\frac{f(p)}{m\omega}\cos Q,
\end{align}
and find the generating function $F$. We have the relation that
\begin{align}
p\drv q -H\drv t&= P\drv Q -\mathcal{H}\drv t + \pp{F}{q} \drv q + \pp{F}{Q} \drv Q.
\end{align}
Hence
\begin{align}
\frac{p}{q} & =m\omega\tan Q\\
p=\left(\frac{\partial F}{\partial q}\right)_{Q} & =m\omega q\tan Q\\
\implies F & =\frac{m\omega q^{2}}{2}\tan Q+g(Q)
\end{align}
Choose $g(Q)=0$. Then
\begin{align}
P=-\left(\frac{\partial F}{\partial Q}\right)_{q} & =-\frac{m\omega q^{2}}{2}\sec^{2}Q\\
 & =-\frac{f^{2}(p)}{2m\omega}\cos^{2}Q\sec^{2}Q\\
\implies f(p) & =i\sqrt{2m\omega P}
\end{align}
\begin{align}
\implies & \boxed{H=-\omega P}.
\end{align}
Test: 
\begin{align}
\begin{cases}
\dot{P} &= -\pp{\mathcal{H}}{Q} = 0\\
\dot{Q} &= \pp{\mathcal{H}}{P} = -\omega.
\end{cases}
\end{align}
Therefore, the Hamiltonian becomes that of the free particle with $P=P(0)=\text{const.}$ and
$Q=Q(0)-\omega t$ which is linear function of $ t $! Say, in the study of fluctuation-dissipation relation,
if you want to phenomenologically put friction in the system, it is
much easier to put friction in at this level after you have transformed
the oscillator into a free particle.

\textbf{Example: }A type-2 transformation of a harmonic oscillator
$H=\frac{p^{2}}{2m}+\frac{1}{2}m\omega^{2}q^{2}$.

Let $P$ be the ``ladder operators''
\begin{align}
P & =\frac{p+im\omega q}{\sqrt{2}},
\end{align}
and maybe
\begin{align}
Q &= \sqrt{2} P-im\omega q. 
\end{align}
which we don't know.
We look for $F(q,P)$.

Now, 
\begin{align}
p\drv q-H\drv t+Q\drv P & =P\drv Q-\mathcal{H}dt+\frac{\partial F}{\partial q}\drv q+\frac{\partial F}{\partial P}\drv P\\
p=\left(\frac{\partial F}{\partial q}\right)_{P} & =\sqrt{2}P-im\omega q\\
\implies \qquad F & =\sqrt{2}Pq-\frac{im\omega q^{2}}{2}+g(P)\\
\implies \qquad Q &=\left(\frac{\partial F}{\partial P}\right)_{q}  =\sqrt{2}q+g^{'}(P).
\end{align}
Choose $g^{'}(P)=\frac{iP}{m\omega}$. Then
\begin{align}
Q & =\sqrt{2}q+\frac{iP}{m\omega}=\sqrt{2}q+\frac{i}{m\omega}\left(\frac{p+im\omega q}{\sqrt{2}}\right)=\sqrt{2}q+\frac{ip}{\sqrt{2}m\omega}-\frac{q}{\sqrt{2}}\\
&=\frac{q+\frac{ip}{m\omega}}{\sqrt{2}}.
\end{align}
We can calculate
\begin{align}
PQ &= \frac{p+im\omega q}{\sqrt{2}}\cdot \frac{q+\frac{ip}{m\omega}}{\sqrt{2}}\\
&= \frac{1}{2}\left(pq + i\frac{p^2}{m\omega}+ im\omega q^2 - pq \right)\\
&= i\frac{p^2}{2m\omega}+ \frac{1}{2}im\omega q^2\\
\implies \qquad \frac{p^2}{2m} &+ \frac{1}{2}m\omega^2 q^2  = -i\omega PQ.
\end{align}
Therefore, the Hamiltonian is
\begin{align}
 & \boxed{\mathcal{H}=-i\omega PQ},
\end{align}
which is the analog of $H=\omega a^{\dagger}a$ in quantum mechanics,
and $P$ and $Q$, like the ladder operators, evolve trivially in
time.
\begin{align}
P(t)=P(0)e^{-i\omega t}, \quad & Q(t)=Q(0)e^{i\omega t}.
\end{align}



%\clearpage
%
%%%% LyX 2.0.3 created this file.  For more info, see http://www.lyx.org/.
%%% Do not edit unless you really know what you are doing.
%\documentclass[11pt,english]{article}
%\usepackage[T1]{fontenc}
%\usepackage[latin9]{inputenc}
%\usepackage{geometry}
%\geometry{verbose,tmargin=1in,bmargin=1in,lmargin=1in,rmargin=1in}
%\usepackage{amsmath}
%\usepackage{graphicx}
%\usepackage{esint}
%
%\makeatletter
%
%%%%%%%%%%%%%%%%%%%%%%%%%%%%%%% LyX specific LaTeX commands.
%%% A simple dot to overcome graphicx limitations
%\newcommand{\lyxdot}{.}
%
%
%\makeatother
%
%\usepackage{babel}
%\begin{document}
%
%\title{Lagrangian and Hamiltonian Mechanics}
%
%
%\date{Lectures by Professor David H. Dunlap and Tzu-Cheng Wu}
%
%
%\author{Ninnat Dangniam}
%
%\maketitle
%\tableofcontents{}


\section{Lagrangian and Hamiltonian formalism}


\subsection{The principle of extremum action}

Newtonian formalism describes a mechanical system as a differential
equation and thus is local in time. In Lagrangian formalism, we assume
that we are given two times $t_{1}$ and $t_{2}$ with a particle
being at the position $x_{1}$ and $x_{2}$ respectively, the path
of the system on the configuration space is the one that extremizes
the action
\begin{align*}
S & =\int_{t_{1}}^{t_{2}}L\left(x,\dot{x},t\right)dt.\\
0=\delta S & =\int_{t_{1}}^{t_{2}}\delta L\left(x,\dot{x},t\right)dt\\
 & =\int\left(\frac{\partial L}{\partial x}\delta x+\frac{\partial L}{\partial\dot{x}}\underbrace{\delta\dot{x}}_{d\left(\delta x\right)/dt}+\frac{\partial L}{\partial t}\underbrace{\delta t}_{0}\right)dt\\
 & =\int\left(\frac{\partial L}{\partial x}\delta x+\frac{\partial L}{\partial\dot{x}}\frac{d}{dt}\delta x\right)dt
\end{align*}
Using integration by path, the second term on the RHS can be expressed
as
\begin{align*}
\int dt\frac{\partial L}{\partial\dot{x}}\frac{d}{dt}\delta x= & \underbrace{\delta x\frac{\partial L}{\partial\dot{x}}\Biggr|_{t_{1}}^{t_{2}}}_{0}-\int\frac{d}{dt}\frac{\partial L}{\partial x}\delta xdt
\end{align*}
where the first term vanishes because the end points are fixed. Therefore,
\begin{align*}
0 & =\int\left(\frac{\partial L}{\partial x}-\frac{d}{dt}\frac{\partial L}{\partial x}\right)\delta xdt.
\end{align*}
It can be shown that this implies the Euler-Lagrange equation
\begin{align*}
 & \boxed{\frac{\partial L}{\partial x}-\frac{d}{dt}\frac{\partial L}{\partial x}=0}.
\end{align*}
For a conservative force field, $L=T-U$ leads to the familiar equation
of motion $m\ddot{x}=-\frac{\partial U}{\partial x}$. For nonconservative
force field, the Euler-Lagrange equation can be modified to take into
account the nonconservative force by separating the generalized force
$Q$ into the conservative and nonconservative parts
\begin{align*}
Q & =-\frac{\partial U(q)}{\partial q}+Q_{\mbox{nc}}.
\end{align*}
Then
\begin{align*}
\frac{d}{dt}\frac{\partial T}{\partial\dot{q}}-\frac{\partial T}{\partial q} & =Q\\
\implies\frac{d}{dt}\frac{\partial L}{\partial\dot{q}}-\frac{\partial L}{\partial q} & =Q_{\mbox{nc}}.
\end{align*}



\subsection{Legendre transform}

A puzzle: given a function $f(x)$, let $\frac{df}{dx}\Bigr|_{x_{0}}=\phi$.
Can we find a function $G(\phi)$ such that $\frac{dG}{d\phi}=x$?

If $f(x)=ax^{n}$, then $G(\phi)=F\left(x(\phi)\right)$ works since
$\frac{dG}{d\phi}=\frac{df}{dx}\frac{dx}{d\phi}=\phi\frac{dx}{d\phi}=\phi\frac{d}{d\phi}\left(\frac{\phi}{an}\right)^{\frac{1}{n-1}}\propto x$.
Note that we have to invert $\phi(x)$ to find $x(\phi)$. The general
recipe for any function of $x$ is the \emph{Legendre transform}
\begin{align*}
 & \boxed{G(\phi)=-\phi x(\phi)-F\left(x(\phi)\right)}
\end{align*}
since
\begin{align*}
\frac{dG}{d\phi} & =x(\phi)+\phi\frac{dx}{d\phi}-\phi\frac{dx}{d\phi}=x.
\end{align*}
Geometrically, $-G(\phi)$ is the intercept of the tangent of $f$
at a point $x_{0}$ with the $x=0$ axis.

At $x_{0}$,
\begin{align*}
y-F\left(x_{0}\right) & =\frac{dF}{dx}\Bigr|_{x_{0}}(\overbrace{x}^{0}-x_{0})\\
y & =F\left(x_{0}\right)-\phi x.
\end{align*}


\textbf{Example. }$f(x,y)=\frac{ax^{2}}{2}+\frac{by^{2}}{2}+cxy$.
Then the process of finding $G(\phi,y)$ is the same given that $y$
is hold constant at all time.


\subsection{Hamiltonian}

Gievn a Lagrangian $L$, the momentum $p$ canonically conjugated
to a coordinate $q$ is defined by
\begin{align*}
p\left(q,\dot{q},t\right) & \equiv\frac{\partial L\left(q,\dot{q},t\right)}{\partial\dot{q}}.
\end{align*}
Substitute this into the Euler-Lagrange equation gives
\begin{align*}
\dot{p}\left(q,\dot{q},t\right) & =\frac{\partial L\left(q,\dot{q},t\right)}{\partial q}.
\end{align*}
Inverting the functions of $\dot{q}$ to functions of $p$ gives the
Hamilton's equations of motion.

\textbf{Example. }A harmonic oscillator $L=\frac{1}{2}m\dot{q}^{2}-\frac{1}{2}kq^{2}$.
\begin{align*}
p=\frac{\partial L}{\partial\dot{q}}=m\dot{q} & \iff\dot{q}=\frac{p}{m},\\
\dot{p}=\frac{\partial L}{\partial q} & \iff\dot{p}=-kq.
\end{align*}
To do this systematically, define $H$ to be the Legendre transform
of $L$: $H(p,q,t)=p\dot{q}-L$. Then the Hamilton's equation of motion
are
\begin{align*}
 & \boxed{\dot{q}=\frac{\partial H}{\partial p},\ \dot{p}=-\frac{\partial H}{\partial q}}
\end{align*}
For the harmonic oscillator,
\begin{align*}
H & =\frac{p^{2}}{2m}+\frac{kq^{2}}{2}\\
\implies\ddot{p}+\omega^{2}p & =0\\
\implies p(t) & =p(0)\cos(\omega t)-\frac{kq(0)}{\omega}\sin(\omega t),\\
q(t) & =q(0)\cos(\omega t)+\frac{p(0)}{m\omega}\sin(\omega t).
\end{align*}
By defining $Q=q(km)^{\frac{1}{4}}$ and $P=\frac{p}{(km)^{\frac{1}{4}}}$,
the conservation of energy takes on a very simple form
\begin{align*}
Q^{2}+P^{2} & =\frac{2E}{\omega},
\end{align*}
where the trajectory in phase space is a circle with radius $\sqrt{\frac{2E}{\omega}}$.
Notice that we have eliminated $q$ first to get the equation of motion
for $p$. The Hamiltonian treats $q$ and $p$ symmetrically.


\subsection{Constants of motion}

Suppose that $\frac{df(q,p,t)}{dt}$ is a constant, then
\begin{align*}
0=\frac{df(q,p,t)}{dt} & =\frac{\partial f}{\partial q}\dot{q}+\frac{\partial f}{\partial p}\dot{p}+\frac{\partial f}{\partial t}\\
 & =\underbrace{\left(\frac{\partial f}{\partial q}\frac{\partial H}{\partial p}-\frac{\partial f}{\partial p}\frac{\partial H}{\partial q}\right)}_{\left\{ f,H\right\} }+\frac{\partial f}{\partial t}
\end{align*}
Useful identities for Poisson brackets:
\begin{align*}
\left\{ f,gh\right\}  & =g\left\{ f,h\right\} +\left\{ f,g\right\} h.\\
\left\{ \left\{ f,g\right\} ,h\right\}  & =\left\{ \left\{ h,f\right\} ,g\right\} =\left\{ \left\{ g,h\right\} ,f\right\} .
\end{align*}


\textbf{Example. }$H=\frac{p^{2}}{2m}-kq$.

Consider an apparently contrived function $f=q-\frac{p}{m}t+\frac{k}{2m}t^{2}$.
It is a constant of motion;
\begin{align*}
\frac{df}{dt}=\left\{ f,H\right\} +\frac{\partial f}{\partial t} & =\frac{p}{m}-\left(-\frac{t}{m}\right)\left(-k\right)-\frac{p}{m}+\frac{kt}{m}=0.
\end{align*}
Consider another function $g=p-kt$. It is also a constant of motion;
\begin{align*}
\frac{dg}{dt}=\left\{ g,H\right\} +\frac{\partial g}{\partial t} & =0\cdot\frac{\partial H}{\partial p}-(-k)-k=0.
\end{align*}
In fact, $f$ and $g$ are nothing but initial conditions $q(0)$
and $p(0)$ respectively. These are called \emph{trajectory constants}.
Any constant of motion is a combination of trajectory constans. Thus,
there always exist $2n$ constants of motion whether $H$ depends
explicitly on time or not.

\textbf{Example. }A free particle $H=\frac{p^{2}}{2m}$. $q$ is a
\emph{cyclic coordinate}, i.e. $q$ is absent from the Hamiltonian$\implies$$p=-\frac{\partial H}{\partial q}$
is conserved. 

\textbf{Example. }$H=p_{1}\left(\frac{p_{2}}{m}-\gamma q_{1}\right)$.
There is no $q_{2}$ in the Hamiltonian, so $p_{2}$ is a constant.
\begin{align*}
\dot{q}_{1} & =\frac{\partial H}{\partial p_{1}}=\frac{p_{2}(0)}{m}-\gamma q_{1}\\
\implies q_{1}(t) & =q(0)e^{-\gamma t}+\frac{p_{2}(0)}{m}\frac{1-e^{\gamma t}}{\gamma}.
\end{align*}


\textbf{Example. $H=p_{1}\left(\frac{p_{2}}{m}-\gamma q_{1}\right)+m\omega^{2}q_{1}q_{2}$}.
\begin{align*}
\dot{p}_{2} & =-\frac{\partial H}{\partial q_{2}}=-m\omega^{2}q_{1}\\
\dot{q}_{1} & =\frac{\partial H}{\partial p_{1}}=\frac{p_{2}}{m}-\gamma q_{1}\\
\implies & \boxed{\ddot{q}_{1}+\gamma\dot{q}_{1}+m\omega q_{1}=0}.
\end{align*}
We have obtained the equation of motion of a damped harmonic oscillator
even though there is no explicit time dependence in the Hamiltonian!

\textbf{Example. }A rotor with two particle with masses $m_{1}$ and
$m_{2}$ and the reduced mass $\mu=\frac{m_{1}m_{2}}{m_{1}+m_{2}}$
and the Lagrangian
\begin{align*}
L & =\frac{\mu}{2}\left[\dot{r}^{2}+\left(r\dot{\theta}\right)^{2}+\left(r\sin\theta\dot{\phi}\right)^{2}\right]-U(r).
\end{align*}
$p_{r}$ is the momentum and $p_{\theta}$ is the angular momentum.
But it is not obvious by looking at $p_{\phi}=\frac{\partial L}{\partial\dot{\phi}}=\mu r^{2}\sin^{2}\theta\dot{\phi}$
whether $p_{\phi}$ is a constant of motion or not. We can write down
the Hamiltonian
\begin{align*}
H & =p_{r}\dot{r}+p_{\theta}\dot{\theta}+p_{\phi}\dot{\phi}-L\\
 & =\frac{p_{r}^{2}}{2\mu}+\frac{p_{\theta}^{2}}{2\mu r^{2}}+\frac{p_{\phi}^{2}}{2\mu r^{2}\sin^{2}\theta}+U(r).
\end{align*}
Consider
\begin{align*}
p_{\theta}\dot{p}_{\theta} & =p_{\phi}^{2}\frac{\cos\theta}{\sin^{2}\theta}\dot{\theta}\\
\frac{1}{2}\frac{d}{dt}\left(p_{\theta}^{2}\right) & =-\frac{1}{2}p_{\phi}^{2}\frac{d}{dt}\frac{1}{\sin^{2}\theta}\\
\implies p_{\theta}^{2}+\frac{p_{\phi}^{2}}{\sin^{2}\theta}\equiv L^{2} & =\mbox{constant}.
\end{align*}
Using what we just found out, the Hamiltonian can be written in the
familiar form (as seen in quantum mechanics, for example)
\begin{align*}
 & \boxed{H=\frac{p_{r}^{2}}{2\mu}+\frac{L^{2}}{2\mu r^{2}}+U(r)}
\end{align*}



\subsection{When the Hamiltonian is not the energy}

Suppose that $L=L\left(q,\dot{q}\right)$. Then
\begin{align*}
\frac{dL}{dt} & =\frac{\partial L}{\partial q}\dot{q}+\frac{\partial L}{\partial\dot{q}}\ddot{q}\\
 & =\frac{d}{dt}\frac{\partial L}{\partial\dot{q}}\dot{q}+\frac{\partial L}{\partial\dot{q}}\ddot{q}\ (\mbox{cancellation of dots})\\
0 & =\frac{d}{dt}\boxed{\underbrace{\left(\dot{q}\frac{\partial L}{\partial\dot{q}}-L\right)}_{H}},
\end{align*}
where $H$ is the Hamiltonian. This shows that $H$ is a constant
of motion, and usually we think of $H$ as the energy. The question
is then whether $H$ is actually the energy. And the answer is ``not
always.''

To demonstrate this point, suppose that $T=\sum\frac{1}{2}m\dot{x}_{i}^{2}$,
that $x=x(q)$, that is, $x$ is a function of generalized coordinates
alone (and not generalized velocity or time), with the change-of-coordinates
matrix $A$, and that the force is conservative. Then the kinetic
energy is of the quadratic form
\begin{align*}
T & =\sum_{j,k}A_{jk}\dot{q}_{j}\dot{q}_{k},
\end{align*}
where $\frac{1}{2}m$ is absorbed into $A$. The hamiltonian is
\begin{align*}
H & =\sum\dot{q}_{i}\frac{\partial T}{\partial\dot{q}_{i}}-T+U.
\end{align*}
$\frac{\partial T}{\partial q_{i}}=\sum_{k}A_{ik}\dot{q}+\sum_{j}A_{ij}\dot{q}_{j}\implies\sum\dot{q}_{i}\frac{\partial T}{\partial q_{i}}=2T$;
\begin{align*}
H & =T+U.
\end{align*}
We see that in this example, the Hamiltonian is the energy only because
\begin{align}
\frac{\partial L}{\partial t} & =0,\label{1}\\
U & =U(q),\label{2}\\
T & =T(q)\ \mbox{is quadratic}.\label{3}
\end{align}


\textbf{Example. }Consider a planar system with a bead confined to
move only along a hoop of radius $r$ whose center is attached to
the end of a stick of length $R$. The stick is rotated with a constant
angular velocity $\omega$ as shown in the picture. There is no gravity.

\begin{center}
\includegraphics[scale=0.75]{/Users/tom/Dropbox/Images/L9}
\par\end{center}

\begin{flushleft}
\begin{align*}
(x,y) & =\left(R\cos(\omega t)+r\cos\left(\omega t+\theta\right),R\sin(\omega t)+r\sin\left(\omega t+\theta\right)\right)\\
\implies T & =\frac{1}{2}m\left[R^{2}\omega^{2}+r^{2}\left(\omega+\theta\right)^{2}+2R\omega\left(\omega+\dot{\theta}\right)\cos\theta\right]\\
\implies\ddot{\theta}+\frac{\omega^{2}R}{r}\sin\theta & =0,
\end{align*}
which is the equation of motion of the physical pendulum even if there
is no gravity here. The reason is that an observer sitting at the
end of the stick would feel a fictitious centrifugal force outward
having the magnitude of $m\omega^{2}R\sin\theta$.
\par\end{flushleft}

Another important observation is that $H$ is not the energy since
apparently $T$ is not quadratic. $\frac{\partial H}{\partial t}=0$,
but the energy $\frac{1}{2}I\omega^{2}$ is not conserved since to
keep the angular momentum $L$ constant while $\omega$ is also constant,
the moment of inertia $I$ has to vary in time.


\subsection{Liouvillian}

Define the Liouvillian $\mathcal{L}=\left(\frac{\partial H}{\partial p}\frac{\partial}{\partial q}-\frac{\partial H}{\partial q}\frac{\partial}{\partial p}\right)$.
Then the Hamilton's equations of motion become
\begin{align*}
\dot{p}=\mathcal{L}(p), & \dot{q}=\mathcal{L}(q).
\end{align*}
Now we can write the time evolution of a Hamiltonian system in terms
of a linear map that maps the initial $q(0)$ and $p(0)$ to $q(t)$
and $p(t)$ at later times analogous to the time evolution by a unitary
in quantum mechanics.
\begin{align*}
q(t) & =q(0)+\int_{0}^{t}dt^{'}\mathcal{L}\left(t^{'}\right)q\left(t^{'}\right)\\
 & =q(0)+\int_{0}^{t}dt^{'}\mathcal{L}\left(t^{'}\right)\left(q(0)+\int_{0}^{t^{'}}dt^{''}\mathcal{L}\left(t^{''}\right)q\left(t^{''}\right)\right)+...\\
 & =\left[1+\underbrace{\int_{0}^{t}dt^{'}\mathcal{L}\left(t^{'}\right)}_{\mathcal{L}}+\underbrace{\int_{0}^{t}dt^{'}\mathcal{L}\left(t^{'}\right)\int_{0}^{t^{'}}dt^{''}\mathcal{L}\left(t^{''}\right)}_{\frac{t^{2}}{2!}\mathcal{L}^{2}}+...\right]q(0)\\
 & =e^{\mathcal{L}t}q(0),
\end{align*}
and similarly,
\begin{align*}
p(t) & =e^{\mathcal{L}t}p(0).
\end{align*}
\textbf{Example. }$H=\frac{p^{2}}{2m}-kq$.
\begin{align*}
\mathcal{L} & =\frac{p}{m}\frac{\partial}{\partial q}+k\frac{\partial}{\partial p}.
\end{align*}
$\mathcal{L}(p)=k$, $\mathcal{L}^{2}(p)=\mathcal{L}\left(\mathcal{L}(p)\right)=\mathcal{L}k=0$.
So the series terminates and we get that
\begin{align*}
p(t) & =e^{\mathcal{L}t}(p)\Bigr|_{0}=p(0)+kt.
\end{align*}
$\mathcal{L}(q)=\frac{p}{m}$, $\mathcal{L}^{2}(q)=\frac{k}{m}$,
$\mathcal{L}^{3}(q)=0$. So
\begin{align*}
q(t) & =e^{\mathcal{L}t}(q)\Bigr|_{0}=q(0)+\frac{p(0)}{m}t+\frac{k}{2m}t^{2}.
\end{align*}
We can go further and write the Liouvillian as a superoperator similar
to the time evolution $A(t)=U^{\dagger}A(0)U$ of an operator $A$
in the Heisenberg picture.


\subsection{Coordinate transformation}

A mass on a spring is attached to a car moving with velocity $v$
such that $H=\frac{p^{2}}{2m}+\frac{1}{2}k(x-vt)^{2}.$ Suppose that
we want to simplify the problem by going into the moving frame and
define $X=x-vt$. What is the conjugate momentum $P$?

Go back to the Lagrangian;
\begin{align*}
L & =\frac{1}{2}m\dot{x}^{2}-\frac{1}{2}k(x-vt)^{2}\\
X & =x-vt\\
\dot{X} & =\dot{x}-v.
\end{align*}
The new Lagrangian is
\begin{align*}
\tilde{L} & =\frac{1}{2}m\left(\dot{X}+v\right)^{2}-\frac{1}{2}kX^{2}.
\end{align*}
The new momentum is
\begin{align*}
P=\frac{\partial\tilde{L}}{\partial\dot{X}} & =m\left(\dot{X}+v\right).
\end{align*}
Then
\begin{align*}
\tilde{H}=P\dot{X}-\tilde{L} & =\frac{\left(P-mv\right)^{2}}{2m}+\frac{1}{2}kX^{2}-\underbrace{\frac{1}{2}mv^{2}}_{\mbox{constant}}.
\end{align*}
Without going back to the Lagrangian, we might be tempt to simply
write
\begin{align*}
P & =m\dot{X}=p-mv\\
\implies\frac{p^{2}}{2m} & =\frac{\left(P+mv\right)^{2}}{2m}.
\end{align*}
But that would be off by a sign.


\subsubsection{Action Principle for the Hamiltonian}

In searching for a set of transformations that preserve Hamilton's
equations of motion, we go back to the action principle. Can the action
principle be formulated in terms of the Hamiltonian alone? The answer
is yes. Let the action be
\begin{align*}
S & =\int_{t_{1}}^{t_{2}}dt\left(p\dot{q}-H(q,p,t)\right).\\
0=\delta S & =\int_{t_{1}}^{t_{2}}dt\left(\delta p\dot{q}+\underbrace{p\delta\dot{q}}_{\frac{d}{dt}(p\delta q)-\dot{p}\delta q}-\frac{\partial H}{\partial q}\delta q-\frac{\partial H}{\partial p}\delta p\right).
\end{align*}
Assume that $\delta q$ and $\delta p$ are varied independently.
\begin{align*}
0 & =p\underbrace{\delta q\Big|_{t_{1}}^{t_{2}}}_{0}+\int_{t_{1}}^{t_{2}}dt\left(\delta p\left(\dot{q}-\frac{\partial H}{\partial p}\right)-\left(\dot{p}+\frac{\partial H}{\partial q}\right)\delta q\right).
\end{align*}
The demand that the action is extremized is equivalent to Hamilton's
equations
\begin{align*}
\dot{q}-\frac{\partial H}{\partial p} & =0,\\
\dot{p}+\frac{\partial H}{\partial q} & =0.
\end{align*}
Curiously, this does not required the variation in the nomentum at
the end points to vanish, although we will assume that from now on.
Then for any scalar function $F(q,p,t)$, the action can be modified
\begin{align*}
S & =\int_{t_{1}}^{t_{2}}dt\left(p\dot{q}-H(q,p,t)+\frac{dF}{dt}\right).
\end{align*}



\subsubsection{Canonical transformations}

It is clear from the previous section that for any new Hamiltonian
$\mathcal{H}(Q,P,T)$ to give Hamilton's equations, we need

\begin{align*}
dt\left(p\dot{q}-H(q,p,t)\right) & =dt\left(P\dot{Q}-\mathcal{H}(Q,P,t)+\frac{dF(q,p,Q,P,t)}{dt}\right)\\
pdq-Hdt-PdQ & =\mathcal{H}dt+dF,
\end{align*}
where $F(q,p,Q,P,t)$ is the \emph{generating function }for a canonical
transformation (preserving Hamilton's equations).

We define four types of transformations:

Type 1:\textbf{ }$F=F(q,Q)$

Type 2: $F=F(q,P)$

Type 3: $F=F(p,Q)$

Type 4: $F=F(p,P)$

\textbf{Example. }For type 1 transformation $F=qQ$,
\begin{align*}
pdq-Hdt-PdQ & =\mathcal{H}dt+\frac{\partial F}{\partial q}dq+\frac{\partial F}{\partial Q}dQ=\mathcal{H}dt+Qdq+qdQ
\end{align*}
\begin{align*}
 & \implies\begin{cases}
q & =-P,\\
p & =Q.
\end{cases}
\end{align*}


\textbf{Example. }A type 1 transformation of a harmonic oscillator
$H=\frac{p^{2}}{2m}+\frac{1}{2}m\omega^{2}q^{2}$. It would be nice
if we could transform the Hamiltonian such that it has no $Q$ dependence.
That is, 
\begin{align*}
H & =\frac{f^{2}(p)}{2m}.
\end{align*}
We therefore consider this transformation
\begin{align*}
p & =f(p)\sin Q,\\
q & =\frac{f(p)}{m\omega}\cos Q,
\end{align*}
and find the generating function $F$.
\begin{align*}
\frac{p}{q} & =m\omega\tan Q\\
p=\left(\frac{\partial F}{\partial q}\right)_{Q} & =m\omega q\tan Q\\
\implies F & =\frac{m\omega q^{2}}{2}\tan Q+g(Q)
\end{align*}
Choose $g(Q)=0$. Then
\begin{align*}
P=-\left(\frac{\partial F}{\partial Q}\right)_{q} & =-\frac{m\omega q^{2}}{2}\sec^{2}Q\\
 & =-\frac{f^{2}(p)}{2m\omega}\cos^{2}Q\sec^{2}Q\\
\implies f(p) & =i\sqrt{2m\omega P}
\end{align*}
\begin{align*}
\implies & \boxed{H=-\omega P}.
\end{align*}
The Hamiltonian becomes that of the free particle with $P=P(0)$ and
$Q=Q(0)-\omega t$! Say, in the study of fluctuation-dissipation relation,
if you want to phenomenologically put friction in the system, it is
much easier to put friction in at this level after you have transformed
the oscillator into a free particle.

\textbf{Example. }A type 2 transformation of a harmonic oscillator
$H=\frac{p^{2}}{2m}+\frac{1}{2}m\omega^{2}q^{2}$.

Let $P$ be the ``ladder operators.''
\begin{align*}
P & =\frac{p+im\omega q}{\sqrt{2}}
\end{align*}
Look for $F(q,P)$.
\begin{align*}
pdq-Hdt+QdP & =\mathcal{H}dt+\frac{\partial F}{\partial q}dq+\frac{\partial F}{\partial P}dP\\
p=\left(\frac{\partial F}{\partial q}\right)_{P} & =\sqrt{2}P-im\omega q\\
\implies F & =\sqrt{2}Pq-\frac{im\omega q^{2}}{2}+g(P)\\
Q=\left(\frac{\partial F}{\partial P}\right)_{q} & =\sqrt{2}q+g^{'}(P).
\end{align*}
Choose $g^{'}(P)=\frac{iP}{m\omega}$. Then
\begin{align*}
Q & =\sqrt{2}q+\frac{iP}{m\omega}=\sqrt{2}q+\frac{i}{m\omega}\left(\frac{p+im\omega q}{\sqrt{2}}\right)=\sqrt{2}q+\frac{ip}{\sqrt{2}m\omega}-\frac{q}{\sqrt{2}}=\frac{q+\frac{ip}{m\omega}}{\sqrt{2}}.
\end{align*}
(This is wrong but it was from the lecture. Anyway, assuming the result,
the Hamiltonian is)
\begin{align*}
 & \boxed{\mathcal{H}=i\omega PQ},
\end{align*}
which is the analog of $H=\omega a^{\dagger}a$ in quantum mechanics,
and $P$ and $Q$, like the ladder operators, evolve trivially in
time.
\begin{align*}
P(t)=P(0)e^{-i\omega t}, & Q(t)=Q(0)e^{i\omega t}.
\end{align*}

%\end{document}




%\bibliographystyle{amsplain}
% \nocite{*}
%\bibliography{F:/References/Research/Queen}
%\bibliography{/home/qxd/Documents/References/Research/Queen}

\appendix

\chapter{Modeling classical systems}

We divide classical systems into two kinds: one is the systems with only one particle or object; the other one is the systems with many interacting particles. 

For the first kind of systems, we can either use physics laws such as Newton's laws or Lagrangian/Hamiltonian mechanical knowledge to establish a set of equations of motion of the object. 
The systems can usually be described by up-to second-order differential equations of displacement. Examples are given in Section~\ref{sec:equationofphysics} and Chapter~\ref{chap:dampledosc}. The basis of lagrangian and Hamiltonian mechanics is studied in Chapter~\ref{chap:lagham}. 

For the second kind of systems, the interactions between particles should be considered. Therefore, if the particles are discrete, we can conduct the modeling process similar to the sections from~\ref{sec:particles} to~\ref{sec:defects}; if the particles can be viewed as continua, as proved in Section~\ref{sec:field}, a field theory model can be obtained by reducing the distance of particles in a multi-particle model. Specifically, based on continuity equation and conductive relations, a fluid field can be simplified to typical field models described by diffusion/telegrapher's/wave/Fisher's equations. In most cases of this course, we make our equations dimensionless for simplicity.

As a special case of the second kind of systems, synchronization phenomenon is common in few-body systems (see Chapter.~\ref{chap:synchronization}). This model has its counterparts in two-level quantum systems. 

Notice that, for the first kind of systems, equations are usually derived with respect to displacement. But for the second kind of systems, especially for field problems, the equations are usually established with respect to wave/density/probability functions which is no longer a direct observable physics quantity. Usually, to model the second kind of systems, we need to know the interaction rules between elements and some extra boundary/natural conditions or relations to the variables. Finding a proper analyzable variable and completed relations is the key to model a complicated system. 

Once we obtain the equations governing the motion of systems, we can perform all sorts of skills to solve the mathimatical equations. Alternatively, we can also study some of the properties of the system by looking at the stability, singularity and solvability of the equations. One related theory called flow and bifurcation theory is introduced in Chapter~\ref{chap:bifurcation}. 

By solving the equations of classical mechanical systems, we can analogize the systems to basic physics models, including physical pendulum (see Chapter~\ref{chap:dampledosc}), wave propagator, diffusion (see Section~\ref{sec:field} and Chapter~\ref{chap:fluid}) and so on. In most cases, the system can be analyzed using Green function (in the $ x $- or $ k $-domain) or memory function (in the time-domain, see Chapter~\ref{chap:memoryfunc}) theory to treat the system as a propagator evolving from a initial point or time to any position and time points.  

%\documentclass[12pt]{report}

\chapter{Methods of solving Differential Equations}
Since a primary part of this course is to solve differential equations. We summarize as below the methods we have discussed in this course. 

\section{Solving ODEs}
Ordinary differential equations (ODEs) can usually be solved numerically. For a linear ODE, the analytical solution can be found by combining the general solution of the corresponding homogeneous equation and one special solution of the equation (if it is inhomogeneous). Many differential equation or advanced calculus textbooks have discussed the techniques to solve ODEs. Here, we list a few techniques mentioned in this course of study in classical mechanics. 

\subsection{Laplace transformation}
This method is suitable for simple equation of $(\vec{x},\vec{v},t)$. Any functions in the equation are better to be of $ t $; otherwise, it will be difficult to solve. See Section~\ref{Sec:Laplace} and HW1. 

\subsection {Lower the order by variable replacements}
For a function in the form of $f\left(\dt{x},\,\frac{d^2 x}{dt^2},\,\ldots,\,\frac{d^n x}{dt^n},\ldots \right)$, one can replace $ \dt{x} $ by $ v $ so that the order is lowered down. See Section~\ref{sec:vreplace}~\nameref{sec:vreplace}.

\subsection{Energy method for $ \sdt{x}=f(x) $}
If the DE can be written as $ \sdt{x}=f(x) $, one can apply the energy method to solve it and visualize the behavior of the equation. The solution can be generalized as 
\begin{align}
\int{\frac{dx}{\sqrt{2[E-U(x)]}}}=t+const.
\end{align}
See Section~\ref{sec:energymethod}.

Because of the formalism of the solution, one can often apply the property of elliptic equations to give the solution of DEs in terms of elliptic functions. See Section~\ref{sec:ellipticequ} and HW2-3. 

Know forms of equations which can be solved in terms of elliptic functions:
\begin{align}
\sdt{x}&=f(x) \leftrightarrow \int{\frac{dx}{\sqrt{2[E-U(x)]}}}=t+const.\\
\frac{dt}{dx} &=\sqrt{1-z^2}\sqrt{1-k^2z^2} \leftrightarrow 
t=\int^x_0 \frac{dz}{\sqrt{1-z^2}\sqrt{1-k^2z^2}} \\
\end{align}
One practical example is the physics pendulum which can be described by
\begin{align}
-mg \sin{\theta} &=ml\frac{d^2}{dT^2}\theta\\
\Rightarrow
\sdt{x}+\sin{x} &=0,
\end{align}
where  $ t=\frac{g}{l}T $ and $ x=\theta $. See Section~\ref{sec:pendulum} and the part before for details.

Another example is the coupled multi-level quantum system which can be simplified as a physics pendulum problem. See Section~\ref{sec:quantumlevels}.

\subsection{Singular perturbation theory to solve high-order equations}
A typical equation we studied which can be solved using the singular perturbation method is
\begin{equation}
\frac{d^2x}{dt^2}+ \alpha \left(\frac{dx}{dt}\right)^{2n+1} + \omega^2 x = 0,\quad n=0,\,1,\,\cdots 
\end{equation}
Normal perturbation method generates a dangerous term, while the singular perturbation method can give a high accurate solution. 
See Chapter~\ref{chap:singularperturbation}. 

\subsection{Solving a memory equation}
Differentiate to give a second order equation. See Chapter~\ref{chap:memoryfunc}. Usual form: damped pendulum. 

\subsection{Synchronization}
Coupled equations to rearrange and simplify. See Chapter~\ref{chap:synchronization}. 

\subsection{Analyzing nonlinear equations}
Since the solution of nonlinear equations usually depends on the initial value and have some issues on uniqueness and stabilization, singularization and bifurcation theory can be used to analyze those properties of an equation system (not limited in nonlinear equations). See Chapter~\ref{chap:bifurcation}. 

\subsection{Solving many-particle systems}
For discrete many-body systems, coupled equations can be derived to describe the behavior of the system. 

Based on the symmetry/periodicity of equations, we may be able to only look at one cell of the system (see Sections.~\ref{sec:particles} to~\ref{sec:defects}). The basic idea is that we apply Fourier transformation to the equation in real space so that we can get a set of equation in phase space, where the order of the equations is lower. The characteristics of the equations can be obtained through solving the corresponding eigenvalue problem (see cell method in Section~\ref{sec:periodicchain}). 

Alternatively, we can also replace the sinusoidal functions with exponential functions where the characteristic frequencies indicate the spectra. By comparing the coefficients of different spectra, we can also obtain the properties of the system. See the Alternating method in Section~\ref{sec:periodicchain} and corresponding homeworks. 

In the case that the distance of the particle is very small compared to the total length, the problem can be transformed to a field problem. PDEs can be obtained to give a better analysis. 

\section{Solving PDEs}
Once the object we are analyzing is not as easy as $ x(t) $, or when we step in high-dimensional problems, PDEs are usually used to describe a system of motion. As stated in earlier sections, the object described in PDE systems are usually density function, field function or energy function and so on. 

In physics, the PDEs are usually given respect to time and spatial axis. Therefore, there are boundary problems and initial condition problems. 

The solution of a boundary problem may be derived through expanding the eigenfunctions, or separating variables. Related methods have been widely studied in various textbooks on Mathematical Methods of Physics, etc. 

Generally, the order of a high-order PDE can be 
lowered through Laplace and Fourier transformations (see Section~\ref{sec:field}). For a first-order PDE, we can use the method of characteristics to transform the PDE problem into ODE problem along characteristic curves. In solving the ODEs, initial conditions may be used to identify unknown coefficients which are generated in the process of integration. Then through the inversion of Fourier and Laplace transformations, we can finally get the solution of a PDE problem. 

There are many well known equations named as, for example, wave equation, diffusion equation, advective-diffusion equation, telegrapher's equation, Smoluchowski equation, Fisher's equation and Burger's equation. They are introduced through Section~\ref{sec:field} to Chapter~\ref{chap:fluid}. 

To my understanding, the art of solving PDEs may be related to transferring coordinates. Take the D'Alambert wave equation and the method of characteristics for example, we find another coordinate so that we can separate variables and transform the PDEs into ODEs which is easier to solve. In this sense, the Lagrange and Hamilton mechanics and coordinate transformation may provide a general way of modeling and solving differential problems in solvable coordinates. 


%\end{document}

\end{document} 