\chapter {Synchronization and nonlinear systems}\label{chap:synchronization}
Q: what is meant by \textit{nonlinearity}? (nonlinear terms in equations). And so what? (the properties are changed, nonlinear phenomenon and their richness.) 

Two aspects to be explored in this section: toughness and richness.

Comparing 2 equaitons--one is linear, the other one is nonlinear:
\begin{align}
\dt{x} &=-\alpha x,\label {examplelinear1}\\
\dt{x} &=-\beta x^2.\label {examplenonlinear2}
\end{align}

Aside on synchronization. We let

\begin{align}
\dt{\theta_1} &=\omega_1 + l(\theta_2-\theta_1),\label {coupling1}\\
\dt{\theta_2} &= \omega_2 + l(\theta_1-\theta_2).\label {coupling2}
\end{align}

Two of them are coupled together. Add together to give
\begin{align}
\dt{(\theta_1+ \theta_2)} = \omega_1 + \omega_2.
\end{align}
Subtract them to give
\begin{align}
\dt{(\theta_1-\theta_2)}= (\omega_1 - \omega_2) -2l(\theta_1 - \theta_2).
\end{align}
Let $ D=\theta_1 - \theta_2  $, then 
\begin{align}
\dt{D} + 2lD=\omega_1-\omega_2.
\end{align}
If alternating $ \omega_1-\omega_2 \rightarrow \frac{\omega_1-\omega_2}{2l}$, we can write the equation in the form of
\begin{align}
\dt{y}+ \alpha y =B.
\end{align}
The diagram of $ y(t) $ is shown below.
%\missingfigure{925-2}
\scalefig{../Figs/handdraw925_2}{0.5}{Evolution of $ y(t) $.}

This coupling system is studied by Huygens before. If the interaction $ l(\theta_1 - \theta_2) $ is nonlinear, for example, the interaction term is $ \sin(\theta_1- \theta_2) $, there will be rich behaviors of the equation. The synchronization exists or not depends on the interaction term (whether it exists or not). 

The solution of the first equation (Equ.~\eqref{examplelinear1}) is
\begin{align}
x(t)= x(0) e^{-\alpha t}.
\end{align} 
The solution of the second equation (Equ.~\eqref{examplenonlinear2}) is
\begin{align}
x(t)= x(0)\frac{1}{1+x(0)\beta t}.
\end{align}
The second one depends on $ x(0) $, the first one does not. So, the second solution cannot be superimposed.


\textbf{Sep 27}

In a nonlinear case, for the coupling oscillation case, we can have
\begin{align}
\dt{\theta_1} &=\omega_1 - l\sin(\theta_2-\theta_1),\label {coupling1nonlinear}\\
\dt{\theta_2} &= \omega_2 - l\sin(\theta_1-\theta_2).\label {coupling2nonlinear}
\end{align}
Therefore, 
\begin{align}
\dt{(\theta_1 + \theta_2)}&= \omega_1+\omega_2\\
\dt{\theta}+2l\sin\theta &= \Delta,
\end{align}
where $ \theta=\theta_1-\theta_2 $ and $ \Delta=\omega_1 -\omega_2 $. 

Let us solve the equation that
\begin{align}
\dt{\theta}+2l\sin\theta=0.
\end{align}
We obtain 
\begin{align}
\int \frac{d\theta}{\sin \theta}=-2l t + const.
\end{align}
Now, the integral is
\begin{align}
\int \frac{d\theta}{\sin \theta} &= \int \frac{d\left(\theta/2 \right)}{\sin\left(\frac{\theta}{2} \right)\cos\left(\frac{\theta}{2} \right)} \\
&= \int \frac{d\left(\theta/2 \right) \frac{1}{\cos^2\frac{\theta}{2}}}{\dfrac{\sin\left(\frac{\theta}{2} \right)}{\cos\left(\frac{\theta}{2} \right)}}\\
&= \int \frac{dy\sec^2 y}{\tan^2 y}\\
&= \int \frac{d \left(\tan y \right)}{\tan y}\\
&= \ln \left(\tan y \right) ,\quad (\text{depending on the limits})
\end{align}
and the equation can be solved. 

In this lecture, we are going to answer the following questions:

\begin{itemize}
\item What exactly nonlinearity means?
\item What is bad about it?
\item What is good about it?
\end{itemize}

Back to our equations~\eqref{examplelinear1} and~\eqref{examplelinear2}. The solutions are 
\begin{equation}
\begin{array}{ccc}
\boxed{\frac{x(t)}{x(0)}=e^{-\alpha t}} & \text{versus} & \boxed{\frac{x(t)}{x(0)}=\frac{1}{x(0)\beta t +1}}.
\end{array}
\end{equation}

Concepts of Propagators or Green functions are USELESS for the second function case, which is a  nonlinear equation. That means superposition does not work in the nonlinear regime. 

There are two factors affecting the behaviour of the nonlinear equation:
one is the initial condition, the other is the stimulus response. 
If the two factors are both linear, the two factors work as the same effect.

This can be seen in as follows.
\begin{align}
\dt{\theta}+ 2l \theta &= \Delta\\
\Rightarrow \varepsilon \tilde{\theta} &= \theta(0)+2l \tilde{\theta}=\frac{\Delta}{\varepsilon}\\
\tilde{\theta}&= \theta(0)\frac{1}{\varepsilon +2l}+ \frac{\Delta}{\varepsilon}\frac{1}{\varepsilon +2l}
\end{align}
where the $ \frac{1}{\varepsilon + 2l} $ is the linear term of the propagator $ e^{-2lt} $.
\begin{align}
\Rightarrow \theta(t)= \theta(0) G(t)+ \int_0^t dt' G(t-t')\Delta.
\end{align}
The initial condition and the stimulus can be the same if they are linear.

With stimulus, $ S $,
\begin{align}
\dt{x}= -\alpha x +S\quad \text{versus} \quad \dt{x}= -\beta x^2 +S.
\end{align}
In the limit that $ t\rightarrow \infty $, $ x(\infty)=\frac{S}{\alpha} $. 
The general solution gives
\begin{align}
\int \frac{dx}{1-\dfrac{\beta}{S}x^2}= St+const.
\end{align}
The integral can be solved using the following relationship.
\begin{align}
\int \frac{dy}{1-y^2}= \int \frac{dy}{(1-y)(1+y)}=\frac{1}{2} \int \left[\frac{1}{1-y}+ \frac{1}{1+y} \right]dy.
\end{align}

If we have 
\begin{align}
\dt{x}=-\alpha x -\beta x^2- \delta x^3,
\end{align}
it can be solved similarly. 