%%% LyX 2.0.3 created this file.  For more info, see http://www.lyx.org/.
%%% Do not edit unless you really know what you are doing.
%\documentclass[11pt,english]{article}
%\usepackage[T1]{fontenc}
%\usepackage[latin9]{inputenc}
%\usepackage{geometry}
%\geometry{verbose,tmargin=1in,bmargin=1in,lmargin=1in,rmargin=1in}
%\usepackage{amsmath}
%\usepackage{graphicx}
%\usepackage{esint}
%
%\makeatletter
%
%%%%%%%%%%%%%%%%%%%%%%%%%%%%%%% LyX specific LaTeX commands.
%%% A simple dot to overcome graphicx limitations
%\newcommand{\lyxdot}{.}
%
%
%\makeatother
%
%\usepackage{babel}
%\begin{document}
%
%\title{Lagrangian and Hamiltonian Mechanics}
%
%
%\date{Lectures by Professor David H. Dunlap and Tzu-Cheng Wu}
%
%
%\author{Ninnat Dangniam}
%
%\maketitle
%\tableofcontents{}


\section{Lagrangian and Hamiltonian formalism}


\subsection{The principle of extremum action}

Newtonian formalism describes a mechanical system as a differential
equation and thus is local in time. In Lagrangian formalism, we assume
that we are given two times $t_{1}$ and $t_{2}$ with a particle
being at the position $x_{1}$ and $x_{2}$ respectively, the path
of the system on the configuration space is the one that extremizes
the action
\begin{align*}
S & =\int_{t_{1}}^{t_{2}}L\left(x,\dot{x},t\right)dt.\\
0=\delta S & =\int_{t_{1}}^{t_{2}}\delta L\left(x,\dot{x},t\right)dt\\
 & =\int\left(\frac{\partial L}{\partial x}\delta x+\frac{\partial L}{\partial\dot{x}}\underbrace{\delta\dot{x}}_{d\left(\delta x\right)/dt}+\frac{\partial L}{\partial t}\underbrace{\delta t}_{0}\right)dt\\
 & =\int\left(\frac{\partial L}{\partial x}\delta x+\frac{\partial L}{\partial\dot{x}}\frac{d}{dt}\delta x\right)dt
\end{align*}
Using integration by path, the second term on the RHS can be expressed
as
\begin{align*}
\int dt\frac{\partial L}{\partial\dot{x}}\frac{d}{dt}\delta x= & \underbrace{\delta x\frac{\partial L}{\partial\dot{x}}\Biggr|_{t_{1}}^{t_{2}}}_{0}-\int\frac{d}{dt}\frac{\partial L}{\partial x}\delta xdt
\end{align*}
where the first term vanishes because the end points are fixed. Therefore,
\begin{align*}
0 & =\int\left(\frac{\partial L}{\partial x}-\frac{d}{dt}\frac{\partial L}{\partial x}\right)\delta xdt.
\end{align*}
It can be shown that this implies the Euler-Lagrange equation
\begin{align*}
 & \boxed{\frac{\partial L}{\partial x}-\frac{d}{dt}\frac{\partial L}{\partial x}=0}.
\end{align*}
For a conservative force field, $L=T-U$ leads to the familiar equation
of motion $m\ddot{x}=-\frac{\partial U}{\partial x}$. For nonconservative
force field, the Euler-Lagrange equation can be modified to take into
account the nonconservative force by separating the generalized force
$Q$ into the conservative and nonconservative parts
\begin{align*}
Q & =-\frac{\partial U(q)}{\partial q}+Q_{\mbox{nc}}.
\end{align*}
Then
\begin{align*}
\frac{d}{dt}\frac{\partial T}{\partial\dot{q}}-\frac{\partial T}{\partial q} & =Q\\
\implies\frac{d}{dt}\frac{\partial L}{\partial\dot{q}}-\frac{\partial L}{\partial q} & =Q_{\mbox{nc}}.
\end{align*}



\subsection{Legendre transform}

A puzzle: given a function $f(x)$, let $\frac{df}{dx}\Bigr|_{x_{0}}=\phi$.
Can we find a function $G(\phi)$ such that $\frac{dG}{d\phi}=x$?

If $f(x)=ax^{n}$, then $G(\phi)=F\left(x(\phi)\right)$ works since
$\frac{dG}{d\phi}=\frac{df}{dx}\frac{dx}{d\phi}=\phi\frac{dx}{d\phi}=\phi\frac{d}{d\phi}\left(\frac{\phi}{an}\right)^{\frac{1}{n-1}}\propto x$.
Note that we have to invert $\phi(x)$ to find $x(\phi)$. The general
recipe for any function of $x$ is the \emph{Legendre transform}
\begin{align*}
 & \boxed{G(\phi)=-\phi x(\phi)-F\left(x(\phi)\right)}
\end{align*}
since
\begin{align*}
\frac{dG}{d\phi} & =x(\phi)+\phi\frac{dx}{d\phi}-\phi\frac{dx}{d\phi}=x.
\end{align*}
Geometrically, $-G(\phi)$ is the intercept of the tangent of $f$
at a point $x_{0}$ with the $x=0$ axis.

At $x_{0}$,
\begin{align*}
y-F\left(x_{0}\right) & =\frac{dF}{dx}\Bigr|_{x_{0}}(\overbrace{x}^{0}-x_{0})\\
y & =F\left(x_{0}\right)-\phi x.
\end{align*}


\textbf{Example. }$f(x,y)=\frac{ax^{2}}{2}+\frac{by^{2}}{2}+cxy$.
Then the process of finding $G(\phi,y)$ is the same given that $y$
is hold constant at all time.


\subsection{Hamiltonian}

Gievn a Lagrangian $L$, the momentum $p$ canonically conjugated
to a coordinate $q$ is defined by
\begin{align*}
p\left(q,\dot{q},t\right) & \equiv\frac{\partial L\left(q,\dot{q},t\right)}{\partial\dot{q}}.
\end{align*}
Substitute this into the Euler-Lagrange equation gives
\begin{align*}
\dot{p}\left(q,\dot{q},t\right) & =\frac{\partial L\left(q,\dot{q},t\right)}{\partial q}.
\end{align*}
Inverting the functions of $\dot{q}$ to functions of $p$ gives the
Hamilton's equations of motion.

\textbf{Example. }A harmonic oscillator $L=\frac{1}{2}m\dot{q}^{2}-\frac{1}{2}kq^{2}$.
\begin{align*}
p=\frac{\partial L}{\partial\dot{q}}=m\dot{q} & \iff\dot{q}=\frac{p}{m},\\
\dot{p}=\frac{\partial L}{\partial q} & \iff\dot{p}=-kq.
\end{align*}
To do this systematically, define $H$ to be the Legendre transform
of $L$: $H(p,q,t)=p\dot{q}-L$. Then the Hamilton's equation of motion
are
\begin{align*}
 & \boxed{\dot{q}=\frac{\partial H}{\partial p},\ \dot{p}=-\frac{\partial H}{\partial q}}
\end{align*}
For the harmonic oscillator,
\begin{align*}
H & =\frac{p^{2}}{2m}+\frac{kq^{2}}{2}\\
\implies\ddot{p}+\omega^{2}p & =0\\
\implies p(t) & =p(0)\cos(\omega t)-\frac{kq(0)}{\omega}\sin(\omega t),\\
q(t) & =q(0)\cos(\omega t)+\frac{p(0)}{m\omega}\sin(\omega t).
\end{align*}
By defining $Q=q(km)^{\frac{1}{4}}$ and $P=\frac{p}{(km)^{\frac{1}{4}}}$,
the conservation of energy takes on a very simple form
\begin{align*}
Q^{2}+P^{2} & =\frac{2E}{\omega},
\end{align*}
where the trajectory in phase space is a circle with radius $\sqrt{\frac{2E}{\omega}}$.
Notice that we have eliminated $q$ first to get the equation of motion
for $p$. The Hamiltonian treats $q$ and $p$ symmetrically.


\subsection{Constants of motion}

Suppose that $\frac{df(q,p,t)}{dt}$ is a constant, then
\begin{align*}
0=\frac{df(q,p,t)}{dt} & =\frac{\partial f}{\partial q}\dot{q}+\frac{\partial f}{\partial p}\dot{p}+\frac{\partial f}{\partial t}\\
 & =\underbrace{\left(\frac{\partial f}{\partial q}\frac{\partial H}{\partial p}-\frac{\partial f}{\partial p}\frac{\partial H}{\partial q}\right)}_{\left\{ f,H\right\} }+\frac{\partial f}{\partial t}
\end{align*}
Useful identities for Poisson brackets:
\begin{align*}
\left\{ f,gh\right\}  & =g\left\{ f,h\right\} +\left\{ f,g\right\} h.\\
\left\{ \left\{ f,g\right\} ,h\right\}  & =\left\{ \left\{ h,f\right\} ,g\right\} =\left\{ \left\{ g,h\right\} ,f\right\} .
\end{align*}


\textbf{Example. }$H=\frac{p^{2}}{2m}-kq$.

Consider an apparently contrived function $f=q-\frac{p}{m}t+\frac{k}{2m}t^{2}$.
It is a constant of motion;
\begin{align*}
\frac{df}{dt}=\left\{ f,H\right\} +\frac{\partial f}{\partial t} & =\frac{p}{m}-\left(-\frac{t}{m}\right)\left(-k\right)-\frac{p}{m}+\frac{kt}{m}=0.
\end{align*}
Consider another function $g=p-kt$. It is also a constant of motion;
\begin{align*}
\frac{dg}{dt}=\left\{ g,H\right\} +\frac{\partial g}{\partial t} & =0\cdot\frac{\partial H}{\partial p}-(-k)-k=0.
\end{align*}
In fact, $f$ and $g$ are nothing but initial conditions $q(0)$
and $p(0)$ respectively. These are called \emph{trajectory constants}.
Any constant of motion is a combination of trajectory constans. Thus,
there always exist $2n$ constants of motion whether $H$ depends
explicitly on time or not.

\textbf{Example. }A free particle $H=\frac{p^{2}}{2m}$. $q$ is a
\emph{cyclic coordinate}, i.e. $q$ is absent from the Hamiltonian$\implies$$p=-\frac{\partial H}{\partial q}$
is conserved. 

\textbf{Example. }$H=p_{1}\left(\frac{p_{2}}{m}-\gamma q_{1}\right)$.
There is no $q_{2}$ in the Hamiltonian, so $p_{2}$ is a constant.
\begin{align*}
\dot{q}_{1} & =\frac{\partial H}{\partial p_{1}}=\frac{p_{2}(0)}{m}-\gamma q_{1}\\
\implies q_{1}(t) & =q(0)e^{-\gamma t}+\frac{p_{2}(0)}{m}\frac{1-e^{\gamma t}}{\gamma}.
\end{align*}


\textbf{Example. $H=p_{1}\left(\frac{p_{2}}{m}-\gamma q_{1}\right)+m\omega^{2}q_{1}q_{2}$}.
\begin{align*}
\dot{p}_{2} & =-\frac{\partial H}{\partial q_{2}}=-m\omega^{2}q_{1}\\
\dot{q}_{1} & =\frac{\partial H}{\partial p_{1}}=\frac{p_{2}}{m}-\gamma q_{1}\\
\implies & \boxed{\ddot{q}_{1}+\gamma\dot{q}_{1}+m\omega q_{1}=0}.
\end{align*}
We have obtained the equation of motion of a damped harmonic oscillator
even though there is no explicit time dependence in the Hamiltonian!

\textbf{Example. }A rotor with two particle with masses $m_{1}$ and
$m_{2}$ and the reduced mass $\mu=\frac{m_{1}m_{2}}{m_{1}+m_{2}}$
and the Lagrangian
\begin{align*}
L & =\frac{\mu}{2}\left[\dot{r}^{2}+\left(r\dot{\theta}\right)^{2}+\left(r\sin\theta\dot{\phi}\right)^{2}\right]-U(r).
\end{align*}
$p_{r}$ is the momentum and $p_{\theta}$ is the angular momentum.
But it is not obvious by looking at $p_{\phi}=\frac{\partial L}{\partial\dot{\phi}}=\mu r^{2}\sin^{2}\theta\dot{\phi}$
whether $p_{\phi}$ is a constant of motion or not. We can write down
the Hamiltonian
\begin{align*}
H & =p_{r}\dot{r}+p_{\theta}\dot{\theta}+p_{\phi}\dot{\phi}-L\\
 & =\frac{p_{r}^{2}}{2\mu}+\frac{p_{\theta}^{2}}{2\mu r^{2}}+\frac{p_{\phi}^{2}}{2\mu r^{2}\sin^{2}\theta}+U(r).
\end{align*}
Consider
\begin{align*}
p_{\theta}\dot{p}_{\theta} & =p_{\phi}^{2}\frac{\cos\theta}{\sin^{2}\theta}\dot{\theta}\\
\frac{1}{2}\frac{d}{dt}\left(p_{\theta}^{2}\right) & =-\frac{1}{2}p_{\phi}^{2}\frac{d}{dt}\frac{1}{\sin^{2}\theta}\\
\implies p_{\theta}^{2}+\frac{p_{\phi}^{2}}{\sin^{2}\theta}\equiv L^{2} & =\mbox{constant}.
\end{align*}
Using what we just found out, the Hamiltonian can be written in the
familiar form (as seen in quantum mechanics, for example)
\begin{align*}
 & \boxed{H=\frac{p_{r}^{2}}{2\mu}+\frac{L^{2}}{2\mu r^{2}}+U(r)}
\end{align*}



\subsection{When the Hamiltonian is not the energy}

Suppose that $L=L\left(q,\dot{q}\right)$. Then
\begin{align*}
\frac{dL}{dt} & =\frac{\partial L}{\partial q}\dot{q}+\frac{\partial L}{\partial\dot{q}}\ddot{q}\\
 & =\frac{d}{dt}\frac{\partial L}{\partial\dot{q}}\dot{q}+\frac{\partial L}{\partial\dot{q}}\ddot{q}\ (\mbox{cancellation of dots})\\
0 & =\frac{d}{dt}\boxed{\underbrace{\left(\dot{q}\frac{\partial L}{\partial\dot{q}}-L\right)}_{H}},
\end{align*}
where $H$ is the Hamiltonian. This shows that $H$ is a constant
of motion, and usually we think of $H$ as the energy. The question
is then whether $H$ is actually the energy. And the answer is ``not
always.''

To demonstrate this point, suppose that $T=\sum\frac{1}{2}m\dot{x}_{i}^{2}$,
that $x=x(q)$, that is, $x$ is a function of generalized coordinates
alone (and not generalized velocity or time), with the change-of-coordinates
matrix $A$, and that the force is conservative. Then the kinetic
energy is of the quadratic form
\begin{align*}
T & =\sum_{j,k}A_{jk}\dot{q}_{j}\dot{q}_{k},
\end{align*}
where $\frac{1}{2}m$ is absorbed into $A$. The hamiltonian is
\begin{align*}
H & =\sum\dot{q}_{i}\frac{\partial T}{\partial\dot{q}_{i}}-T+U.
\end{align*}
$\frac{\partial T}{\partial q_{i}}=\sum_{k}A_{ik}\dot{q}+\sum_{j}A_{ij}\dot{q}_{j}\implies\sum\dot{q}_{i}\frac{\partial T}{\partial q_{i}}=2T$;
\begin{align*}
H & =T+U.
\end{align*}
We see that in this example, the Hamiltonian is the energy only because
\begin{align}
\frac{\partial L}{\partial t} & =0,\label{1}\\
U & =U(q),\label{2}\\
T & =T(q)\ \mbox{is quadratic}.\label{3}
\end{align}


\textbf{Example. }Consider a planar system with a bead confined to
move only along a hoop of radius $r$ whose center is attached to
the end of a stick of length $R$. The stick is rotated with a constant
angular velocity $\omega$ as shown in the picture. There is no gravity.

\begin{center}
\includegraphics[scale=0.75]{/Users/tom/Dropbox/Images/L9}
\par\end{center}

\begin{flushleft}
\begin{align*}
(x,y) & =\left(R\cos(\omega t)+r\cos\left(\omega t+\theta\right),R\sin(\omega t)+r\sin\left(\omega t+\theta\right)\right)\\
\implies T & =\frac{1}{2}m\left[R^{2}\omega^{2}+r^{2}\left(\omega+\theta\right)^{2}+2R\omega\left(\omega+\dot{\theta}\right)\cos\theta\right]\\
\implies\ddot{\theta}+\frac{\omega^{2}R}{r}\sin\theta & =0,
\end{align*}
which is the equation of motion of the physical pendulum even if there
is no gravity here. The reason is that an observer sitting at the
end of the stick would feel a fictitious centrifugal force outward
having the magnitude of $m\omega^{2}R\sin\theta$.
\par\end{flushleft}

Another important observation is that $H$ is not the energy since
apparently $T$ is not quadratic. $\frac{\partial H}{\partial t}=0$,
but the energy $\frac{1}{2}I\omega^{2}$ is not conserved since to
keep the angular momentum $L$ constant while $\omega$ is also constant,
the moment of inertia $I$ has to vary in time.


\subsection{Liouvillian}

Define the Liouvillian $\mathcal{L}=\left(\frac{\partial H}{\partial p}\frac{\partial}{\partial q}-\frac{\partial H}{\partial q}\frac{\partial}{\partial p}\right)$.
Then the Hamilton's equations of motion become
\begin{align*}
\dot{p}=\mathcal{L}(p), & \dot{q}=\mathcal{L}(q).
\end{align*}
Now we can write the time evolution of a Hamiltonian system in terms
of a linear map that maps the initial $q(0)$ and $p(0)$ to $q(t)$
and $p(t)$ at later times analogous to the time evolution by a unitary
in quantum mechanics.
\begin{align*}
q(t) & =q(0)+\int_{0}^{t}dt^{'}\mathcal{L}\left(t^{'}\right)q\left(t^{'}\right)\\
 & =q(0)+\int_{0}^{t}dt^{'}\mathcal{L}\left(t^{'}\right)\left(q(0)+\int_{0}^{t^{'}}dt^{''}\mathcal{L}\left(t^{''}\right)q\left(t^{''}\right)\right)+...\\
 & =\left[1+\underbrace{\int_{0}^{t}dt^{'}\mathcal{L}\left(t^{'}\right)}_{\mathcal{L}}+\underbrace{\int_{0}^{t}dt^{'}\mathcal{L}\left(t^{'}\right)\int_{0}^{t^{'}}dt^{''}\mathcal{L}\left(t^{''}\right)}_{\frac{t^{2}}{2!}\mathcal{L}^{2}}+...\right]q(0)\\
 & =e^{\mathcal{L}t}q(0),
\end{align*}
and similarly,
\begin{align*}
p(t) & =e^{\mathcal{L}t}p(0).
\end{align*}
\textbf{Example. }$H=\frac{p^{2}}{2m}-kq$.
\begin{align*}
\mathcal{L} & =\frac{p}{m}\frac{\partial}{\partial q}+k\frac{\partial}{\partial p}.
\end{align*}
$\mathcal{L}(p)=k$, $\mathcal{L}^{2}(p)=\mathcal{L}\left(\mathcal{L}(p)\right)=\mathcal{L}k=0$.
So the series terminates and we get that
\begin{align*}
p(t) & =e^{\mathcal{L}t}(p)\Bigr|_{0}=p(0)+kt.
\end{align*}
$\mathcal{L}(q)=\frac{p}{m}$, $\mathcal{L}^{2}(q)=\frac{k}{m}$,
$\mathcal{L}^{3}(q)=0$. So
\begin{align*}
q(t) & =e^{\mathcal{L}t}(q)\Bigr|_{0}=q(0)+\frac{p(0)}{m}t+\frac{k}{2m}t^{2}.
\end{align*}
We can go further and write the Liouvillian as a superoperator similar
to the time evolution $A(t)=U^{\dagger}A(0)U$ of an operator $A$
in the Heisenberg picture.


\subsection{Coordinate transformation}

A mass on a spring is attached to a car moving with velocity $v$
such that $H=\frac{p^{2}}{2m}+\frac{1}{2}k(x-vt)^{2}.$ Suppose that
we want to simplify the problem by going into the moving frame and
define $X=x-vt$. What is the conjugate momentum $P$?

Go back to the Lagrangian;
\begin{align*}
L & =\frac{1}{2}m\dot{x}^{2}-\frac{1}{2}k(x-vt)^{2}\\
X & =x-vt\\
\dot{X} & =\dot{x}-v.
\end{align*}
The new Lagrangian is
\begin{align*}
\tilde{L} & =\frac{1}{2}m\left(\dot{X}+v\right)^{2}-\frac{1}{2}kX^{2}.
\end{align*}
The new momentum is
\begin{align*}
P=\frac{\partial\tilde{L}}{\partial\dot{X}} & =m\left(\dot{X}+v\right).
\end{align*}
Then
\begin{align*}
\tilde{H}=P\dot{X}-\tilde{L} & =\frac{\left(P-mv\right)^{2}}{2m}+\frac{1}{2}kX^{2}-\underbrace{\frac{1}{2}mv^{2}}_{\mbox{constant}}.
\end{align*}
Without going back to the Lagrangian, we might be tempt to simply
write
\begin{align*}
P & =m\dot{X}=p-mv\\
\implies\frac{p^{2}}{2m} & =\frac{\left(P+mv\right)^{2}}{2m}.
\end{align*}
But that would be off by a sign.


\subsubsection{Action Principle for the Hamiltonian}

In searching for a set of transformations that preserve Hamilton's
equations of motion, we go back to the action principle. Can the action
principle be formulated in terms of the Hamiltonian alone? The answer
is yes. Let the action be
\begin{align*}
S & =\int_{t_{1}}^{t_{2}}dt\left(p\dot{q}-H(q,p,t)\right).\\
0=\delta S & =\int_{t_{1}}^{t_{2}}dt\left(\delta p\dot{q}+\underbrace{p\delta\dot{q}}_{\frac{d}{dt}(p\delta q)-\dot{p}\delta q}-\frac{\partial H}{\partial q}\delta q-\frac{\partial H}{\partial p}\delta p\right).
\end{align*}
Assume that $\delta q$ and $\delta p$ are varied independently.
\begin{align*}
0 & =p\underbrace{\delta q\Big|_{t_{1}}^{t_{2}}}_{0}+\int_{t_{1}}^{t_{2}}dt\left(\delta p\left(\dot{q}-\frac{\partial H}{\partial p}\right)-\left(\dot{p}+\frac{\partial H}{\partial q}\right)\delta q\right).
\end{align*}
The demand that the action is extremized is equivalent to Hamilton's
equations
\begin{align*}
\dot{q}-\frac{\partial H}{\partial p} & =0,\\
\dot{p}+\frac{\partial H}{\partial q} & =0.
\end{align*}
Curiously, this does not required the variation in the nomentum at
the end points to vanish, although we will assume that from now on.
Then for any scalar function $F(q,p,t)$, the action can be modified
\begin{align*}
S & =\int_{t_{1}}^{t_{2}}dt\left(p\dot{q}-H(q,p,t)+\frac{dF}{dt}\right).
\end{align*}



\subsubsection{Canonical transformations}

It is clear from the previous section that for any new Hamiltonian
$\mathcal{H}(Q,P,T)$ to give Hamilton's equations, we need

\begin{align*}
dt\left(p\dot{q}-H(q,p,t)\right) & =dt\left(P\dot{Q}-\mathcal{H}(Q,P,t)+\frac{dF(q,p,Q,P,t)}{dt}\right)\\
pdq-Hdt-PdQ & =\mathcal{H}dt+dF,
\end{align*}
where $F(q,p,Q,P,t)$ is the \emph{generating function }for a canonical
transformation (preserving Hamilton's equations).

We define four types of transformations:

Type 1:\textbf{ }$F=F(q,Q)$

Type 2: $F=F(q,P)$

Type 3: $F=F(p,Q)$

Type 4: $F=F(p,P)$

\textbf{Example. }For type 1 transformation $F=qQ$,
\begin{align*}
pdq-Hdt-PdQ & =\mathcal{H}dt+\frac{\partial F}{\partial q}dq+\frac{\partial F}{\partial Q}dQ=\mathcal{H}dt+Qdq+qdQ
\end{align*}
\begin{align*}
 & \implies\begin{cases}
q & =-P,\\
p & =Q.
\end{cases}
\end{align*}


\textbf{Example. }A type 1 transformation of a harmonic oscillator
$H=\frac{p^{2}}{2m}+\frac{1}{2}m\omega^{2}q^{2}$. It would be nice
if we could transform the Hamiltonian such that it has no $Q$ dependence.
That is, 
\begin{align*}
H & =\frac{f^{2}(p)}{2m}.
\end{align*}
We therefore consider this transformation
\begin{align*}
p & =f(p)\sin Q,\\
q & =\frac{f(p)}{m\omega}\cos Q,
\end{align*}
and find the generating function $F$.
\begin{align*}
\frac{p}{q} & =m\omega\tan Q\\
p=\left(\frac{\partial F}{\partial q}\right)_{Q} & =m\omega q\tan Q\\
\implies F & =\frac{m\omega q^{2}}{2}\tan Q+g(Q)
\end{align*}
Choose $g(Q)=0$. Then
\begin{align*}
P=-\left(\frac{\partial F}{\partial Q}\right)_{q} & =-\frac{m\omega q^{2}}{2}\sec^{2}Q\\
 & =-\frac{f^{2}(p)}{2m\omega}\cos^{2}Q\sec^{2}Q\\
\implies f(p) & =i\sqrt{2m\omega P}
\end{align*}
\begin{align*}
\implies & \boxed{H=-\omega P}.
\end{align*}
The Hamiltonian becomes that of the free particle with $P=P(0)$ and
$Q=Q(0)-\omega t$! Say, in the study of fluctuation-dissipation relation,
if you want to phenomenologically put friction in the system, it is
much easier to put friction in at this level after you have transformed
the oscillator into a free particle.

\textbf{Example. }A type 2 transformation of a harmonic oscillator
$H=\frac{p^{2}}{2m}+\frac{1}{2}m\omega^{2}q^{2}$.

Let $P$ be the ``ladder operators.''
\begin{align*}
P & =\frac{p+im\omega q}{\sqrt{2}}
\end{align*}
Look for $F(q,P)$.
\begin{align*}
pdq-Hdt+QdP & =\mathcal{H}dt+\frac{\partial F}{\partial q}dq+\frac{\partial F}{\partial P}dP\\
p=\left(\frac{\partial F}{\partial q}\right)_{P} & =\sqrt{2}P-im\omega q\\
\implies F & =\sqrt{2}Pq-\frac{im\omega q^{2}}{2}+g(P)\\
Q=\left(\frac{\partial F}{\partial P}\right)_{q} & =\sqrt{2}q+g^{'}(P).
\end{align*}
Choose $g^{'}(P)=\frac{iP}{m\omega}$. Then
\begin{align*}
Q & =\sqrt{2}q+\frac{iP}{m\omega}=\sqrt{2}q+\frac{i}{m\omega}\left(\frac{p+im\omega q}{\sqrt{2}}\right)=\sqrt{2}q+\frac{ip}{\sqrt{2}m\omega}-\frac{q}{\sqrt{2}}=\frac{q+\frac{ip}{m\omega}}{\sqrt{2}}.
\end{align*}
(This is wrong but it was from the lecture. Anyway, assuming the result,
the Hamiltonian is)
\begin{align*}
 & \boxed{\mathcal{H}=i\omega PQ},
\end{align*}
which is the analog of $H=\omega a^{\dagger}a$ in quantum mechanics,
and $P$ and $Q$, like the ladder operators, evolve trivially in
time.
\begin{align*}
P(t)=P(0)e^{-i\omega t}, & Q(t)=Q(0)e^{i\omega t}.
\end{align*}

%\end{document}
